% Makros

\newcommand{\N}{\mathbb{N}} % natuerliche Zahlen
\newcommand{\Z}{\mathbb{Z}} % ganze Zahlen
\newcommand{\Q}{\mathbb{Q}} % rationale Zahlen
\newcommand{\R}{\mathbb{R}} % reelle Zahlen
\newcommand{\K}{\mathbb{K}} % Körper
\newcommand{\co}{\mathrm{co}} % compact open
\newcommand{\W}{\mathcal{W}} % Weyl-Gruppe

\DeclareMathOperator{\Ret}{Re} % Realteil
%\newcommand{\PIT}{\textbf{PIT }} % Prime Ideal Theorem
\newcommand{\AC}{\textbf{AC}\xspace} % Prime Ideal Theorem
\newcommand{\UFT}{\textbf{UFT}\xspace} % Prime Ideal Theorem
\newcommand{\PIT}{\textbf{PIT}\xspace}
\newcommand{\ZF}{\textbf{ZF}\xspace}
\newcommand{\ZFC}{\textbf{ZFC}\xspace}
\newcommand{\CC}{\textbf{CC}\xspace}
\newcommand{\CCR}{\textbf{CC}($\R$)\xspace}
\newcommand{\PCC}{\textbf{PCC}\xspace}
\newcommand{\PCCR}{\textbf{PCC}($\R$)\xspace}

\newcommand{\grad}[1]{\gv{\nabla} #1} % for gradient
\let\divsymb=\div % rename builtin command \div to \divsymb
\renewcommand{\div}[1]{\mathrm{div\,} #1} % for divergence
\renewcommand{\d}[1]{\ensuremath\, {\operatorname{d}\!{#1}}}
\newcommand{\norm}[1]{\lVert #1 \rVert}

\newcommand{\banach}{\textsc{Banach}}
\newcommand{\cauchy}{\textsc{Cauchy}}
\newcommand{\euklid}{\textsc{Euklid}}
\newcommand{\fubini}{\textsc{Fubini}}
\newcommand{\hahn}{\textsc{Hahn}}
\newcommand{\helmholtz}{\textsc{Helmholtz}}
\newcommand{\hoelder}{\textsc{Hölder}}
\newcommand{\killing}{\textsc{Killing}}
\newcommand{\lebesgue}{\textsc{Lebesgue}}
\newcommand{\leibniz}{\textsc{Leibniz}}
\newcommand{\lie}{\textsc{Lie}}
\newcommand{\lipschitz}{\textsc{Lipschitz}}
\newcommand{\navier}{\textsc{navier}}
\newcommand{\poincare}{\textsc{Poincaré}}
\newcommand{\riesz}{\textsc{Riesz}}
\newcommand{\stokes}{\textsc{Stokes}}
\newcommand{\sobolev}{\textsc{Sobolev}}
\newcommand{\weyl}{\textsc{Weyl}}



% Umgebungen für Definitionen, Sätze, usw.

%\newtheorem{defn}{Definition}[section]
%\newtheorem{ex}{Beispiel}[chapter]

\theoremstyle{plain}
\newtheorem{thm}{Satz}[chapter]
\newtheorem{lem}[thm]{Lemma}
\newtheorem{cor}[thm]{Korollar}
\newtheorem{prop}[thm]{Proposition}

\theoremstyle{definition}
\newtheorem{defn}[thm]{Definition}

\theoremstyle{remark}
\newtheorem*{bem}{Bemerkung}
\newtheorem{bemnumber}[thm]{Bemerkung}

\def\Satzrefname{Satz}

\DeclareMathOperator{\spn}{span}
\DeclareMathOperator{\sign}{sign}
\DeclareMathOperator{\GL}{GL}
\DeclareMathOperator{\Sym}{Sym}
\DeclareMathOperator{\hgt}{ht}
\DeclareMathOperator{\dist}{dist}
\DeclareMathOperator{\loc}{loc}
\DeclareMathOperator{\supp}{supp}

%\renewcommand{\refname}{Literaturverzeichnis}
\addto\captionsngerman{%
\renewcommand*{\bibname}{Literaturverzeichnis}
}

%\makeglossaries

\pagestyle{fancy}
\fancypagestyle{MyStyle}{%
\fancyhf{} %Clean headers
\fancyhead[LE,RO]{\thepage}
\fancyhead[RE]{\normalfont \nouppercase{\leftmark}}
\fancyhead[LO]{\normalfont \nouppercase{\rightmark}}
\fancyfoot[C]{\tiny Version vom \today}
\renewcommand{\chaptermark}[1]{\markboth{\thechapter\ {\normalfont{##1}}}{}}
\renewcommand{\sectionmark}[1]{\markright{\thesection\ {\normalfont{##1}}}{}}
}

\fancypagestyle{plain}{%
\fancyhf{} % clear all header and footer fields
\fancyfoot[LE,RO]{\thepage} % except the center
\renewcommand{\headrulewidth}{0pt}
\renewcommand{\footrulewidth}{0pt}}
