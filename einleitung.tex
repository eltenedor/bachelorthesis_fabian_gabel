\chapter*{Einleitung}\index{Einleitung}
\addcontentsline{toc}{chapter}{Einleitung}\index{Einleitung}

\begin{addmargin}[2em]{2em}% 1em left, 2em right
  \textit{... as Sir Cyril Hinshelwood has observed ... fluid dynamicists
were divided into hydraulic engineers who observed things that
could not be explained and mathematicians who explained things
that could not be observed.} 
  \flushright(James Lighthill)
\end{addmargin}
\hspace{2cm}

Die akkurate Modellierung des Verhaltens \newton scher Fluide ist zentral für unzählige Anwendungen der Aerodynamik, Verbrennungsforschung oder chemischen Industrie.
Die Grundlage dafür bildet ein System partieller Differentialgleichungen, welches \navier\ und \stokes\ unabhängig voneinander in der ersten Hälfte des 19. Jahrhunderts einführten. 
Man bezeichnet dieses Gleichungssystem heute als \navier\hyp\stokes\hyp{}Gleichungen.
Sie umfassen im imkompressiblen dreidimensionalen Fall eine Gleichung zur Beschreibung der Massenerhaltung, die sogenannte Kontinuitätsgleichung, und für jede Raumrichtung eine Impulsgleichung:

\begin{align*}
  \div u &= 0, \\
  \frac{\partial u}{\partial t} - \nu \Delta u + u \cdot \nabla u + \nabla p &= f.
\end{align*}

Die Existenz und Eindeutigkeit klassischer glatter Lösungen dieses Gleichungssystems gehört nach dem \textsc{Clay}\hyp{}Institute zu einem der wichtigsten ungelösten mathematischen Probleme unseres Jahrtausends und ist daher in der Liste der Millenium Pobleme zu finden.

Auf der Suche nach Lösungen der \navier\hyp\stokes\hyp{}Gleichungen, unter der Annahme einschränkender Rand- und Anfangsbedingungen, schlägt man oft einen Umweg ein. 
Dieser führt zunächst auf sogenannte schwache Lösungen.
In einem zweiten Schritt wird dann nachgewiesen, dass die schwachen Lösungen über zusätzliche Regularitätseigenschaften verfügen, welche die Existenz einer Lösung im klassischen Sinne garantieren.

Im Rahmen der Suche nach schwachen Lösungen im \hilbert\hyp{}Raum $L^2(\Omega)^n$, wobei $\Omega$ ein Teilgebiet des $R^n$, $n \geq 2$, bezeichnet, findet auch das in dieser Arbeit vorgestellte Hilfsmittel Verwendung: die \helmholtz\hyp{}Zerlegung.
Dabei handelt es sich um eine orthogonale Zerlegung des Lösungsraumes, welche es unter anderem ermöglicht den neben dem gesuchten Geschwindigkeitsfeld $u$ unbekannten Druck $p$ vorerst aus dem Gleichungssystem zu eliminieren und so die Anzahl der Unbekannten zu reduzieren.
Die Anwendung der \helmholtz\hyp{}Zerlegung auf ein Element des Lösungsraumes zerlegt diesen in einen divergenzfreien und einen rotationsfreien Anteil.

Ziel dieser Arbeit ist es, aufbauend auf den im Grundstudium vermittelden Kenntnissen der Funktionentheorie, Integrationstheorie und Funktionalanalysis die Existenz und Eindeutigkeit der \helmholtz\hyp{}Zerlegung auf dem Raum $L^2(\Omega)^n$ zu beweisen.

Das erste Kapitel dient dazu die nötigen funktionalanalytischen Grundlagen bereitzustellen und die in der Arbeit verwendete Notation einzuführen.
Das Hauptaugenmerk dieses Kapitels liegt auf der Definition der schwachen Differenzierbarkeit und des damit einhergehenden Distributionsbegriffs.

Im zweiten Kapitel werden die zum Beweis der \helmholtz\hyp{}Zerlegung nötigen Hilfsaussagen bereitgestellt.
Das zentrale Resultat dieses Teils der Arbeit ist ein Kriterium, welches unter gewissen Bedingungen die schwache Lösbarkeit der Gradientengleichung $\nabla p = f$ liefert.

Kapitel drei behandelt schließlich die \helmholtz\hyp{}Zerlegung und eine Charakterisierung der Zerlegung $L^2(\Omega)^n = L^2_\sigma(\Omega) \oplus G_2(\Omega)$ für den Fall, dass $\Omega = \R^n$ gilt.

Das letzte Kapitel fasst nochmals die zentralen Resultate zusammen und gibt einen Ausblick auf Problemstellungen in denen die \helmholtz\hyp{}Zerlegung zum Einsatz kommt.
