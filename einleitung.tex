\chapter*{Einleitung}\index{Einleitung}
\addcontentsline{toc}{chapter}{Einleitung}\index{Einleitung}

Die akkurate Modellierung des Verhaltens von Flüssigkeiten und Gasen ist zentral für unzählige Anwendungen der Aerodynamik, Verbrennungsforschung oder chemischen Industrie.
Die Grundlage dafür bildet ein System partieller Differentialgleichung welches erstmals 1827 von \navier\ eingeführt wurde: die \navier\hyp{}\stokes\ Gleichungen. 
Sie umfassen im imkomplressiblen dreidimensionalen Fall eine Gleichung zur Beschreibung der Massenerhaltung, die sogenannte Kontinuitätsgleichung, und für jede Raumrichtung ein Impulsgleichung:

\begin{align*}
  \div u &= 0, \\
  \frac{\partial u}{\partial t} - \nu \Delta u + u \cdot \nabla u + \nabla p &= f,
\end{align*}

Die Existenz und Eindeutigkeit allgemeiner glatter Lösungen dieses Gleichungssystems gehört nach dem \textsc{Clay}\hyp{}Institute zu einem der wichtigsten ungelösten mathematischen Probleme unseres Jahrtausends und ist daher in der Liste der Millenium Pobleme zu finden.




\begin{itemize}
  \item Physikalische Motivation dieses Gleichungssystems
  \item Linearisierung der Navier-Stokes-Gleichungen (Motivation der Stokes-Gleichung)
  \item Schleichende Strömungen z.B. \cite[S.112,S.489]{spurk10stroemungslehre}. 
\end{itemize}

