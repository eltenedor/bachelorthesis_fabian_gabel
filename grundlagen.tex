\chapter{Grundlagen}
\label{cp:grundlagen}

\section{Physikalische Grundlagen}

\begin{itemize}
  \item Physikalische Motivation dieses Gleichungssystems
  \item Linearisierung der Navier-Stokes-Gleichungen (Motivation der Stokes-Gleichung)
  \item Schleichende Strömungen z.B. \cite{spurk10stroemungslehre}[S.112,S.489]. 
\end{itemize}

\section{Funktionalanalytische Grundlagen -- Distributionen und Sobolevräume}

Dieses Unterkapitel beschäftigt sich mit den für die kommenden Kapitel zentralen Funktionenräumen und dient zudem der Einführung der verwendeten Notation und Bildung der verwendeten Begriffe.
Die Notation ist an \cite{sohr2001navier} angelehnt.

%\subsection{Theorie des Hilbertraums}
%Hauptsächlich zur Einführung der Verwendeten Notation
\subsection{Glatte Funktionen und Glättungskerne}
\label{subsec:smoothMolly}

Ziel dieses Abschnittes ist es die nötigen Begriffe und Definitionen im Zusammenhang mit glatten Funktionen bereitzustellen.
Im Folgenden bezeichne $\Omega \subseteq \R^n$ stets ein nichtleeres Gebiet.

Sei $k \in \N$ so bezeichne $C^k(\Omega)$ die Menge aller Funktionen
$$
u \colon \Omega \to \R, \quad x \mapsto u(x),
$$
sodass alle partiellen Ableitungen $D^\alpha u$ existieren und stetig sind für alle Multiindices $\alpha \in N_0^n$ mit $0 \leq |\alpha| \leq k$. Mit
$$
  C^\infty(\Omega) := \bigcap_{k = 0}^\infty C^k(\Omega)
$$
bezeichnen wir den Raum der glatten Funktionen auf $\Omega$.

Im Kontext von $L^p$-Räumen spielt ein Funktionenraum eine wichtige Rolle bei der Approximation von $L^p$-Funktionen: der Raum der glatten Funktionen mit kompaktem Träger
$$
C_0^\infty := \{ u \in C^\infty(\Omega) \mid \supp u \text{ kompakt und } \supp u \subseteq \Omega\}.
$$
Wir werden zudem den Raum $C^\infty(\overline\Omega)$ aller Restriktionen $u|_{\overline\Omega}$ von Funktionen aus $C^\infty(\Omega)$ mit
\begin{displaymath}
  \sup_{|\alpha| < \infty, x \in \R^n} |D^\alpha u(x)| < \infty \tag{$\ast$}
\end{displaymath}
benötigen.
Unter der Voraussetzung aus ($\ast$) lässt sich der Raum $C^\infty(\overline\Omega)$ mit einer Norm ausstatten:
$$
\norm{u}_{C^\infty} = \norm{u}_{C^\infty(\overline\Omega} := \sup_{|\alpha|\leq k, x \in \overline\Omega} |D^\alpha u(x)|.
$$

Alle eingeführten Räume lassen sich auf natürliche Weise auf Räume von Vektorfeldern verallgemeinern.
Man erhält so
\begin{align*}
  C^\infty(\Omega)^m &:= \{(u_1,\dots,u_m) \mid u_j \in C^\infty(\Omega), j = 1,\dots,m\} \\
  C_0^\infty(\Omega)^m &:= \{(u_1,\dots,u_m) \mid u_j \in C_0^\infty(\Omega), j = 1,\dots,m\} \\
  C^\infty(\overline\Omega)^m &:= \{(u_1,\dots,u_m) \mid u_j \in C^\infty(\overline\Omega), j = 1,\dots,m\} 
\end{align*}
wobei der letzte Vektorraum durch die Norm
$$
\norm{u}_{C^\infty} = \norm{u}_{C^\infty(\overline\Omega)}^m := \sup_{j = 1,\dots, m} \norm{u_j}_{C^k(\overline\Omega)}
$$
zu einem normierten Vektorraum wird.

\begin{itemize}
  \item Andere Definition aus Adams vergleichen und Äquivalenz (?) beweisen.
\end{itemize}

Von besonderer Rolle für die Lösungstheorie der stationären inkompressiblen \navier\hyp{}\stokes\hyp{}Gleichungen ist der Raum
$$
C_{0,\sigma}^\infty(\Omega) := \{u \in C_0^\infty(\Omega)^n \mid \div u = 0\}
$$
der divergenzfreien Vektorfelder.

\begin{itemize}
  \item \cite{sohr2001navier}[Die Glättungsmethode S.64ff.]
  \item \cite{adams2003sobolev}[S.10, S.9(alt), S.36, S.30(alt)]
\end{itemize}

\subsection{Topologisierung des Raums der Testfunktionen und ein Stetigkeitsbegriff}

\begin{itemize}
  \item Inhalte ganz zu Beginn von \cite{sohr2001navier}[S.34] wiedergeben, zusätzliche (topologische Eigenschaften) beweisen, aus Werner s.u.
  \item \cite{werner2011fa}[S.430]
  \item Lemma VIII.5.1 (a)(d), VIII.2.3
  \item Satz VIII.5.4(iii)
  \item lokale Integrierbarkeit beide Definitionen und Äquivalenz
  \item Einbettung von $L^1_{\mathrm{loc}}$ in $C_0^\infty(\Omega)'$
\end{itemize}

\subsection{Differentiation von Distributionen -- Schwache Differenzierbarkeit und Sobolevräume}

\begin{itemize}
  \item \cite{sohr2001navier}[S.34ff.]
  \item \cite{werner2011fa}[S.433f.]
  \item Differentiation von Distributionen
  \item Divergenzfreie Test-Funktionen
  \item Sobolevräume und ihre Normen \cite{sohr2001navier}[S.38ff.]
\end{itemize}

