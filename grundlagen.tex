\chapter{Funktionalanalytische Grundlagen}
\label{cp:grundlagen}

Dieses Unterkapitel beschäftigt sich mit Definitionen und Eigenschaften der für die kommenden Kapitel zentralen Funktionenräume.
Die Notation orientiert sich an \cite{sohr2001navier}.
In Anlehnung an die für \hilbert\hyp{}Räume verbreitete Schreibweise $\langle \cdot, \cdot \rangle$ für das Skalarprodukt wollen wir nicht nur für den Raum $L^2$, sondern auch für alle anderen $L^q$\hyp{}Räume diese Symbolik verwenden.
Wir definieren also für geeignete Funktionen $f$ und $g$ den Ausdruck
$$
\langle f, g \rangle := \int f \cdot g \d x.
$$
Falls nicht anders vermerkt, bezeichnet $n$ immer eine von Null verschiedene natürliche Zahl.

\section{Glatte Funktionen und der Raum der Testfunktionen}
\label{subsec:smooth}

Ziel dieses Abschnittes ist es, die nötigen Begriffe und Definitionen im Zusammenhang mit glatten Funktionen bereitzustellen.
Im Folgenden bezeichne $\Omega \subseteq \R^n$ stets ein Gebiet, also eine nichtleere, offene und zusammenhängente Teilmenge.

Sei $k \in \N$. Mit $C^k(\Omega)$ bezeichnen wir die Menge aller Funktionen $u$, deren partielle Ableitungen $D^\alpha u$ für alle Multiindices $\alpha \in \N_0^n$ mit $0 \leq |\alpha| \leq k$ existieren und stetig sind. 
Mit
$$
  C^\infty(\Omega) := \bigcap_{k = 0}^\infty C^k(\Omega)
$$
bezeichnen wir den Raum der glatten Funktionen auf $\Omega$.

Im Kontext der Approximation von $L^q$-Funktionen spielt ein Funktionenraum eine wichtige Rolle: der Raum der glatten Funktionen mit kompaktem Träger
$$
C_0^\infty(\Omega) := \{ u \in C^\infty(\Omega) \mid \supp u \text{ kompakt und } \supp u \subseteq \Omega\}.
$$
Wir werden diesem Raum in Abschnitt \ref{subsec:distributionsSobolev} nochmals als Raum der Testfunktionen begegnen.
Zudem werden wir den Raum $C^\infty(\overline\Omega)$ aller Restriktionen $u|_{\overline\Omega}$ von Funktionen aus $C^\infty(\R^n)$ mit
\begin{displaymath}
  \sup_{\alpha  \in \N^n, x \in \R^n} |D^\alpha u(x)| < \infty 
\end{displaymath}
benötigen.
Aufgrund dieser Eigenschaft lässt sich der Raum $C^\infty(\overline\Omega)$ mit einer Norm ausstatten:
\begin{equation}
  \label{eq:cInftyNorm}
  \norm{u}_{C^\infty} := \norm{u}_{C^\infty(\overline\Omega)} := \sup_{|\alpha| \in \N^n, x \in \overline\Omega} |D^\alpha u(x)|.
\end{equation}

Alle eingeführten Räume lassen sich auf natürliche Weise zu Räumen von Vektor\-feldern verallgemeinern.
Man erhält so
\begin{align*}
  C^\infty(\Omega)^m &:= \{(u_1,\dots,u_m) \mid u_j \in C^\infty(\Omega), j = 1,\dots,m\}, \\
  C_0^\infty(\Omega)^m &:= \{(u_1,\dots,u_m) \mid u_j \in C_0^\infty(\Omega), j = 1,\dots,m\} \quad\text{und} \\
  C^\infty(\overline\Omega)^m &:= \{(u_1,\dots,u_m) \mid u_j \in C^\infty(\overline\Omega), j = 1,\dots,m\},
\end{align*}
wobei der letzte Vektorraum durch die Norm
$$
\norm{u}_{C^\infty} := \norm{u}_{C^\infty(\overline\Omega)}^m := \sup_{j = 1,\dots, m} \norm{u_j}_{C^\infty(\overline\Omega)}
$$
zu einem normierten Vektorraum wird.

In der Literatur findet man eine Reihe anderer Definitionen des Symbols $C^\infty(\overline\Omega)$.
Für beschränkte Gebiete stimmen sie jedoch alle überein, wie das folgende Lemma beweist.

\begin{lem}
  \label{lem:CInftyClosedOmega}
  Sei $\Omega \subseteq \R^n$ ein beschränktes Gebiet.
  Für eine Funktion $u \colon \overline\Omega \to \R$ schreiben wir:
  \begin{enumerate}[(a)]
    \item $u \in C^\infty(\overline\Omega)$,
      falls eine Funktion $f \in C^\infty(\R^n)$ mit $\sup_{\alpha \in \N^n, x \in \R^n} \norm{D^\alpha f(x)} <  \infty$ existiert mit $u = f|_{\overline\Omega}$.\cite[S.23, I.3.1]{sohr2001navier}

    \item $u \in C^\infty_{\bullet}(\overline\Omega)$,
      falls eine offene Obermenge $\overline\Omega \subseteq U \subseteq \R^n$ und eine glatte Fortsetzung $\tilde u \colon U \to \R$ von $u$ existieren.

    \item $ u \in C^\infty_{\bullet\bullet}(\overline\Omega)$,
    falls $u|_\Omega \in C^\infty(\Omega)$ gilt und sich für alle Multiindices $\alpha \in \N^n$ die Funktion $D^\alpha u$ stetig auf $\overline\Omega$ fortsetzen lässt.\cite[S.10, 1.28]{adams2003sobolev},\cite[S.35, II.1.3]{galdi2011navier}
  \end{enumerate}

  Dann gilt
  $$
    C^\infty(\overline\Omega) =
    C^\infty_{\bullet}(\overline\Omega) =
    C^\infty_{\bullet\bullet}(\overline\Omega).
  $$
\end{lem}

\begin{proof}
  %Zunächst ist $\overline\Omega$ als beschränkte abgeschlossene Teilmenge von $\R^n$ kompakt.
  Wir erhalten die Inklusion
  $$
  C^\infty(\overline\Omega) \subseteq C_{\bullet},
  $$ 
  da jedes $u\in C^\infty(\overline\Omega)$ bereits eine glatte Fortsetzung auf die offene Obermenge $\R^n$ besitzt.

  Die Inklusion $$C^\infty_{\bullet}(\overline\Omega) \subseteq C^\infty_{\bullet\bullet}(\overline\Omega)$$ folgt direkt aus der Definition.
  
  Die letzte Inklusion $$C^\infty_{\bullet\bullet}(\overline\Omega) \subseteq C^\infty(\overline\Omega)$$ erfordert mehr Arbeit.
  Sei dazu $u \in C^\infty_{\bullet\bullet}(\overline\Omega)$.
  %Es ist eine Eigenschaft von \hausdorff\hyp{}Räumen, dass stetige Abbildungen höchstens eine Fortsetzung auf den Abschluss ihres Definitionsberechs besitzen \cite[S.112, Lemma 4.2.4]{bartsch2015topologie}.
  %Wir wollen daher $u$ mit seiner Fortsetzung auf $\overline\Omega$ identifizieren.
  Der \emph{Fortsetzungssatz von \stein} \cite[S.172, Proposition 2.2]{stein1970singular} erlaubt es, die stetige Funktion $u$ ausgehend von der abgeschlossenen Menge $\overline\Omega$ auf ganz $\R^n$ zu einer Funktion $\tilde u$ fortzusetzen.
  Diese Fortsetzung besitzt jedoch im Allgemeinen keine beschränkten Ableitungen wie in der Definition von $C^\infty(\overline\Omega)$ gefordert.
  Wir werden daher $\tilde u$ außerhalb von $\overline\Omega$ geeignet modifizieren.
  Da $\Omega$ als beschränkt vorausgesetzt wurde, existiert ein offener Ball $B_r(0)$ mit Radius $r > 0$ und 
  $$
   \Omega \subset\subset B_r(0).
  $$ 
  Wir multiplizieren nun die Fortsetzung $\tilde u$ mit einer glatten \emph{cut-off}\hyp{}Funktion $\psi$, für die
  $$ 
  \psi(x) = \begin{cases}
                          1 &\text{ für alle } x \in \overline\Omega \\
                          0 &\text{ für alle } x \in \R^n \setminus B_r(0)
             \end{cases}
  $$
  und folglich $\supp \psi \subseteq B_r(0)$ gilt.
  Eine solche Funktion erhält man zum Beispiel aus der Faltung der charakteristischen Funktion $\chi_{\overline\Omega}$ mit einem Glättungskern, wie der Beweis des \urysohn\hyp{}Lemmas für glatte Funktionen zeigt. \cite[S.88, Proposition 6.5]{folland2009ra}
  Für $f := \tilde u \cdot \psi$ erhalten wir
  $$
  \sup_{|\alpha| < \infty, x \in \R^n}\norm{ D^\alpha f}
   = \sup_{|\alpha| < \infty, x \in \overline B_r}\norm{ D^\alpha f}
   < \infty,
  $$
  da $\overline B_r$ kompakt ist.
  Aufgrund der Fortsetzungseigenschaft gilt weiterhin $f|_{\overline\Omega} = u$ und damit ist $f$ die für (a) gesuchte Fortsetzung.
\end{proof}

Zuletzt erwähnen wir den Untervektorraum der divergenzfreien glatten Vektorfelder
$$
C_{0,\sigma}^\infty(\Omega) := \{u \in C_0^\infty(\Omega)^n \mid \div u = 0\},
$$
 welcher in der schwachen Lösungstheorie der stationären inkompressiblen \navier\hyp{}\stokes\hyp{}Gleichungen als Testraum eingesetzt wird.

\section{Schwache Differenzierbarkeit}
\label{subsec:distributionsSobolev}

Wir wenden uns nun einer Verallgemeinerung des Differenzierbarkeitsbegriffs zu, welcher für die Lösungstheorie partieller Differentialgleichungen von zentraler Bedeutung ist: die \emph{schwache} Differenzierbarkeit.
Es bezeichne im Folgenden wieder $\Omega \subseteq \R^n$ ein Gebiet mit $n \geq 1$.

\subsection{Lokal integrierbare Funktionen}

Wir werden im Folgenden oft eine Obermenge der bezüglich des \lebesgue\hyp{}Maßes auf $\R^n$ integrierbaren Funktionen verwenden:
Für $1 \leq q \leq \infty$ schreiben wir
$$
  u \in L^q_{\loc}(\Omega)
$$
und nennen $u$ \emph{lokal integrierbar}, falls $u \in L^q(B)$ für alle offenen Bälle $B \subseteq \Omega$ mit $\overline B \subseteq \Omega$ gilt.

\begin{bem}
  Eine Funktion $u$ ist genau dann lokal integrierbar über $\Omega$, falls $u \in L^q(K)$ für alle Kompakta $K \subseteq \Omega$ gilt.

  Diese Aussage findet man ebenfalls in der Literatur zur Definition lokaler Integrierbarkeit.
  Tatsächlich ist sie äquivalent zur obigen Definition.
  Denn einerseits ist für alle Bälle $B$ auch $\overline B$ ein Kompaktum und die Aussage folgt aus der Inklusionsbeziehung $L^q(\overline B) \subseteq L^q(B)$.
  Andererseits lässt sich jedes Kompaktum $K$ durch endlich viele Bälle $B_1, \dots, B_n$ mit $\overline B_i \subseteq \Omega, i =1,\dots,n,$ überdecken. 
  Gilt nun $u \in L^q(B_i)$ für alle $i = 1, \dots, n$, so gilt insbesondere $u \in L^q(\bigcup_{i = 1}^n B_i)$ und die Umkehrung der Aussage folgt aus der Inklusionsbeziehung $L^q(\bigcup_{i = 1}^n B_i) \subseteq L^q(K)$.
\end{bem}

Des Weiteren schreiben wir
$$
u \in L^q_{\loc}(\overline\Omega),
$$
falls $u \in L^q(B \cap \Omega)$ für jeden offenen Ball $B \subseteq \R^n$ mit $B \cap \Omega \neq \emptyset$ gilt.
Zusammenfassend gilt also die folgende Inklusionsbeziehung:
$$
L^q(\Omega) \subseteq L^q_{\loc}(\overline\Omega) \subseteq L^q_{\loc}(\Omega).
$$

\subsection{Distributionen}

In Abschnitt \ref{subsec:smooth} hatten wir bereits den Raum der Testfunktionen $C_0^\infty(\Omega)$ kennengelernt.
Wir werden uns vor allem für seinen Dualraum, den Raum der stetigen Funktionale auf $C_0^\infty(\Omega)$, interessieren.
Um überhaupt über Stetigkeit von Funktionalen auf $C_0^\infty(\Omega)$ sprechen zu können, müssen wir jedoch zunächst eine Topologie auf dem besagten Raum festlegen.
Dabei beschränken wir uns auf den aus dieser Topologie resultierenden Stetigkeitsbegriff:
Ein lineares Funktional $F \colon C_0^\infty(\Omega) \to \R$ ist stetig, genau dann, wenn für jedes beschränkte Teilgebiet $G \subseteq \Omega$ mit $\overline G \subseteq \Omega$ ein $k \in \N_0$ und $C = C(F,G)$ existiert, sodass
$$
|F(\varphi)| \leq C \norm{\varphi}_{C^k(\overline G)}
$$
gilt, wobei $\norm{\cdot}_{C^k(\overline G)}$ gerade die in Gleichung (\ref{eq:cInftyNorm}) definierte Norm bezeichnet.
Die hinter diesem Stetigkeitsbegriff stehende Topologie macht $C_0^\infty(\Omega)$ zu einem lokalkonvexen Vektorraum.
Eine ausführlichere Beschreibung der beschriebenen Topologie findet sich in \cite[S.433f.]{werner2011fa}.
Den Raum $C_0^\infty(\Omega)'$ aller im obigen Sinne stetigen Funktionale
$$
F \colon C_0^\infty(\Omega) \to \R, \quad \varphi \mapsto F(\varphi) =: [F, \varphi],
$$
bezeichnen wir als den Raum der \emph{Distributionen}.
Hierbei stellt $[\cdot,\cdot]$ die duale Paarung auf $\Omega$ dar. 

Wir betrachten nun spezielle Distributionen.
Ist $f \in L^1_{\loc}$, so verstehen wir unter 
$$
f \mapsto \langle f, \varphi \rangle := \int_\Omega f \varphi \d x, \quad \varphi \in C_0^\infty(\Omega), 
$$
die zu $f$ gehörige \emph{reguläre} Distribution.
Dass diese Zuordnung sogar injektiv ist und somit zur Inklusionsbeziehung $L^1_{\loc}(\Omega) \subseteq C^\infty_0(\Omega)'$ führt, lässt sich in \cite[S.432, Beispiel (a)]{werner2011fa} nachlesen.

\subsection{Differentiation von Distributionen}

Distributionen ermöglichen es, den für die Analysis grundlegenden Begriff der Differenzierbarkeit zu verallgemeinern.
Ist $\alpha \in \N_0^n$ ein Multiindex und $D^\alpha \colon C_0^\infty(\Omega) \to C_0^\infty(\Omega)$ der zugehörige Ableitungsoperator, so definieren wir für eine Distribution $F \in C_0^\infty(\Omega)'$ ihre distributionelle Ableitung $D^\alpha F \in C_0^\infty(\Omega)'$ durch
$$
  [D^\alpha F, \varphi] := (-1)^{|\alpha|} [F, D^\alpha \varphi].
$$
Bis auf ein Vorzeichen stimmt also die distributionelle Ableitung mit der zum Ableitungsoperator dualen Abbildung $(D^\alpha)' \colon C_0^\infty(\Omega)' \to C_0^\infty(\Omega)'$ überein.
Dass die distributionelle Ableitung wohldefiniert und sogar schwach*-stetig ist, lässt sich in \cite[S.434, Lemma VIII.5.7]{werner2011fa} nachlesen.

Im Unterabschnitt zu glatten Funktionen hatten wir die dortigen Definitionen auf Produkträume ausgeweitet.
Wir versehen den so entstandenen Produktraum $C_0^\infty(\Omega)^m$ mit der Produkttopologie und wenden uns nun dem zugehörigen Distributionenraum $(C_0^\infty(\Omega)^m)'$ zu.

Wir betrachten dazu die Distribution
$$
  F = (F_1, \dots, F_m), \quad F_j \in C_0^\infty(\Omega)', \quad j = 1, \dots, m,
$$
und definieren für alle 
$$\varphi = (\varphi_1, \dots, \varphi_m) \in C_0^\infty(\Omega)^m$$
die Distribution
\begin{equation}
  \label{eq:pairingVector}
  [F, \varphi] := [F_1, \varphi_1] + \dots + [F_m, \varphi_m].
\end{equation}
Durch Betrachtung von Funktionen des Typs $\psi^{(i)} = (0, \dots,0, \psi_i, 0,\dots,0) \in C_0^\infty(\Omega)^m$ mit $i \in \{1,\dots,m\}$ erkennen wir, dass jedes bezüglich der Produkttopologie stetige Funktional auf $C_0^\infty(\Omega)^m$ diese Form besitzt.
Für den Raum der Distributionen auf den Testfunktionen $C_0^\infty(\Omega)^m$ gilt also
\begin{align*}
  (C_0^\infty(\Omega)')^m 
  &= (C_0^\infty(\Omega)^m)' \\
  &= \{(F_1,\dots,F_m) \mid F_j \in C_0^\infty(\Omega)', j = 1, \dots,m\}.
\end{align*}

Wir erinnern uns an den im Unterabschnitt zu glatten Funktionen definierten Funktionenraum $C_{0, \sigma}^\infty(\Omega) \subseteq C_0^\infty(\Omega)^n$ der divergenzfreien Funktionen.
Nach dem Fortsetzungssatz von \hahn\hyp{}\banach\ für lokalkonvexe Vektorräume (\cite[S.408, Satz VIII.2.8]{werner2011fa}) erhalten wir die stetigen Funktionale auf $C_{0,\sigma}^\infty$ gerade als Einschränkungen der stetigen Funktionale auf $C_0^\infty(\Omega)^n$. 
Es gilt somit
$$
C_{0,\sigma}^\infty(\Omega)' = \{F|_{C_{0,\sigma}^\infty(\Omega)} \mid F \in (C_0^\infty(\Omega)^n)'\}.
$$

Betrachten wir nun den \hilbert\hyp{}Raum $L^2(\Omega)^n$.
Wie Lemma \ref{lem:mollification} noch zeigt, dass die glatten Funktionen mit kompaktem Träger dicht in $L^2(\Omega)$ liegen. 
Ebenso stellt der Produktraum $C_0^\infty(\Omega)^n$ einen dichten Unterraum von $L^2(\Omega)^n$ dar.
Um diese Tatsache zu imitieren, definieren wir den Unterraum 
$$
  L^2_\sigma(\Omega) 
  := \overline{C_{0,\sigma}^\infty(\Omega)}^{\norm{\cdot}_2}
  \subseteq L^2(\Omega)^n.
$$
Identifizieren wir nun jede Funktion $u \in L^2(\Omega)^n$ mit dem Funktional
$$
\langle u, \cdot \rangle \colon \varphi \mapsto \langle u, \varphi \rangle, \quad \varphi \in C_0^\infty(\Omega)^n,
$$
so erhalten wir die Einbettung 
$$
  L^2(\Omega)^n \subseteq (C_0^\infty(\Omega)^n)'.
$$
Ebenso lassen sich Funktionen $u \in L^2_\sigma(\Omega)$ mit der entsprechenden Einschränkung
$$
\langle u, \cdot \rangle \colon \varphi \mapsto \langle u, \varphi \rangle, \quad \varphi \in C_{0,\sigma}^\infty(\Omega)^n
$$
identifizieren, was die Einbettung
$$
  L^2_\sigma(\Omega) \subseteq (C_{0,\sigma}^\infty(\Omega)^n)'
$$
liefert.

\subsection{Sobolev-Räume}

Im Allgemeinen wird die Ableitung einer regulären Distribution nicht wieder regulär sein.
Wir definieren daher für alle $k \in \N$ und $1 \leq q \leq \infty$ den $L^q$\hyp{}\sobolev\hyp{}Raum $W^{k,q}(\Omega)$ der Ordnung $k$ durch
$$
W^{k,q}(\Omega) := \{ u \in L^q(\Omega) \mid D^\alpha u \in L^q(\Omega) \text{ für alle } |\alpha| \leq k\}.
$$
Für $u \in W^{k,q}(\Omega)$ bezeichnen wir $D^\alpha u \in L^q(\Omega)$ als \emph{schwache Ableitung} von $u$.
Hierbei identifizieren wir die reguläre Distribution $D^\alpha u$ immer direkt mit der korrespondierenden $L^q$\hyp{}Funktion.

Wir machen $W^{k,q}$ zu einem normierten Vektorraum durch die \sobolev\hyp{}Norm
$$
  \norm{u}_{W^{k,q}(\Omega)} 
  := \norm{u}_{W^{k,q}} 
  := \norm{u}_{k,q} 
  := \begin{cases} 
    \left(\sum_{|\alpha| \leq k} \norm{D^\alpha u}_q^q \right)^\frac{1}{q}\quad&\text{für } 1 \leq q < \infty \\
    \max_{|\alpha| \leq k} \norm{D^\alpha u}_\infty \quad&\text{für }q = \infty.
    \end{cases}
$$

\begin{lem}
  Für ein Gebiet $\Omega \subseteq \R^n$ mit $n \geq 1$ und $1 \leq q \leq \infty$ ist $W^{k,q}(\Omega)$ ausgestattet mit der \sobolev\hyp{}Norm ein \banach\hyp{}Raum.
\end{lem}

\begin{proof}
  Sei $(u_j)_{j \in \N}$ eine \cauchy\hyp{}Folge in $W^{k,q}(\Omega)$.
  Nach Definition ist dann auch $(D^\alpha u_j)_{j \in \N}$ für alle $|\alpha| \leq k$ eine \cauchy\hyp{}Folge in $L^q$.
  Folglich existieren für alle $|\alpha| \leq k$ Funktionen $u_\alpha \in L^q$ mit 
  $$
  \lim_{j \to \infty} \norm{D^\alpha u_j - u_\alpha}_q = 0.
  $$
  Sei $u_0 := u_{(0,\dots,0)}$.
  Wir behaupten nun, dass im distributionellen Sinne die Identität $u_\alpha = D^\alpha u_0$ für alle $\alpha \in \N^n$ gilt.
  Wir zeigen dazu, dass für alle $\varphi \in C_0^\infty(\Omega)$ die Identität
  $$
  [u_\alpha, \varphi] = [D^\alpha u_0, \varphi]
  $$
  gilt.
  Dazu halten wir zunächst fest, dass mit der \hoelder\hyp{}Ungleichung
  $$
  \int_\Omega |(D^\alpha u_j - u_\alpha)| \varphi \d x
  \leq \norm{(D^\alpha u_j - u_\alpha)}_q \norm{\varphi}_{q'} \to 0, \quad j \to \infty
  $$
  sowie
  $$
  \int_\Omega |(u_j - u_0) D^\alpha \varphi| \d x
  \leq \norm{(u_j - u_0)}_q \norm{D^\alpha \varphi}_{q'} \to 0, \quad j \to \infty
  $$
  folgen, wobei $q'$ den zu $q$ konjugierten Exponenten bezeichne.
  Hieraus folgen die Identitäten
  $$
  \lim_{j \to \infty} [D^\alpha u_j, \varphi] = \lim_{j \to \infty} \int_\Omega D^\alpha u_j \varphi \d x =  \int_\Omega u_\alpha \varphi \d x =  [u_\alpha, \varphi] 
  $$
  und
  $$
    \lim_{j \to \infty} [u_j, D^\alpha\varphi] = \lim_{j \to \infty} \int_\Omega u_j D^\alpha\varphi \d x =  \int_\Omega u_0 D^\alpha \varphi \d x = [u_0, D^\alpha \varphi].
  $$
  Dies impliziert schließlich
  $$ 
  [u_\alpha, \varphi] 
  = \lim_{j \to \infty} [D^\alpha u_j, \varphi] \\
  = \lim_{j \to \infty} (-1)^{|\alpha|}[u_j, D^\alpha \varphi] \\
  =  (-1)^{|\alpha|}[u_0, D^\alpha \varphi]
  = [D^\alpha u_0, \varphi].
  $$
  Daraus folgt jedoch gerade die Behauptung $D^\alpha u_0 = u_\alpha$.
  Insgesamt ergibt sich also $\lim_{j \to \infty} \norm{u_j - u_0}_{k,q} =  0$, was zu beweisen war.
\end{proof}

Wir definieren den Unterraum
$$
  W_0^{k,q}(\Omega) := \overline{C_0^\infty(\Omega)}^{\norm{\cdot}_{k,q}},
$$
welcher die glatten Funktionen mit kompaktem Träger als dichte Teilmenge bezüglich der \sobolev\hyp{}Norm enthält.

Bisher hatten wir nur \sobolev\hyp{}Räume positiver Ordnung betrachtet.
Wir definieren für $1 < q < \infty$ die \sobolev\hyp{}Räume negativer Ordnung 
$$
  W^{-k,q}(\Omega) := W_0^{k,q'}(\Omega)'.
$$
Wie für den Dualraum eines normierten Raumes üblich, ist $W^{-k,q}(\Omega)$ mit Operatornorm 
$$
\norm{F}_{W^{-k,q}(\Omega)} := \norm{F}_{W^{-k,q}} := \norm{F}_{-k,q} := \sup_{0 \neq \varphi \in C_0^\infty(\Omega)} \frac{|[F,\varphi]|}{\norm{\varphi}_{k,q'}}
$$
ein \banach\hyp{}Raum.
Wir beachten, dass die Operatornorm bereits durch die bezüglich \sobolev\hyp{}Norm in $W^{k,q}(\Omega)$ dichte Teilmenge $C_0^\infty(\Omega)$ eindeutig festgelegt ist.

Es ist möglich einen \sobolev\hyp{}Raum $W^{k,q}(\Omega)$ für $1 < q < \infty$ als abgeschlossenen Teilraum eines $L^q$\hyp{}Raumes zu realisieren \cite[S.61, 3.5]{adams2003sobolev}.
Als solcher ist er insbesondere reflexiv.
Wir halten diese wichtige Eigenschaft in einem Lemma fest.

\begin{lem}
  Für ein Gebiet $\Omega \subseteq \R^n$ mit $n \geq 1$ und $1 < q < \infty$ ist $W^{k,q} = W^{k,q}(\Omega)$ ausgestattet mit der \sobolev\hyp{}Norm ein reflexiver Raum.
\end{lem}

Aus der Reflexivität der \sobolev\hyp{}Räume leiten wir für $1 < q < \infty$ die folgende Identität ab:
$$
  W_0^{k,q'}(\Omega) = W^{-k,q}(\Omega)'.
$$
Wir können also jede Funktion $u \in W_0^{k,q'}(\Omega)$ mit dem Funktional
$$
[\cdot, u] \colon F \mapsto [F, u], \quad F \in W^{-1,q}(\Omega)
$$
identifizieren.

%Wir definieren nun die $W^{k,q}_{\loc}$\hyp{}Räume.
Wir bezeichnen mit $W^{k,q}_{\loc}(\Omega)$ den Raum aller Funktionen $u$ mit $D^\alpha u \in L^q_{\loc}(\Omega)$ für alle Multiindices $\alpha \in \N_0^n$ mit $|\alpha| \leq k$.
Des Weiteren definieren wir $W^{k,q}(\overline\Omega)$ als den Raum aller Funktionen $u$ mit $D^\alpha u \in L^q_{\loc}(\overline\Omega)$.
Nun zu den \sobolev\hyp{}Räumen negativer Ordnung. 
Wir definieren den Raum $W^{-k,q}_{\loc}(\Omega)$ als den Raum aller Distributionen
$$
F \colon \varphi \mapsto [F,\varphi], \quad \varphi \in C_0^\infty(\Omega),
$$
sodass für die Operatornorm die Ungleichung
\begin{equation}
  \label{eq:operatornorm}
\norm{F}_{-k,q} := \sup_{0 \neq \varphi \in C_0^\infty(\Omega_0)}  \frac{|[F,\varphi]|}{\norm{\varphi}_{k,q'}} < \infty
\end{equation}
für alle beschränkten Teilgebiete $\Omega_0 \subseteq \Omega$ mit $\overline\Omega_0 \subseteq \Omega$ gilt.

Wie schon im Unterabschnitt zu glatten Funktionen vorgestellt, können wir auch die Definition der \sobolev\hyp{}Räume auf Produkträume ausdehnen. 
Wir definieren hierzu
$$
W^{k,q}(\Omega)^m := \{ (u_1,\dots,u_m) \mid u_j \in W^{k,q}(\Omega), \quad j = 1,\dots, m\}
$$
und versehen den so entstandenen Vektorraum mit der Norm
$$
\norm{u}_{W^{k,q}(\Omega)^m} 
:= \norm{u}_{k,q}
:= \left( \sum_{j = 1}^m \norm{u_j}_{k,q}^q \right)^{\frac{1}{q}},
$$
wodurch dieser zu einem \banach\hyp{}Raum wird.


\newpage
\section{Eigenschaften der Faltung}
\label{subsec:mollification}

Aus der Integrationstheorie ist bekannt, dass sich $L^q$-Funktionen durch Faltungen mit Glättungskernen $\mathcal{F}_\varepsilon$ approximieren lassen.
Für unsere Zwecke wird es dafür ausreichend sein den Standardkern
$$
\mathcal{F}(x) := \begin{cases}
                     & c \cdot \exp\left(-\frac{1}{1 - \norm{x}}\right), \quad \norm{x} < 1 \\
                     & 0, \quad  \text{sonst}
                  \end{cases} \quad,\quad x \in \R^n
$$
zu betrachten und
$$
\mathcal{F}_\varepsilon(x) := \frac{1}{\varepsilon^n} \mathcal{F}\left(\frac{x}{\varepsilon}\right)
$$ 
zu setzen.
Hierbei sei $c$ so gewählt, dass $\int_{B_1(0)} \mathcal{F} \d x = 1$ gilt.

Folgendes Lemma fasst nochmals die Approximationseigenschaft von Glättungskernen zusammen. Ein Beweis findet sich in \cite[S.36, Theorem 2.29(c)]{adams2003sobolev}.
\begin{lem}
  \label{lem:mollification}
  Sei $\Omega \subseteq \R^n$ mit $n \geq 1$ ein Gebiet und $1 \leq q < \infty$ sowie $\varepsilon > 0$.
  Für alle $u \in L^q(\Omega)$ gilt dann
  $$
  \norm{(\mathcal{F}_\varepsilon \ast u)}_{L^q(\Omega)} \leq  \norm{u}_{L^q(\Omega)}
  $$
  und
  \begin{displaymath}
    \lim_{\varepsilon \to 0} \norm{(\mathcal{F}_\varepsilon \ast u) - u}_{L^q(\Omega)} = 0. 
  \end{displaymath}
\end{lem}

Sei nun $\Omega_0 \subseteq \overline\Omega_0 \subseteq \Omega$ ein beschränktes Teilgebiet und
\begin{equation}
  \label{eq:convolutionEpsilon}
  0 < \varepsilon < \dist(\Omega_0, \partial\Omega).
\end{equation}
Zusätzlich sei $u \in L^1_{\loc}(\Omega)$.
Setzt man $u(x) := 0$ für alle $x \in \R^n \setminus \Omega$, so folgt $u \in L^1_{\loc}(\R^n)$.
Wie \cite[S.171, Theorem 6.30(b)]{rudin1991fa} zeigt, lassen sich aus der Integrationstheorie bekannte Eigenschaften der Glättungen auch auf Faltungen 
$$
u^\varepsilon := \mathcal{F}_\varepsilon \ast u
$$
von Glättungskernen $\mathcal{F}_\varepsilon \ast u$ mit Distributionen $u \in C_0^\infty(\Omega)'$ verallgemeinern.
Einerseits zählt dazu, dass $u^\varepsilon \in C^\infty(\R^n)$ gilt.
Da $\Omega_0$ als beschränkt vorausgesetzt wurde, gilt nach Lemma \ref{lem:CInftyClosedOmega}(b$\Rightarrow$a) sogar $u^\varepsilon \in C^\infty(\overline\Omega_0)$.
Andererseits ist für alle Multiindices $\alpha \in \N_0^n$ und alle $x \in \Omega_0$ die folgende Gleichung gültig:
\begin{equation}
  \label{eq:convolutionDiff}
  (D^\alpha u^\varepsilon)(x) 
  = (\mathcal{F}_\varepsilon \ast (D^\alpha u))(x)
  = ( (D^\alpha \mathcal{F}_\varepsilon) \ast u)(x), \quad x \in \Omega_0.
\end{equation}

\begin{lem}
  Sei $\Omega \subseteq \R^n$ mit $n \geq 1$ ein Gebiet und $1 \leq q < \infty$.
  Ist $u \in L^1_{\loc}(\Omega)$ und gilt $\nabla u = 0$ im distributionellen Sinne, dann ist $u$ konstant.
\end{lem}

\begin{proof}
  Für alle beschränkten Teilgebiete $\Omega_0 \subseteq \Omega$ und $\varepsilon$ wie in Ungleichung (\ref{eq:convolutionEpsilon}) folgern wir unter Verwendung der Gleichung (\ref{eq:convolutionDiff})
  \begin{align*}
    \nabla u^\varepsilon
    &= (D_1 (\mathcal{F}_\varepsilon \ast u), \dots, D_n (\mathcal{F}_\varepsilon \ast u))^T \\
    &= ( \mathcal{F}_\varepsilon \ast D_1 u, \dots, \mathcal{F}_\varepsilon \ast D_n u)^T \\
    &= (0, \dots, 0)^T.
  \end{align*}
  Es ist $u^\varepsilon$ eine glatte Funktion.
  Zudem ist $\Omega_0$ als offene zusammenhängende Teilmenge von $\R^n$ insbesondere wegzusammenhängend.
  Ausgehend von der Integraldarstellung des Funktionszuwachses \cite[S.57, (11)]{koenigsberger2004ana2} gilt $u_\varepsilon = C_\varepsilon$ auf ganz $\Omega_0$.
  Mit Lemma \ref{lem:mollification} folgt nun die Netzkonvergenz $C_\varepsilon \to C$ für $\varepsilon \to 0$ auf $\Omega_0$.
  Wir greifen auf eine im Abschnitt \ref{sec:lipschitzDomains} zu \lipschitz\hyp{}Gebieten eingeführte Konstruktion vor:
  Aufgrund von Lemma \ref{lem:lipschitzExhaust} lässt sich das Gebiet $\Omega$ von einer Folge $(\Omega_j)_{j \in \N}$ offener beschränkter \lipschitz\hyp{}Gebiete ausschöpfen. 
  Durch Anwendung des für das Gebiet $\Omega_0$ beschriebenen Arguments auf die Folgenglieder $\Omega_j$ erhalten wir somit $u = C$ auf ganz $\Omega$. 
  Hieraus folgt die Behauptung.
\end{proof}


