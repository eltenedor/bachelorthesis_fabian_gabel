\chapter{Grundlagen}
\label{cp:grundlagen}

\section{Physikalische Grundlagen}

\begin{itemize}
  \item Physikalische Motivation dieses Gleichungssystems
  \item Linearisierung der Navier-Stokes-Gleichungen (Motivation der Stokes-Gleichung)
  \item Schleichende Strömungen z.B. \cite{spurk10stroemungslehre}[S.112,S.489]. 
\end{itemize}

\section{Funktionalanalytische Grundlagen -- Distributionen und Sobolevräume}

Dieses Unterkapitel beschäftigt sich mit den für die kommenden Kapitel zentralen Funktionenräumen und dient zudem der Einführung der verwendeten Notation und Bildung der verwendeten Begriffe.
Die Notation ist an \cite{sohr2001navier} angelehnt.

%\subsection{Theorie des Hilbertraums}
%Hauptsächlich zur Einführung der Verwendeten Notation
\subsection{Glatte Funktionen und Glättungskerne}
\label{subsec:smoothMolly}

Ziel dieses Abschnittes ist es die nötigen Begriffe und Definitionen im Zusammenhang mit glatten Funktionen bereitzustellen.
Im Folgenden bezeichne $\Omega \subseteq \R^n$ stets ein nichtleeres Gebiet.

Sei $k \in \N$ und es bezeichne $C^k(\Omega)$ die Menge aller Funktionen
$$
u \colon \Omega \to \R, \quad x \mapsto u(x),
$$
sodass alle partiellen Ableitungen $D^\alpha u$ existieren und stetig sind für alle Multiindices $\alpha \in N_0^n$ mit $0 \leq |\alpha| \leq k$. Mit
$$
  C^\infty(\Omega) := \bigcap_{k = 0}^\infty C^k(\Omega)
$$
bezeichnen wir den Raum der glatten Funktionen auf $\Omega$.

Im Kontext der Approximation von $L^p$-Funktionen spielt ein Funktionenraum eine wichtige Rolle: der Raum der glatten Funktionen mit kompaktem Träger
$$
C_0^\infty(\Omega) := \{ u \in C^\infty(\Omega) \mid \supp u \text{ kompakt und } \supp u \subseteq \Omega\}.
$$
Wir werden zudem den Raum $C^\infty(\overline\Omega)$ aller Restriktionen $u|_{\overline\Omega}$ von Funktionen aus $C^\infty(\R^n)$ mit
\begin{displaymath}
  \sup_{|\alpha| < \infty, x \in \R^n} |D^\alpha u(x)| < \infty \tag{$\ast$}
\end{displaymath}
benötigen.
Unter der Voraussetzung aus ($\ast$) lässt sich der Raum $C^\infty(\overline\Omega)$ mit einer Norm ausstatten:
$$
\norm{u}_{C^\infty} = \norm{u}_{C^\infty(\overline\Omega} := \sup_{|\alpha|\leq k, x \in \overline\Omega} |D^\alpha u(x)|.
$$

Alle eingeführten Räume lassen sich auf natürliche Weise zu Räumen von Vektorfeldern verallgemeinern.
Man erhält so
\begin{align*}
  C^\infty(\Omega)^m &:= \{(u_1,\dots,u_m) \mid u_j \in C^\infty(\Omega), j = 1,\dots,m\}, \\
  C_0^\infty(\Omega)^m &:= \{(u_1,\dots,u_m) \mid u_j \in C_0^\infty(\Omega), j = 1,\dots,m\} \quad\text{und} \\
  C^\infty(\overline\Omega)^m &:= \{(u_1,\dots,u_m) \mid u_j \in C^\infty(\overline\Omega), j = 1,\dots,m\},
\end{align*}
wobei der letzte Vektorraum durch die Norm
$$
\norm{u}_{C^\infty} = \norm{u}_{C^\infty(\overline\Omega)}^m := \sup_{j = 1,\dots, m} \norm{u_j}_{C^k(\overline\Omega)}
$$
zu einem normierten Vektorraum wird.

\begin{itemize}
  \item Andere Definition aus Adams vergleichen und Äquivalenz (?) beweisen.
\end{itemize}

Von besonderer Rolle für die Lösungstheorie der stationären inkompressiblen \navier\hyp{}\stokes\hyp{}Gleichungen ist der Untervektorraum
$$
C_{0,\sigma}^\infty(\Omega) := \{u \in C_0^\infty(\Omega)^n \mid \div u = 0\}
$$
der divergenzfreien Vektorfelder.

Aus der Integrationstheorie ist bekannt, dass sich $L^q$-Funktionen durch Faltungen mit Glättungskernen approximieren lassen.

\begin{lem}
  Sei $\Omega \subseteq \R^n$ mit $n \geq 1$ ein Gebiet und $1 \leq q < \infty$ sowie $\varepsilon > 0$.
  Für alle $u \in L^q(\Omega)$ gilt dann
  $$
  \norm{(\mathcal{F}_\varepsilon \ast u)}_{L^q(\Omega)} \leq  \norm{u}_{L^q(\Omega)}
  $$
  und
  \begin{displaymath}
    \lim_{\varepsilon \to 0} \norm{(\mathcal{F}_\varepsilon \ast u) - u}_{L^q(\Omega)} = 0. 
  \end{displaymath}
\end{lem}

Sei nun $\Omega_0 \subseteq \overline\Omega_0 \subseteq \Omega$ ein beschränktes Teilgebiet und
$$
0 < \varepsilon < \dist(\Omega_0, \partial\Omega).
$$
Zusätzlich sei $u \in L^1_{\loc}(\Omega)$.
Setzt man $u(x) := 0$ für alle $x \in \R^n \setminus \Omega$ so gilt $u \in L^1_{\loc}(\R^n)$.
Ein Resultat aus der Integrationstheorie besagt nun dass für die Glättung
$$
u^\varepsilon := \mathcal{F}_\varepsilon \ast u
$$
auch $u^\varepsilon \in C^\infty(\R^n)$ und da $\Omega_0$ als beschränkt vorausgestzt wurde sogar $u^\varepsilon \in C^\infty(\overline\Omega_0)$ gilt.
Wie \cite{rudin1991fa}[S.171, Theorem 6.30(b)] zeigt, gilt dies sogar noch unter der Voraussetzung, dass $u$ eine Distribution ist.
Zudem ist für alle Multiindices $\alpha$ die folgende Gleichung gültig:
\begin{equation}
  (D^\alpha u^\varepsilon)(x) 
  = (\mathcal{F}_\varepsilon \ast (D^\alpha u))(x)
  = ( (D^\alpha \mathcal{F}_\varepsilon) \ast u)(x), \quad x \in \Omega_0.
\end{equation}



\begin{itemize}
  \item \cite{sohr2001navier}[Die Glättungsmethode S.64ff.]
  \item \cite{adams2003sobolev}[S.10, S.9(alt), S.36, S.30(alt)]
\end{itemize}

\subsection{Topologisierung des Raums der Testfunktionen und ein Stetigkeitsbegriff}

\begin{itemize}
  \item Inhalte ganz zu Beginn von \cite{sohr2001navier}[S.34] wiedergeben, zusätzliche (topologische Eigenschaften) beweisen, aus Werner s.u.
  \item \cite{werner2011fa}[S.430]
  \item Lemma VIII.5.1 (a)(d), VIII.2.3
  \item Satz VIII.5.4(iii)
  \item lokale Integrierbarkeit beide Definitionen und Äquivalenz
  \item Einbettung von $L^1_{\mathrm{loc}}$ in $C_0^\infty(\Omega)'$
\end{itemize}

\subsubsection{Lokal integrierbare Funktionen}

Sei $\Omega \subseteq \R^n$ ein Gebiet. Für $1 \leq q \leq \infty$ schreiben wir
$$
  u \in L^q_{\loc}(\Omega),
$$
und nennen $u$ \emph{lokal integrierbar}, falls $u \in L^q(B)$ für alle offenen Bälle $B \subseteq \Omega$ mit $\overline B \subseteq \Omega$ gilt.


\begin{bem}
  Eine Funktion $u$ ist genau dann lokal integrierbar über $\Omega$, falls $u \in L^q(K)$ für alle Kompakta $K \subseteq \Omega$ gilt.

  Diese Aussage findet man ebenfalls in der Literatur zur Definition lokaler Integrierbarkeit.
  Tatsächlich ist sie äquivalent zur obigen Definiton.
  Denn einerseits ist für alle Bälle $B$ auch $\overline B$ ein Kompaktum und die Aussage folgt aus der Inklusionsbeziehung $L^q(\overline B) \subseteq L^q(B)$.
  Andererseits lässt sich jedes Kompaktum $K$ mit endlich vielen Bällen $B_1, \dots, B_n$ mit $\overline B_i \subseteq \Omega, i =1,\dots,n,$ überdecken und die Umkehrung der Aussage folgt aus der Inklusionsbeziehung $\bigcap_{i = 1}^n L^q(B_i) \subseteq L^q(K)$.
\end{bem}

\subsection{Differentiation von Distributionen -- Schwache Differenzierbarkeit und Sobolevräume}

\begin{itemize}
  \item \cite{sohr2001navier}[S.34ff.]
  \item \cite{werner2011fa}[S.433f.]
  \item Differentiation von Distributionen
  \item Divergenzfreie Test-Funktionen
  \item Sobolevräume und ihre Normen \cite{sohr2001navier}[S.38ff.]
\end{itemize}

