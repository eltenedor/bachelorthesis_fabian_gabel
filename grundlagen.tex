\chapter{Grundlagen}
\label{cp:grundlagen}

\section{Physikalische Grundlagen}

\begin{itemize}
  \item Physikalische Motivation dieses Gleichungssystems
  \item Linearisierung der Navier-Stokes-Gleichungen (Motivation der Stokes-Gleichung)
  \item Schleichende Strömungen z.B. \cite{spurk10stroemungslehre}[S.112,S.489]. 
\end{itemize}

\section{Funktionalanalytische Grundlagen -- Distributionen und Sobolevräume}

%\subsection{Theorie des Hilbertraums}
%Hauptsächlich zur Einführung der Verwendeten Notation
\subsection{Glatte Funktionen}

\begin{itemize}
  \item \cite{sohr2001navier}[S.22ff.]
  \item glatte / Testfunktionen definieren
  \item Normfamilien und Teilräume angeben
\end{itemize}

\subsection{Topologisierung des Raums der Testfunktionen und ein Stetigkeitsbegriff}

\begin{itemize}
  \item Inhalte ganz zu Beginn von \cite{sohr2001navier}[S.34] wiedergeben, zusätzliche (topologische Eigenschaften) beweisen, aus Werner s.u.
  \item \cite{werner2011fa}[S.430]
  \item Lemma VIII.5.1 (a)(d), VIII.2.3
  \item Satz VIII.5.4(iii)
  \item lokale Integrierbarkeit
  \item Einbettung von $L^1_{\mathrm{loc}}$ in $C_0^\infty(\Omega)'$
\end{itemize}

\subsection{Differentiation von Distributionen -- Schwache Differenzierbarkeit und Sobolevräume}

\begin{itemize}
  \item \cite{sohr2001navier}[S.34ff.]
  \item \cite{werner2011fa}[S.433f.]
  \item Differentiation von Distributionen
  \item Divergenzfreie Test-Funktionen
  \item Sobolevräume und ihre Normen \cite{sohr2001navier}[S.38ff.]
\end{itemize}

\chapter{Lösungen von $\nabla p = f$}
\section{Lipschitzgebiete und Gebietszerlegungen}

\begin{itemize}
  \item \cite{sohr2001navier}[S.55, Lemma 1.4.1]
\end{itemize}

\section{Kompakte Einbettungen}

\begin{itemize}
  \item \cite{sohr2001navier}[S.58, Lemma 1.5.4]
\end{itemize}

\section{Darstellung von Funktionalen}

\begin{itemize}
  \item \cite{sohr2001navier}[S.61, Lemma 1.6.1]
\end{itemize}


\section{Die Glättungsmethode}

\begin{itemize}
  \item \cite{sohr2001navier}[S.64ff.]
\end{itemize}

\section{Das Gradientenkriterium}

\begin{itemize}
  \item \cite{sohr2001navier}[Lemma 2.2.1, S.73]
\end{itemize}

\chapter{Helmholtz-Zerlegung in $L^2$}

\begin{itemize}
  \item Lemma 2.5.1, 2.5.2 \cite{sohr2001navier}[S.81ff.]
\end{itemize}

\chapter{Zusammenfassung und Ausblick}
