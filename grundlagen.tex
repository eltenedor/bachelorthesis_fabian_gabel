\chapter{Grundlagen}
\label{cp:grundlagen}

\section{Physikalische Grundlagen}

\begin{itemize}
  \item Physikalische Motivation dieses Gleichungssystems
  \item Linearisierung der Navier-Stokes-Gleichungen (Motivation der Stokes-Gleichung)
  \item Schleichende Strömungen z.B. \cite{spurk10stroemungslehre}[S.112,S.489]. 
\end{itemize}

\section{Funktionalanalytische Grundlagen -- Distributionen und Sobolevräume}

Dieses Unterkapitel beschäftigt sich mit den für die kommenden Kapitel zentralen Funktionenräumen und dient zudem der Einführung der verwendeten Notation und Bildung der verwendeten Begriffe.
Die Notation ist an \cite{sohr2001navier} angelehnt.

\subsection{Glatte Funktionen und der Raum der Testfunktionen}
\label{subsec:smooth}

Ziel dieses Abschnittes ist es die nötigen Begriffe und Definitionen im Zusammenhang mit glatten Funktionen bereitzustellen.
Im Folgenden bezeichne $\Omega \subseteq \R^n$ stets ein nichtleeres Gebiet.

Sei $k \in \N$ und es bezeichne $C^k(\Omega)$ die Menge aller Funktionen
$$
u \colon \Omega \to \R, \quad x \mapsto u(x),
$$
sodass alle partiellen Ableitungen $D^\alpha u$ existieren und stetig sind für alle Multiindices $\alpha \in N_0^n$ mit $0 \leq |\alpha| \leq k$. Mit
$$
  C^\infty(\Omega) := \bigcap_{k = 0}^\infty C^k(\Omega)
$$
bezeichnen wir den Raum der glatten Funktionen auf $\Omega$.

Im Kontext der Approximation von $L^p$-Funktionen spielt ein Funktionenraum eine wichtige Rolle: der Raum der glatten Funktionen mit kompaktem Träger
$$
C_0^\infty(\Omega) := \{ u \in C^\infty(\Omega) \mid \supp u \text{ kompakt und } \supp u \subseteq \Omega\}.
$$
Wir werden diesem Raum in Abschnitt \ref{subset:distributionsSobolev} nochmals als Raum der Testfunktionen begegnen.
Wir werden zudem den Raum $C^\infty(\overline\Omega)$ aller Restriktionen $u|_{\overline\Omega}$ von Funktionen aus $C^\infty(\R^n)$ mit
\begin{displaymath}
  \sup_{|\alpha| < \infty, x \in \R^n} |D^\alpha u(x)| < \infty \tag{$\ast$}
\end{displaymath}
benötigen.
Aufgrund der Eigenschaft ($\ast$) lässt sich der Raum $C^\infty(\overline\Omega)$ mit einer Norm ausstatten:
$$
\norm{u}_{C^\infty} = \norm{u}_{C^\infty(\overline\Omega} := \sup_{|\alpha|\leq k, x \in \overline\Omega} |D^\alpha u(x)|.
$$

Alle eingeführten Räume lassen sich auf natürliche Weise zu Räumen von Vektorfeldern verallgemeinern.
Man erhält so
\begin{align*}
  C^\infty(\Omega)^m &:= \{(u_1,\dots,u_m) \mid u_j \in C^\infty(\Omega), j = 1,\dots,m\}, \\
  C_0^\infty(\Omega)^m &:= \{(u_1,\dots,u_m) \mid u_j \in C_0^\infty(\Omega), j = 1,\dots,m\} \quad\text{und} \\
  C^\infty(\overline\Omega)^m &:= \{(u_1,\dots,u_m) \mid u_j \in C^\infty(\overline\Omega), j = 1,\dots,m\},
\end{align*}
wobei der letzte Vektorraum durch die Norm
$$
\norm{u}_{C^\infty} = \norm{u}_{C^\infty(\overline\Omega)}^m := \sup_{j = 1,\dots, m} \norm{u_j}_{C^k(\overline\Omega)}
$$
zu einem normierten Vektorraum wird.

\begin{itemize}
  \item Andere Definition aus Adams vergleichen und Äquivalenz (?) beweisen.
\end{itemize}

Von besonderer Rolle für die Lösungstheorie der stationären inkompressiblen \navier\hyp{}\stokes\hyp{}Gleichungen ist der Untervektorraum
$$
C_{0,\sigma}^\infty(\Omega) := \{u \in C_0^\infty(\Omega)^n \mid \div u = 0\}
$$
der divergenzfreien Vektorfelder.

\subsection{Falten glättet und weitere Eigenschaften}
\label{subsec:mollification}

Aus der Integrationstheorie ist bekannt, dass sich $L^q$-Funktionen durch Faltungen mit Glättungskernen $\mathcal{F}_\varepsilon$ approximieren lassen.

\begin{lem}
  \label{lem:mollification}
  Sei $\Omega \subseteq \R^n$ mit $n \geq 1$ ein Gebiet und $1 \leq q < \infty$ sowie $\varepsilon > 0$.
  Für alle $u \in L^q(\Omega)$ gilt dann
  $$
  \norm{(\mathcal{F}_\varepsilon \ast u)}_{L^q(\Omega)} \leq  \norm{u}_{L^q(\Omega)}
  $$
  und
  \begin{displaymath}
    \lim_{\varepsilon \to 0} \norm{(\mathcal{F}_\varepsilon \ast u) - u}_{L^q(\Omega)} = 0. 
  \end{displaymath}
\end{lem}

Sei nun $\Omega_0 \subseteq \overline\Omega_0 \subseteq \Omega$ ein beschränktes Teilgebiet und
\begin{equation}
  \label{eq:convolutionEpsilon}
  0 < \varepsilon < \dist(\Omega_0, \partial\Omega).
\end{equation}
Zusätzlich sei $u \in L^1_{\loc}(\Omega)$.
Setzt man $u(x) := 0$ für alle $x \in \R^n \setminus \Omega$, so folgt $u \in L^1_{\loc}(\R^n)$.
Wie \cite{rudin1991fa}[S.171, Theorem 6.30(b)] zeigt, lassen sich aus der Integrationstheorie bekannte Eigenschaften der Glättungen auch auf Faltungen 
$$
u^\varepsilon := \mathcal{F}_\varepsilon \ast u
$$
von Glättungskernen $\mathcal{F}_\varepsilon \ast u$ mit Distributionen $u \in C_0^\infty(\Omega)'$ erweitern.
Dazu zählt einerseits, dass $u^\varepsilon \in C^\infty(\R^n)$ gilt.
Da $\Omega_0$ als beschränkt vorausgestzt wurde, gilt sogar $u^\varepsilon \in C^\infty(\overline\Omega_0)$.
Andererseits ist für alle Multiindices $\alpha \in \N_0^n$ und alle $x \in \Omega_0$ die folgende Gleichung gültig:
\begin{equation}
  \label{eq:convolutionDiff}
  (D^\alpha u^\varepsilon)(x) 
  = (\mathcal{F}_\varepsilon \ast (D^\alpha u))(x)
  = ( (D^\alpha \mathcal{F}_\varepsilon) \ast u)(x), \quad x \in \Omega_0.
\end{equation}

\begin{lem}
  Sei $\Omega \subseteq \R^n$ mit $n \geq 1$ ein Gebiet und $1 \leq q < \infty$.
  Ist $u \in L^1_{\loc}(\Omega)$ und gilt $\nabla u = 0$ im distributionellen Sinne, dann ist $u$ konstant.
\end{lem}

\begin{proof}
  Für alle $x \in \Omega_0$ und $\varepsilon$ wie in Ungleichung \ref{eq:convolutionEpsilon} folgern wir unter Verwendung der Gleichung \ref{eq:convolutionDiff}
  \begin{align*}
    \nabla u^\varepsilon(x)
    &= (D^1 (\mathcal{F}_\varepsilon \ast u), \dots, D^n (\mathcal{F}_\varepsilon \ast u))^T \\
    &= ( \mathcal{F}_\varepsilon \ast D^1 u, \dots, \mathcal{F}_\varepsilon \ast D^n u)^T \\
    &= (0, \dots, 0)^T.
  \end{align*}
  Es ist $u^\varepsilon$ eine glatte Funktion.
  Zudem ist $\Omega_0$ als offene zusammenhängende Teilmenge von $\R^n$ insbesondere Wegzusammenhängend.
  Ausgehend von der Integraldarstellung des Funktionszuwachses \cite{koenigsberger2004ana2}[S.57] gilt $u_\varepsilon = C_\varepsilon$ auf ganz $\Omega_0$.
  Mit Lemma \ref{lem:mollification} folgt nun die Netzkonvergenz $C_\varepsilon \to C$ für $\varepsilon \to 0$ auf $\Omega_0$.
  Da sich zudem aufgrund von Lemma \ref{lem:lipschitzExhaust} das Gebiet $\Omega$ von einer Folge $(\Omega_j)_{j \in \N}$ offener beschränkter \lipschitz\hyp{}Gebiete ausschöpfen lässt, erhalten wir durch Anwendung des beschriebenen Arguments auf die Folgenglieder $\Omega_j$ sogar $u = C$ auf ganz $\Omega$.
\end{proof}

\begin{itemize}
  \item \cite{sohr2001navier}[Die Glättungsmethode S.64ff.]
  \item \cite{adams2003sobolev}[S.10, S.9(alt), S.36, S.30(alt)]
\end{itemize}

\subsection{Differentiation von Distributionen -- Schwache Differenzierbarkeit und Sobolevräume}
\label{subsec:distributionsSobolev}

In Abschnitt \ref{subsec:smooth} hatten wir bereits

\begin{itemize}
  \item \cite{sohr2001navier}[S.34ff.]
  \item \cite{werner2011fa}[S.433f.]
  \item Differentiation von Distributionen
  \item Divergenzfreie Test-Funktionen
  \item Sobolevräume und ihre Normen \cite{sohr2001navier}[S.38ff.]
\end{itemize}

