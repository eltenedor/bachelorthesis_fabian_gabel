\chapter{Grundlagen}
\label{cp:grundlagen}

\section{Physikalische Grundlagen}

\begin{itemize}
  \item Physikalische Motivation dieses Gleichungssystems
  \item Linearisierung der Navier-Stokes-Gleichungen (Motivation der Stokes-Gleichung)
  \item Schleichende Strömungen z.B. \cite{spurk10stroemungslehre}[S.112,S.489]. 
\end{itemize}

\section{Funktionalanalytische Grundlagen -- Distributionen und Sobolevräume}

%\subsection{Theorie des Hilbertraums}
%Hauptsächlich zur Einführung der Verwendeten Notation
\subsection{Glatte Funktionen}

\begin{itemize}
  \item \cite{sohr2001navier}[S.22ff.]
  \item glatte / Testfunktionen definieren
  \item Normfamilien und Teilräume angeben
\end{itemize}

\subsection{Topologisierung des Raums der Testfunktionen und ein Stetigkeitsbegriff}

\begin{itemize}
  \item Inhalte ganz zu Beginn von \cite{sohr2001navier}[S.34] wiedergeben, zusätzliche (topologische Eigenschaften) beweisen, aus Werner s.u.
  \item \cite{werner2011fa}[S.430]
  \item Lemma VIII.5.1 (a)(d), VIII.2.3
  \item Satz VIII.5.4(iii)
  \item lokale Integrierbarkeit
  \item Einbettung von $L^1_{\mathrm{loc}}$ in $C_0^\infty(\Omega)'$
\end{itemize}

\subsection{Differentiation von Distributionen -- Schwache Differenzierbarkeit und Sobolevräume}

\begin{itemize}
  \item \cite{sohr2001navier}[S.34ff.]
  \item \cite{werner2011fa}[S.433f.]
  \item Differentiation von Distributionen
  \item Divergenzfreie Test-Funktionen
  \item Sobolevräume und ihre Normen \cite{sohr2001navier}[S.38ff.]
\end{itemize}

\chapter{Lösungen von $\nabla p = f$}
\section{Lipschitzgebiete und Gebietszerlegungen}

\begin{defn}
  \lipschitz\hyp{}Gebiete  
\end{defn}

\begin{lem}
  \label{lem:lipschitzExhaust}
  Sei $\Omega \subseteq \R^n$ mit $n \geq 2$ ein Gebiet.
  Dann existiert eine Folge $(\Omega_j)_{j \in \N}$ beschränkter \lipschitz\hyp{}Gebiete $\Omega_j \subseteq \Omega$ und eine Folge $(\varepsilon_j)_{j \in \N}$ positiver reeller Zahlen mit folgenden Eigenschaften:
  \begin{enumerate}[a)]
    \item Für alle $j \in \N$ gilt $\overline{\Omega_j} \subseteq \Omega_{j + 1}$.
    \item Für alle $j \in \N$ gilt $\dist( \partial \Omega_{j + 1}, \Omega_j) \geq \varepsilon_{j + 1}$. 
    \item Es gilt $\lim_{j \to \infty} \varepsilon_j = 0$.
    \item Die Gebiete $\Omega_j$ schöpfen $\Omega$ aus.
  \end{enumerate}
\end{lem}

\begin{proof}
Im Folgenden bezeichne $B_r(x) \subseteq \R^n$ den bezüglich \euklid ischer Topologie offenen Ball mit Radius $r$ und Mittelpunkt $x$.

  Für ein festgewähltes $x_0 \in \Omega$ betrachten wir den Schnitt 
  $$
  \Omega' := \Omega \cap B_1(x_0).$$ 
  Als Schnitt offener Mengen ist $\Omega'$ wiederum offen. 
  Bezüglich der Teilraumtopologie muss $\Omega'$ jedoch nicht zwingend zusammenhängend sein.
  Wir bezeichnen nun mit $\widetilde{\Omega}$ die Zusammenhangskomponente von $\Omega'$, welche $x_0$ enthält.
  Da die Zusammenhangskomponenten eines topologischen Raumes immer eine Partition desselben bilden, ist $\Omega'$ eindeutig bestimmt.
  Insbesondere gilt für den Rand
  $$ 
  \partial \widetilde\Omega \subseteq \overline{B_1(x_0)}, 
  $$
  er ist also als abgeschlossene Teilmenge des Kompaktums $\overline{B_1(x_0)}$ selbst kompakt.
  Für alle $\varepsilon > 0$ lässt sich daher $\partial \widetilde\Omega$ durch endlich viele Bälle $B\varepsilon(x_j)$, mit $x_j \in \partial \widetilde\Omega$ für alle $j = 1,\dots,m$, überdecken:
  $$ 
  \partial \widetilde\Omega \subseteq \bigcup_{j = 1}^m B_\varepsilon(x_j).
  $$

  Wir definieren nun $\widehat\Omega := \widetilde\Omega \setminus \bigcap_{j = 1}^m \overline{B_\varepsilon(x_j)}$.
  Zusätzlich nehmen wir an, dass $0 < \varepsilon < 1$ gilt und so klein gewählt wurde, dass 
  Dies lässt sich immer erreichen, da $\widetilde\Omega$ als bezüglich Teilraumtopologie offen-abgeschlossene Menge auch in $\R^n$ offen ist und daher ein $\delta > 0$ existiert
  
\end{proof}

\begin{itemize}
  \item \cite{sohr2001navier}[S.55, Lemma 1.4.1]
\end{itemize}

\newpage
\section{Kompakte Einbettungen}

\begin{itemize}
  \item \cite{sohr2001navier}[S.58, Lemma 1.5.4]
\end{itemize}

\newpage
\section{Darstellung von Funktionalen}

\begin{itemize}
  \item \cite{sohr2001navier}[S.61, Lemma 1.6.1]
\end{itemize}


\section{Die Glättungsmethode}

\begin{itemize}
  \item \cite{sohr2001navier}[S.64ff.]
\end{itemize}

\section{Das Gradientenkriterium}

\begin{itemize}
  \item \cite{sohr2001navier}[Lemma 2.2.1, S.73]
\end{itemize}

\chapter{Helmholtz-Zerlegung in $L^2$}

\begin{itemize}
  \item Lemma 2.5.1, 2.5.2 \cite{sohr2001navier}[S.81ff.]
\end{itemize}

\chapter{Zusammenfassung und Ausblick}
