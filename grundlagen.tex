\chapter{Grundlagen}
\label{cp:grundlagen}

\section{Physikalische Grundlagen}

\begin{itemize}
  \item Physikalische Motivation dieses Gleichungssystems
  \item Linearisierung der Navier-Stokes-Gleichungen (Motivation der Stokes-Gleichung)
  \item Schleichende Strömungen z.B. \cite{spurk10stroemungslehre}[S.112,S.489]. 
\end{itemize}

\section{Funktionalanalytische Grundlagen}

Dieses Unterkapitel beschäftigt sich mit den für die kommenden Kapitel zentralen Funktionenräumen und dient zudem der Einführung der verwendeten Notation und Bildung der verwendeten Begriffe.
Die Notation ist an \cite{sohr2001navier} angelehnt.

\subsection{Glatte Funktionen und der Raum der Testfunktionen}
\label{subsec:smooth}

Ziel dieses Abschnittes ist es die nötigen Begriffe und Definitionen im Zusammenhang mit glatten Funktionen bereitzustellen.
Im Folgenden bezeichne $\Omega \subseteq \R^n$ stets ein nichtleeres Gebiet.

Sei $k \in \N$ und es bezeichne $\C^k(\Omega)$ die Menge aller Funktionen
$$
u \colon \Omega \to \R, \quad x \mapsto u(x),
$$
sodass alle partiellen Ableitungen $D^\alpha u$ existieren und stetig sind für alle Multiindices $\alpha \in N_0^n$ mit $0 \leq |\alpha| \leq k$. Mit
$$
  \C^\infty(\Omega) := \bigcap_{k = 0}^\infty \C^k(\Omega)
$$
bezeichnen wir den Raum der glatten Funktionen auf $\Omega$.

Im Kontext der Approximation von $L^p$-Funktionen spielt ein Funktionenraum eine wichtige Rolle: der Raum der glatten Funktionen mit kompaktem Träger
$$
\C_0^\infty(\Omega) := \{ u \in \C^\infty(\Omega) \mid \supp u \text{ kompakt und } \supp u \subseteq \Omega\}.
$$
Wir werden diesem Raum in Abschnitt \ref{subset:distributionsSobolev} nochmals als Raum der Testfunktionen begegnen.
Wir werden zudem den Raum $\C^\infty(\overline\Omega)$ aller Restriktionen $u|_{\overline\Omega}$ von Funktionen aus $\C^\infty(\R^n)$ mit
\begin{displaymath}
  \sup_{|\alpha| < \infty, x \in \R^n} |D^\alpha u(x)| < \infty \tag{$\ast$}
\end{displaymath}
benötigen.
Aufgrund der Eigenschaft ($\ast$) lässt sich der Raum $\C^\infty(\overline\Omega)$ mit einer Norm ausstatten:
$$
\norm{u}_{\C^\infty} = \norm{u}_{\C^\infty(\overline\Omega} := \sup_{|\alpha|\leq k, x \in \overline\Omega} |D^\alpha u(x)|.
$$

Alle eingeführten Räume lassen sich auf natürliche Weise zu Räumen von Vektorfeldern verallgemeinern.
Man erhält so
\begin{align*}
  \C^\infty(\Omega)^m &:= \{(u_1,\dots,u_m) \mid u_j \in \C^\infty(\Omega), j = 1,\dots,m\}, \\
  \C_0^\infty(\Omega)^m &:= \{(u_1,\dots,u_m) \mid u_j \in \C_0^\infty(\Omega), j = 1,\dots,m\} \quad\text{und} \\
  \C^\infty(\overline\Omega)^m &:= \{(u_1,\dots,u_m) \mid u_j \in \C^\infty(\overline\Omega), j = 1,\dots,m\},
\end{align*}
wobei der letzte Vektorraum durch die Norm
$$
\norm{u}_{\C^\infty} = \norm{u}_{\C^\infty(\overline\Omega)}^m := \sup_{j = 1,\dots, m} \norm{u_j}_{\C^k(\overline\Omega)}
$$
zu einem normierten Vektorraum wird.

\begin{itemize}
  \item Andere Definition aus Adams vergleichen und Äquivalenz (?) beweisen.
\end{itemize}

Von besonderer Rolle für die Lösungstheorie der stationären inkompressiblen \navier\hyp{}\stokes\hyp{}Gleichungen ist der Untervektorraum
$$
\C_{0,\sigma}^\infty(\Omega) := \{u \in \C_0^\infty(\Omega)^n \mid \div u = 0\}
$$
der divergenzfreien Vektorfelder.

\subsection{``Falten glättet'' und weitere Eigenschaften}
\label{subsec:mollification}

Aus der Integrationstheorie ist bekannt, dass sich $L^q$-Funktionen durch Faltungen mit Glättungskernen $\mathcal{F}_\varepsilon$ approximieren lassen.

\begin{lem}
  \label{lem:mollification}
  Sei $\Omega \subseteq \R^n$ mit $n \geq 1$ ein Gebiet und $1 \leq q < \infty$ sowie $\varepsilon > 0$.
  Für alle $u \in L^q(\Omega)$ gilt dann
  $$
  \norm{(\mathcal{F}_\varepsilon \ast u)}_{L^q(\Omega)} \leq  \norm{u}_{L^q(\Omega)}
  $$
  und
  \begin{displaymath}
    \lim_{\varepsilon \to 0} \norm{(\mathcal{F}_\varepsilon \ast u) - u}_{L^q(\Omega)} = 0. 
  \end{displaymath}
  \qed
\end{lem}

Sei nun $\Omega_0 \subseteq \overline\Omega_0 \subseteq \Omega$ ein beschränktes Teilgebiet und
\begin{equation}
  \label{eq:convolutionEpsilon}
  0 < \varepsilon < \dist(\Omega_0, \partial\Omega).
\end{equation}
Zusätzlich sei $u \in L^1_{\loc}(\Omega)$.
Setzt man $u(x) := 0$ für alle $x \in \R^n \setminus \Omega$, so folgt $u \in L^1_{\loc}(\R^n)$.
Wie \cite{rudin1991fa}[S.171, Theorem 6.30(b)] zeigt, lassen sich aus der Integrationstheorie bekannte Eigenschaften der Glättungen auch auf Faltungen 
$$
u^\varepsilon := \mathcal{F}_\varepsilon \ast u
$$
von Glättungskernen $\mathcal{F}_\varepsilon \ast u$ mit Distributionen $u \in \C_0^\infty(\Omega)'$ erweitern.
Dazu zählt einerseits, dass $u^\varepsilon \in \C^\infty(\R^n)$ gilt.
Da $\Omega_0$ als beschränkt vorausgestzt wurde, gilt sogar $u^\varepsilon \in \C^\infty(\overline\Omega_0)$.
Andererseits ist für alle Multiindices $\alpha \in \N_0^n$ und alle $x \in \Omega_0$ die folgende Gleichung gültig:
\begin{equation}
  \label{eq:convolutionDiff}
  (D^\alpha u^\varepsilon)(x) 
  = (\mathcal{F}_\varepsilon \ast (D^\alpha u))(x)
  = ( (D^\alpha \mathcal{F}_\varepsilon) \ast u)(x), \quad x \in \Omega_0.
\end{equation}

\begin{lem}
  Sei $\Omega \subseteq \R^n$ mit $n \geq 1$ ein Gebiet und $1 \leq q < \infty$.
  Ist $u \in L^1_{\loc}(\Omega)$ und gilt $\nabla u = 0$ im distributionellen Sinne, dann ist $u$ konstant.
\end{lem}

\begin{proof}
  Für alle $x \in \Omega_0$ und $\varepsilon$ wie in Ungleichung (\ref{eq:convolutionEpsilon}) folgern wir unter Verwendung der Gleichung (\ref{eq:convolutionDiff})
  \begin{align*}
    \nabla u^\varepsilon(x)
    &= (D^1 (\mathcal{F}_\varepsilon \ast u), \dots, D^n (\mathcal{F}_\varepsilon \ast u))^T \\
    &= ( \mathcal{F}_\varepsilon \ast D^1 u, \dots, \mathcal{F}_\varepsilon \ast D^n u)^T \\
    &= (0, \dots, 0)^T.
  \end{align*}
  Es ist $u^\varepsilon$ eine glatte Funktion.
  Zudem ist $\Omega_0$ als offene zusammenhängende Teilmenge von $\R^n$ insbesondere Wegzusammenhängend.
  Ausgehend von der Integraldarstellung des Funktionszuwachses \cite{koenigsberger2004ana2}[S.57] gilt $u_\varepsilon = C_\varepsilon$ auf ganz $\Omega_0$.
  Mit Lemma \ref{lem:mollification} folgt nun die Netzkonvergenz $C_\varepsilon \to C$ für $\varepsilon \to 0$ auf $\Omega_0$.
  Da sich zudem aufgrund von Lemma \ref{lem:lipschitzExhaust} das Gebiet $\Omega$ von einer Folge $(\Omega_j)_{j \in \N}$ offener beschränkter \lipschitz\hyp{}Gebiete ausschöpfen lässt, erhalten wir durch Anwendung des beschriebenen Arguments auf die Folgenglieder $\Omega_j$ sogar $u = C$ auf ganz $\Omega$.
\end{proof}

\begin{itemize}
  \item \cite{sohr2001navier}[Die Glättungsmethode S.64ff.]
  \item \cite{adams2003sobolev}[S.10, S.9(alt), S.36, S.30(alt)]
\end{itemize}

\subsection{Schwache Differenzierberkeit \\-- Distributionen und Sobolev-Räume}
\label{subsec:distributionsSobolev}

Es bezeichne im Folgenden wieder $\Omega \subseteq \R^n$ ein offenes Gebiet mit $n \geq 1$.

\subsubsection{Lokal integrierbare Funktionen}

Wir werden im Folgenden oft eine Obermenge der bezüglich des \lebesgue\hyp{}Maßes auf $\R^n$ integrierbaren Funktionen verwenden.
Für $1 \leq q \leq \infty$ schreiben wir
$$
  u \in L^q_{\loc}(\Omega),
$$
und nennen $u$ \emph{lokal integrierbar}, falls $u \in L^q(B)$ für alle offenen Bälle $B \subseteq \Omega$ mit $\overline B \subseteq \Omega$ gilt.

\begin{bem}
  Eine Funktion $u$ ist genau dann lokal integrierbar über $\Omega$, falls $u \in L^q(K)$ für alle Kompakta $K \subseteq \Omega$ gilt.

  Diese Aussage findet man ebenfalls in der Literatur zur Definition lokaler Integrierbarkeit.
  Tatsächlich ist sie äquivalent zur obigen Definiton.
  Denn einerseits ist für alle Bälle $B$ auch $\overline B$ ein Kompaktum und die Aussage folgt aus der Inklusionsbeziehung $L^q(\overline B) \subseteq L^q(B)$.
  Andererseits lässt sich jedes Kompaktum $K$ mit endlich vielen Bällen $B_1, \dots, B_n$ mit $\overline B_i \subseteq \Omega, i =1,\dots,n,$ überdecken und die Umkehrung der Aussage folgt aus der Inklusionsbeziehung $\bigcap_{i = 1}^n L^q(B_i) \subseteq L^q(K)$.
\end{bem}

Des Weiteren schreiben wir
$$
u \in L^q_{\loc}(\overline\Omega),
$$
falls $u \in L^q(B \cap \Omega)$ für jeden offenen Ball $B \subseteq \R^n$ mit $B \cap \Omega \neq \emptyset$ gilt.
Zusammenfassend gilt also die folgende Inklusionsbeziehung:
$$
L^q(\Omega) \subseteq L^q_{\loc}(\overline\Omega) \subseteq L^q_{\loc}(\Omega).
$$

\subsubsection{Distributionen}

In Abschnitt \ref{subsec:smooth} hatten wir bereits den Raum der Testfunktionen $\C_0^\infty(\Omega)$ kennegelernt.
Wir werden und vor allem für seinen Dualraum, den Raum der stetigen Funktionale auf $\C_0^\infty(\Omega)$ existieren.
Um überhaupt über Stetigkeit von Funktionalen auf $\C_0^\infty(\Omega)$ sprechen zu können, müssen wir jedoch zunächst eine Topologie festlegen.
Wir werden dies durch die Spezifikation einer Konvergenzstruktur tun.
Ein lineares Funktional $F \colon \C_0^\infty(\Omega) \to \R$ ist stetig, genau dann wenn für jedes beschränkte Teilgebiet $G \subseteq \Omega$ mit $\overline G \subseteq \Omega$ ein $k \in \N_0$ und $C = C(F,G)$ existiert, sodass
$$
|F(\varphi)| \leq C \norm{\varphi}_{\C^k}(\overline G)
$$
gilt, wobei $\norm{\cdot}_{\C^k}(\overline G)$ gerade die
Diese Konvergenzstruktur ist topologisierbar und macht $\C_0^\infty(\Omega)$ zu einem lokalkonvexen Vektorraum.
Genauere Ausführungen finden sich in \cite{werner2011fa}[S.433f.].
Den Raum $\C_0^\infty(\Omega)'$ aller im obigen Sinne stetigen Funktionale
$$
F \colon \C_0^\infty(\Omega) \to \R, \quad \varphi \mapsto F(\varphi) = [F, \varphi],
$$
bezeichnen wir als den Raum der \emph{Distributionen}.
Hierbei stellt $[\cdot,\cdot]$ die duale Paarung auf $\Omega$ dar. 

Wir betrachten nun spezielle Distributionen.
Ist $f \in L^1_{\loc}$ so induziert bezeichnet man 
$$
f \mapsto \langle f, \varphi \rangle := \int_\Omega f \varphi \d x.
$$
als zugehörige \emph{reguläre} Distribution.
Dass diese Zuodrnung sogar injektiv ist und somit zur Inklusionsbeziehung $L^1_{\loc}(\Omega) \subseteq \C^\infty_0(\Omega)'$ führt, lässt sich in \cite{werner2011fa}[S.432, Beispiel (a)] nachlesen.

\subsubsection{Differentiation von Distributionen}

Distributionen ermöglichen es, den für die Analysis grundlegenden Begriff der Differenzierbarkeit zu verallgemeinern.
Ist $\alpha \in \N_0^n$ ein Multiindex und $D^\alpha \colon \C_0^\infty(\Omega) \to \C_0^\infty(\Omega)$ der zugehörige Ableitungsoperator, so definieren wir für eine Distribution $F \in \C_0^\infty(\Omega)'$ ihre distributionelle Ableitung $D^\alpha F \in \C_0^\infty(\Omega)'$ durch
$$
  [D^\alpha F, \varphi] := (-1)^{|\alpha|} [F, D^\alpha \varphi].
$$
Bis auf ein Vorzeichen stimmt also die distributionelle Ableitung mit der zum Ableitungsoperator dualen Abbildung $(D^\alpha)' \colon \C_0^\infty(\Omega)' \to \C_0^\infty(\Omega)'$ überein.
Dass die distributionelle Ableitung wohldefiniert und sogar schwach*-stetig ist, lässt sich in \cite{werner2011fa}[S.434, Lemma VIII.5.7] nachlesen.

Im Unterabschnitt zu glatten Funktionen hatten wir die dortigen Definitionen auf auf Produkträume ausgeweitet.
Wir versehen den so entstandenen Produktraum $\C_0^\infty(\Omega)^m$ mit der Produkttopologie und wenden uns nun dem zugehörigen Distributionenraum $\C_0^\infty(\Omega)'^m$ zu.

Wir betrachten dazu die Funktion
$$
  F = (F_1, \dots, F_m), \quad F_j \in \C_0^\infty(\Omega)', \quad j = 1, \dots, m
$$
und definieren für alle 
$$\varphi = (\varphi_1, \dots, \varphi_m) \in \C_0^\infty(\Omega)^m$$
die Distribution
$$
[F, \varphi] := [F_1, \varphi_1] + \dots + [F_m, \varphi_m].
$$
Durch Betrachtung von Funktionen des Typs $\psi^{(i)} = (0, \dots,0, \psi_i, 0,\dots,0) \in \C_0^\infty(\Omega)^m$ mit $i \in \{1,\dots,m\}$ erkennen wir, dass jedes stetige Funktional auf $\C_0^\infty(\Omega)^m$ diese Form besitzt.
Der Raum der Distributionen zu den Testfunktionen $\C_0^\infty(\Omega)^m$ besitzt also die Form
\begin{align*}
  \C_0^\infty(\Omega)'^{\,m}
  &= (\C_0^\infty(\Omega)^m)' \\
  &= \{(F_1,\dots,F_m) \mid F_j \in \C_0^\infty(\Omega)', \quad j = 1,\dots,m\}
\end{align*}

Wir erinnern uns an dem im Unterabschnitt zu glatten Funktionen definierten Lösungsraum $\C_{0, \sigma}^\infty(\Omega)$ der divergenzfreien Funktionen.
Nach dem Fortetzungssatz von \hahn\hyp{}\banach\ für lokalkonvexe Vektorräume (\cite{werner2011fa}[S.408, Satz VIII.2.8]) erhalten wir die stetigen Funktionale auf $\C_{0,\sigma}^\infty$ gerade als Einschränkungen der stetigen Funktionale auf $\C_0^\infty(\Omega)^n$. Es gilt somit
$$
\C_{0,\sigma}^\infty(\Omega)' = \{F|_{\C_{0,\sigma}^\infty(\Omega)} \mid F \in (\C_0^\infty(\Omega)^n)'\}.
$$

Betrachten wir nun den \hilbert\hyp{}Raum $L^2(\Omega)^n$.
Aus Lemma \ref{lem:mollification} ist bekannt, dass die glatten Funktionen mit kompaktem Träger dicht in $L^2(\Omega)$ liegen. 
Ebenso stellt der Produktraum $\C_0^\infty(\Omega)^n$ einen dichten Unterraum von $L^2(\Omega)^n$ dar.
Um diese Tatsache zu immitieren definieren wir den Unterraum 
$$
  L^2_\sigma(\Omega) 
  := \overline{C_{0,\sigma}^\infty(\Omega)}^{\norm{\cdot}_2}
  \subseteq L^2(\Omega)^n
$$
Identifizieren wir nun jedes Funktion $u \in L^2(\Omega)^n$ mit dem Funktional
$$
\langle u, \cdot \rangle \colon \varphi \mapsto \langle u, \varphi \rangle, \quad \varphi \in C_0^\infty(\Omega)^n
$$
so erhalten wir die Einbettung 
$$
  L^2(\Omega)^n \subseteq (C_0^\infty(\Omega)^n)'.
$$
Ebenso lassen sich Funktionen $u \in L^2_\sigma(\omega)$ mit der entsprechenden Einschränkung
$$
\langle u, \cdot \rangle \colon \varphi \mapsto \langle u, \varphi \rangle, \quad \varphi \in C_{0,\sigma}^\infty(\Omega)^n
$$
identifizieren, was die Einbettung
$$
  L^2(\Omega)^n \subseteq (C_0^\infty(\Omega)^n)'.
$$
liefert.

\subsubsection{Sobolev-Räume}

Im Allgemeinen wird die Ableitung einer regulären Distribution nicht wieder regulär sein.
Wir definieren daher für alle $k \in \N$ und $1 \leq q \leq \infty$ den $L^q$\hyp{}\sobolev\hyp{}Raum $W^{k,q}(\Omega)$ der Ordnung $k$ durch
$$
W^{k,q} := \{ u \in L^q \mid D^\alpha u \in L^q(\Omega) \text{ für alle } |\alpha| \leq k\}.
$$
Für $u \in W^{k,q}$ bezeichnen wir $D^\alpha u \in L^q(\Omega)$ als \emph{schwache Ableitung} von $u$.
Hierbei identifizieren wir die reguläre Distribution $D^\alpha u$ immer direkt mit der korrespondierenden $L^q$\hyp{}Funktion.

Wir machen $W^{k,q}$ zu einem normierten Vektorraum durch die \sobolev\hyp{}Norm
$$
  \norm{u}_{W^{k,q}(\Omega)} 
  := \norm{u}_{W^{k,q}} 
  := \norm{u}_{k,q} 
  := \begin{cases} 
    \left(\sum_{|\alpha| \leq k} \norm{D^\alpha u}_q^q \right)^\frac{1}{q}\quad&\text{für } 1 \leq q < \infty \\
    \max_{|\alpha| \leq k} \norm{D^\alpha u}_\infty \quad&\text{für }q = \infty.
    \end{cases}
$$

\begin{lem}
  Für ein Gebiet $\Omega \subseteq \R^n$ mit $n \geq 1$ und $1 \leq q \leq \infty$ ist $W^{k,q} = W^{k,q}(\Omega)$ ausgestattet mit der \sobolev\hyp{}Norm ein \banach\hyp{}Raum.
\end{lem}

\begin{proof}
  Sei $(u_j)_{j \in \N}$ eine \cauchy\hyp{}Folge in $W^{k,q}$.
  Nach Definition ist dann auch $(D^\alpha u_j)_{j \in \N}$ für alle $|\alpha| \leq k$ eine \cauchy\hyp{}Folge in $L^q$.
  Folglich existieren für alle $|\alpha| \leq k$ Funktionen $u_\alpha \in L^q$ mit 
  $$
  \lim_{j \to \infty} \norm{D^\alpha u_j - u_\alpha}_q = 0.
  $$
  Sei $f_0 := f_{(0,\dots,0)}$.
  Wir behaupten nun, dass die Identität $D^\alpha f_0 = f_\alpha$ für alle $|\alpha| \leq k$ gilt.
  Dazu halten wir zunächst fest, dass mit der \hoelder\hyp{}Ungleichung
  $$
  \int_\Omega (D^\alpha f_j - f_\alpha) \varphi \d x
  \leq \norm{(D^\alpha f_j - f_\alpha)}_q \norm{\varphi}_{q'} \to 0
  $$
  sowie
  $$
  \int_\Omega (f_j - f_0) D^\alpha \varphi \d x
  \leq \norm{(f_j - f_0)}_q \norm{\varphi}_{q'} \to 0
  $$
  folgt, wobei $q'$ den zu $q$ konjugierten Exponenten bezeichne.
  Damit erhalten wir
  \begin{align*}
    [f_\alpha, \varphi] 
    &= [\lim_{j \to \infty} D^\alpha f_j, \varphi] \tag{Limes bezüglich $\norm{\cdot}_q$} \\
    &= \lim_{j \to \infty} [D^\alpha f_j, \varphi] \\
    &= \lim_{j \to \infty} (-1)^{|\alpha|}[f_j, D^\alpha \varphi] \\
    &=  (-1)^{|\alpha|}[\lim_{j \to \infty} f_j, D^\alpha \varphi] \\
    &=  (-1)^{|\alpha|}[f_0, D^\alpha \varphi].
  \end{align*}
  Daraus folgt jedoch gerade die Behauptung $D^\alpha f_0 = f_\alpha$.
  Insgesamt ergibt sich also $\lim_{j \to \infty} \norm{f_j - f_0}_{k,q} =  0$, was zu beweisen war.
\end{proof}

Wir definieren den Unterraum
$$
  W_0^{k,q}(\Omega) := \overline{\C_0^\infty(\Omega)}^{\norm{\cdot}_{k,q}},
$$
welcher die glatten Funktionen mit kompaktem Träger als dichte Teilmenge bezüglich der \sobolev\hyp{}Norm enthält.

Bisher hatten wir nur \sobolev\hyp{}Räume positiver Ordnung betrachtet.
Für $1 < q < \infty$ definieren wir die \sobolev\hyp{}Räume negativer Ordnung 
$$
  W^{-1,q}(\Omega) := W_0^{k,q'}(\Omega)'.
$$
Wie für den Dualraum eines normierten Raumes üblich, ist auch $W^{-1,q}(\Omega)$ mit Operatornorm ein \banach\hyp{}Raum.

Es ist möglich einen \sobolev\hyp{}Raum $W^{k,q}(\Omega)$ für $1 < q < \infty$ als abgeschlossenen Teilraum eines $L^q$\hyp{}Raumes zu realisieren \cite{adams2003sobolev}[S.61, 3.5].
Als solcher ist er insbesondere reflexiv.
Wir halten diese wichtige Eigenschaft in einem Lemma fest.

\begin{lem}
  Für ein Gebiet $\Omega \subseteq \R^n$ mit $n \geq 1$ und $1 < q < \infty$ ist $W^{k,q} = W^{k,q}(\Omega)$ ausgestattet mit der \sobolev\hyp{}Norm ein reflexiver Raum. \qed
\end{lem}

Aus der Reflexivität der \sobolev\hyp{}Räume leiten wir für $1 < q < \infty$ die folgende Identität ab:
$$
  W_0^{1,q'}(\Omega) = W^{-1,q}(\Omega)'.
$$
Wir können also jede Funktion $u \in W_0^{1,q'}(\Omega)$ mit dem Funktional
$$
[\cdot, u] \colon F \mapsto [F, u], \quad F \in W^{-1,q}(\Omega)
$$
identifizieren.

Wir definieren nun die $W^{k,q}_{\loc}$\hyp{}Räume.
Wir bezeichnen mit $W^{k,q}(\Omega)$ den aller Funktionen $u$ mit $D^\alpha u \in L^q_{\loc}(\Omega)$ für alle Multiindices $|\alpha| \leq k$.
Des Weiteren definieren definieren wir $W^{k,q}(\overline\Omega)$ als den Raum aller Funktionen $u$ mit $D^\alpha u \in L^q_{\loc}(\overline\Omega)$.

Wie schon im Unterabschnitt zu glatten Funktionen können wir auch die Definition der \sobolev\hyp{}Räue auf Produkträume ausdehnen. 
Wir definieren hierzu
$$
W^{k,q}(\Omega)^m := \{ (u_1,\dots,u_m) \mid u_j \in W^{k,q}(\Omega), \quad j = 1,\dots, m\}
$$
und versehen den so entstandenen Vektorraum mit der Norm
$$
\norm{u}_{W^{k,q}(\Omega)^m} 
:= \norm{u}_{k,q}
:= \left( \sum_{j = 1}^m \norm{u_j}_{k,q}^q \right)^{\frac{1}{q}}
$$

