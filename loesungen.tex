\chapter{Lösungen von $\nabla p = f$}
\section{Lipschitz-Gebiete und Gebietszerlegungen}

\begin{defn}
  \lipschitz\hyp{}Gebiete  
\end{defn}

\begin{lem}
  \label{lem:lipschitzExhaust}
  Sei $\Omega \subseteq \R^n$ mit $n \geq 2$ ein Gebiet.
  Dann existiert eine Folge $(\Omega_j)_{j \in \N}$ beschränkter \lipschitz\hyp{}Gebiete $\Omega_j \subseteq \Omega$ und eine Folge $(\varepsilon_j)_{j \in \N}$ positiver reeller Zahlen mit folgenden Eigenschaften:
  \begin{enumerate}[a)]
    \item Für alle $j \in \N$ gilt $\overline{\Omega_j} \subseteq \Omega_{j + 1}$.
    \item Für alle $j \in \N$ gilt $\dist( \partial \Omega_{j + 1}, \Omega_j) \geq \varepsilon_{j + 1}$. 
    \item Es gilt $\lim_{j \to \infty} \varepsilon_j = 0$.
    \item Die Gebiete $\Omega_j$ schöpfen $\Omega$ aus.
  \end{enumerate}
\end{lem}

\begin{proof}
Im Folgenden bezeichne $B_r(x) \subseteq \R^n$ den bezüglich \euklid ischer Topologie offenen Ball mit Radius $r$ und Mittelpunkt $x$.

  Für ein festgewähltes $x_0 \in \Omega$ betrachten wir den Schnitt 
  $$
  \Omega' := \Omega \cap B_1(x_0).
  $$ 
  Als Schnitt offener Mengen ist $\Omega'$ wiederum offen. 
  Bezüglich der Teilraumtopologie muss $\Omega'$ jedoch nicht zwingend zusammenhängend sein.
  Wir bezeichnen nun mit $\widetilde\Omega_1$ die Zusammenhangskomponente von $\Omega'$, welche $x_0$ enthält.
  Da die Zusammenhangskomponenten eines topologischen Raumes immer eine Partition desselben bilden, ist $\Omega'$ eindeutig bestimmt.
  Insbesondere gilt für den Rand
  $$ 
  \partial \widetilde\Omega \subseteq \overline{B_1(x_0)}, 
  $$
  er ist somit als abgeschlossene Teilmenge des Kompaktums $\overline{B_1(x_0)}$ selbst kompakt.
  Für alle $\varepsilon > 0$ lässt sich daher $\partial \widetilde\Omega_1$ durch endlich viele Bälle $B\varepsilon(x_j)$, mit $x_j \in \partial \widetilde\Omega_1$ für alle $j = 1,\dots,m$, überdecken:
  $$ 
  \partial \widetilde\Omega_1 \subseteq \bigcup_{j = 1}^m B_\varepsilon(x_j).
  $$

  Wir definieren nun 
  $$
  \widehat\Omega_1 := \widetilde\Omega_1 \setminus \bigcap_{j = 1}^m \overline{B_\varepsilon(x_j)}
  $$
  und wählen $0 < \varepsilon < 1$ so klein, dass zusätzlich $x_0 \in \widehat\Omega_1$ gilt. 
  Dies lässt sich immer erreichen, da $\widetilde\Omega$ als bezüglich Teilraumtopologie offen-abgeschlossene Menge in $\Omega'$ auch in $\R^n$ offen ist und daher ein $\delta > 0$ mit $B_\delta(x_0) \subseteq \Omega'$ existiert.
  Hiermit besitzt bereits ein $\varepsilon < \dist(x_0, \partial\widetilde\Omega) - \delta$ die geforderte Eigenschaft.

  Man erkennt nun $\widehat\Omega_1$ als beschränktes \lipschitz\hyp{}Gebiet, da $\partial\widehat\Omega_1$ sämtlich aus Teilen der Ränder der Bälle $B_\varepsilon(x_j)$ besteht.
  Wir setzen nun $\Omega_1 := \widehat\Omega_1$ und $\epsilon_1 := \epsilon$ und führen diese Konstruktion weiter fort.

  Wir wählen wieder 
  $$
  \widetilde\Omega_2 \subseteq \Omega \cap B_2(x_0)
  $$
  als die $x_0$ enthaltende Zusammenhangskomponente des Schnitts von $\Omega$ und $B_2(x_0)$ und konstruieren analog zum ersten Schritt ein Gebiet $\widehat\Omega_2$ mit $0 < \varepsilon < \tfrac{1}{2}$ und $\varepsilon < \dist(\partial\widetilde\Omega_2, \Omega_1)$.
  
\end{proof}

\begin{itemize}
  \item \cite{sohr2001navier}[S.55, Lemma 1.4.1]
\end{itemize}

\newpage
\section{Kompakte Einbettungen}

\begin{itemize}
  \item \cite{sohr2001navier}[S.58, Lemma 1.5.4]
\end{itemize}

\newpage
\section{Darstellung von Funktionalen}

\begin{itemize}
  \item \cite{sohr2001navier}[S.61, Lemma 1.6.1]
\end{itemize}


\section{Die Glättungsmethode}

\begin{itemize}
  \item \cite{sohr2001navier}[S.64ff.]
\end{itemize}

\section{Das Gradientenkriterium}

\begin{itemize}
  \item \cite{sohr2001navier}[Lemma 2.2.1, S.73]
\end{itemize}

