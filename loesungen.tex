\chapter{Lösungen von $\nabla p = f$}
\section{Lipschitz-Gebiete und Gebietszerlegungen}

\begin{defn}
  \lipschitz\hyp{}Gebiete  
\end{defn}

\begin{lem}
  \label{lem:distBoundary}
  Seien $\emptyset \subsetneq A,B \subsetneq \R^n$. Gilt $A \subseteq B$, so folgt $\dist(x,\partial A) \leq \dist(x \partial B)$ für alle $x \in A$.
\end{lem}

\begin{bem}
  Wie aus dem Beweis von Lemma \ref{lem:distBoundary} ersichtlich ist, reicht es bereits aus $A$ und $B$ als Teilmengen der konvexen Hülle von $B$ mit euklidischer Spurmetrik und entsprechender Abstandsfunktion $\dist$ zu betrachten.
  Die Konvexität ist hierbei notwendig, wie man an einem Beispiel zeigen kann.
\end{bem}

\begin{lem}
  \label{lem:lipschitzExhaust}
  Sei $\Omega \subseteq \R^n$ mit $n \geq 2$ ein Gebiet.
  Dann existiert eine Folge $(\Omega_j)_{j \in \N}$ beschränkter \lipschitz\hyp{}Gebiete $\Omega_j \subseteq \Omega$ und eine Folge $(\varepsilon_j)_{j \in \N}$ positiver reeller Zahlen mit folgenden Eigenschaften:
  \begin{enumerate}[a)]
    \item Für alle $j \in \N$ gilt $\overline{\Omega}_j \subseteq \Omega_{j + 1}$.
    \item Für alle $j \in \N$ gilt $\varepsilon_{j + 1} \leq \dist(\Omega_j, \partial \Omega_{j + 1}).$
    \item Es gilt $\lim_{j \to \infty} \varepsilon_j = 0$.
    \item Die Gebiete $\Omega_j$ schöpfen $\Omega$ aus.
  \end{enumerate}
\end{lem}

\begin{proof}
Im Folgenden bezeichne $B_r(x) \subseteq \R^n$ den bezüglich \euklid ischer Topologie offenen Ball mit Radius $r$ und Mittelpunkt $x$.

  Für ein festgewähltes $x_0 \in \Omega$ betrachten wir den Schnitt 
  $$
  \Omega' := \Omega \cap B_1(x_0).
  $$ 
  Als Schnitt offener Mengen ist $\Omega'$ wiederum offen. 
  Bezüglich der Teilraumtopologie muss $\Omega'$ jedoch nicht zwingend zusammenhängend sein.
  Wir bezeichnen nun mit $\widetilde\Omega_1$ die Zusammenhangskomponente von $\Omega'$, welche $x_0$ enthält.
  Da die Zusammenhangskomponenten eines topologischen Raumes immer eine Partition desselben bilden, ist $\Omega'$ eindeutig bestimmt.
  Insbesondere gilt für den Rand
  $$ 
  \partial \widetilde\Omega_1 \subseteq \overline{B_1(x_0)}, 
  $$
  er ist somit als abgeschlossene Teilmenge des Kompaktums $\overline{B_1(x_0)}$ selbst kompakt.
  Für alle $\varepsilon > 0$ lässt sich daher $\partial \widetilde\Omega_1$ durch endlich viele Bälle $B_\varepsilon(x_j)$, mit $x_j \in \partial \widetilde\Omega_1$ für alle $j = 1,\dots,m$, überdecken:
  $$ 
  \partial \widetilde\Omega_1 \subseteq \bigcup_{j = 1}^m B_\varepsilon(x_j).
  $$

  Wir definieren nun 
  $$
  \widehat\Omega_1 := \widetilde\Omega_1 \setminus \bigcap_{j = 1}^m \overline{B_\varepsilon(x_j)}
  $$
  und wählen $0 < \varepsilon < 1$ so klein, dass zusätzlich $x_0 \in \widehat\Omega_1$ gilt. 
  Dies lässt sich immer erreichen, da $\widetilde\Omega_1$ als bezüglich Teilraumtopologie offen-abgeschlossene Menge in $\Omega'$ auch in $\R^n$ offen ist und daher ein $\delta > 0$ mit $B_\delta(x_0) \subseteq \Omega'$ existiert.
  Hiermit besitzt bereits ein $\varepsilon < \dist(x_0, \partial\widetilde\Omega_1) - \delta$ die geforderte Eigenschaft.

  Man erkennt nun $\widehat\Omega_1$ als beschränktes \lipschitz\hyp{}Gebiet, da $\partial\widehat\Omega_1$ sämtlich aus Teilen der Ränder der Bälle $B_\varepsilon(x_j)$ besteht.
  Wir setzen nun $\Omega_1 := \widehat\Omega_1$ und $\epsilon_1 := \epsilon$ und führen diese Konstruktion weiter fort.

  Wir wählen wieder 
  $$
  \widetilde\Omega_2 \subseteq \Omega \cap B_2(x_0)
  $$
  als die $x_0$ enthaltende Zusammenhangskomponente des Schnitts von $\Omega$ und $B_2(x_0)$ und konstruieren analog zum ersten Schritt ein Gebiet $\widehat\Omega_2$ mit $0 < \varepsilon < \tfrac{1}{2}$ und $\varepsilon < \dist(\Omega_1, \partial\widetilde\Omega_2)$.
  Dies ist jedoch nur möglich, falls $0 < \dist(\Omega_1, \partial\widetilde\Omega_2)$ gilt, was wir im Folgenden beweisen werden.

  Zunächst gilt nach Konstruktion die Inklusionskette
  $$
  \widehat\Omega_1 \subseteq \widetilde\Omega_1 \subseteq \widetilde\Omega_2.
  $$
  Hieraus folgt mit Lemma \ref{lem:distBoundary}, dass
  \begin{displaymath}
    \dist(x,\partial\widehat\Omega_1) 
    \leq \dist(x,\partial\widetilde\Omega_1) 
    \leq \dist(x,\partial\widetilde\Omega_2) \tag{$\ast$}
  \end{displaymath}
  für alle $x \in \widehat\Omega_1$ gilt.
  Des Weiteren gilt 
  \begin{displaymath}
  0 < \lambda \leq \dist(\widehat\Omega_1,\partial\widetilde\Omega_1) \tag{$\ast\ast$},
  \end{displaymath}
  wobei $\lambda$ die \lebesgue sche Zahl der Überdeckung $B_{\varepsilon_1}(x_1),\dots,B_{\varepsilon_1}(x_m)$ von $\partial\widetilde\Omega_1$ aus dem ersten Schritt des Beweises bezeichne.
  Die Ungleichungen ($\ast$) und ($\ast\ast$) zusammen ergeben nun die Behauptung.

  Setzen wir noch $\Omega_2 := \widehat\Omega_2$ und $\varepsilon_2 := \varepsilon$, so erhalten wir einerseits $\overline\Omega_1 \subseteq \Omega_2$, denn $\Omega_1 \subseteq \Omega_2$ gilt nach Konstruktion, sowie $0 < \dist(x,\partial\widehat\Omega_2) =: d$ für alle $x \in \partial\widehat\Omega_1$. Dann gilt aber auch $B_{\frac{d}{2}}(x) \subseteq \widehat\Omega_2$, also insbesondere $x \in \widehat\Omega_2$ für alle $x \in \partial\widehat\Omega_1$. Damit folgt 
  $$
  \widehat\Omega_1 \cup \partial\widehat\Omega_1 = \overline{\widehat\Omega}_1 \subseteq \widehat\Omega_2.
  $$

  Andererseits gilt $\varepsilon_2 < \dist(\Omega_1, \partial\Omega_2)$, denn
  \begin{align*}
  \varepsilon_2 
  &< \frac{1}{2} \dist(\Omega_1, \partial\widetilde\Omega_2) \\
  & \leq \frac{1}{2} (\dist(\Omega_1, \partial\widehat\Omega_2) + \dist(\partial\widehat\Omega_2, \widetilde\Omega_2)) \\
  & \leq \frac{1}{2} (\dist(\Omega_1, \partial\widehat\Omega_2) + \varepsilon_2).
  \end{align*}

  Setzt man das beschriebene Vorgehen induktiv fort, so erhält man eine Folge $(\Omega_j)_{j \in \N}$ von \lipschitz\hyp{}Gebieten und eine Folge $(\varepsilon_j)_{j \in \N}$ für die nach Konstruktion $0 < \varepsilon_j < \tfrac{1}{j}$ für alle $j \in \N$ gilt.
  Die Eigenschaften a), b) und c) werden also erfüllt.

  Es gilt noch zu zeigen, dass die so konstruierte Folge $(\Omega_j)_{j \in \N}$ auch Eigenschaft d) erfüllt, also $ \Omega \subseteq \bigcup_{j \in \N} \Omega_j $ gilt.
  Sei dazu $x \in \Omega$ beliebig.
  Weil $\Omega$ zusammenhängend ist, existiert ein $j_0 \in \N$, sodass
  $$
  x \in \widetilde\Omega_{j_0} \subseteq \Omega \cap B_{j_0}(x_0)
  $$
  gilt.
  Sei $d:=\dist(x, \partial\widetilde\Omega_{j_0})$.
  Dann existiert ein $j_1 > j_0$ mit $\varepsilon_{j_1} < d$.
  Da die Inklusion $\widetilde\Omega_{j_0} \subseteq \widetilde\Omega_{j_1}$ gilt, folgt mit Lemma \ref{lem:distBoundary} die Ungleichung $\varepsilon_{j_1} < \dist(x,\partial\widetilde\Omega_{j_1})$, was wiederum $x \in \widehat\Omega_{j_1} = \Omega_{j_1}$ impliziert.
  Damit gilt auch Eigenschaft d).
\end{proof}

\section{Kompakte Einbettungen}

\begin{itemize}
  \item \cite{sohr2001navier}[S.58, Lemma 1.5.4]
\end{itemize}

\newpage
\section{Darstellung von Funktionalen}

\begin{lem}
  \label{lem:divRepresentation}
  Sei $\Omega \subseteq \R^n$ mit $n \geq 2$ ein beschränktes Gebiet und $f \in W^{-1,q}(\Omega)^n$ mit $1 < q < \infty$.
  Dann existiert eine Matrix $F \in L^q(\Omega)^{n^2}$, welche die Gleichung
  $$ f = \div F $$
  im distributionellen Sinne und die Ungleichungen
  $$
  || f ||_{W^{-1,q}(\Omega)^n} 
  \leq || F ||_{L^q(\Omega)^{n^2}} 
  \leq C || f ||_{W^{-1,q}(\Omega)^n}
  $$
  mit $C = C(\Omega) > 0$ erfüllt.
\end{lem}

\begin{proof}
  Wir betrachten den Raum
  $$
  D := \{\nabla v \in L^{q'}(\Omega)^{n^2} \colon v \in W_0^{1,q'}(\Omega)^n\} \subseteq L^{q'}(\Omega)^{n^2}
  $$
  der Gradienten $\nabla v = (D_j v_l)_{j,l=1}^n$ von Funktionen $v = (v_1,\dots,v_n) \in W_0^{1,q'}(\Omega)^n$.
  Wir definieren das Funktional
  $$
  \tilde f \colon \nabla v \mapsto [\tilde f, \nabla v]\;, \quad \nabla v \in D
  $$
  durch $[\tilde f, \nabla v] := [f, v]$ für alle $v \in W_0^{1,q'}(\Omega)^n$.
  Dann liefert die \poincare\hyp{}Ungleichung zusammen mit der \hoelder\hyp{}Ungleichung eine Konstante $C = C(\Omega) > 0$, sodass 
  $$
  |[\tilde f, \nabla v]| 
  = |[f, v]| 
  \leq ||f||_{-1,q} ||v||_{1,q'}
  \leq C ||f||_{-1,q} ||\nabla v ||_{q'}
  $$
  für alle $\nabla v \in D$ gilt.
  Somit ist $\tilde f$ ein stetiges Funktional auf $D \subseteq L^{q'}(\Omega)^{n^2}$.
  Der Satz von \hahn\hyp{}\banach\ liefert eine normgleiche Fortsetzung von von $D$ nach $L^{q'}(\Omega)^{n^2}$.
  Nach dem Darstellungssatz über Funktionale existiert nun eine Matrix $F \in L^{q}(\Omega)^{n^2}$ mit
  $$
  \langle F, \nabla v \rangle
  = \sum_{j,l=1}^n \int_\Omega F_{jl}(D_j v_l) \d x
  = \int_\Omega F \cdot \nabla v \d x
  = [\tilde f, \nabla v] 
  = [f, v]
  $$
\end{proof}

\begin{itemize}
  \item \cite{sohr2001navier}[S.61, Lemma 1.6.1]
\end{itemize}


\section{Das Gradientenkriterium}

\begin{lem}
  Sei $\Omega \subseteq \R^n$ mit $n \geq 2$ ein Gebiet $\Omega_0 \subseteq \Omega$ ein beschränktes Teilgebiet mit $\emptyset \neq \overline\Omega_0 \subseteq \Omega$ und $1 < q < \infty$.
  Angenommen, für $f \in W_{\loc}^{-1,q}(\Omega)^n$ gelte
  \begin{equation}
    \label{eq:vKernel}
    [f,v] = 0, \quad \text{für alle} \quad v \in C_{0,\sigma}^\infty(\Omega).
  \end{equation}
  Dann existiert ein eindeutig bestimmtes $p \in L^q_{\loc}(\Omega)$, welches die Gleichung $\nabla p = f$ im distributionellen Sinne erfüllt und für das zusätzlich
  \begin{equation}
    \int_{\Omega_0} p \d x = 0
    \label{eq:0integral}
  \end{equation}
  gilt.
\end{lem}

\begin{proof}
  Wir zeigen zunächst, dass für ein beliebiges beschränktes \lipschitz\hyp{}Gebiet $\Omega_1 \subseteq \Omega$ mit $\overline\Omega_0 \subseteq \Omega_1 \subseteq \overline\Omega_1 \subseteq \Omega$ ein eindeutig bestimmtes $p \in L^q(\Omega_1)$ existiert, welches die Behauptung des Lemmas erfüllt.

  Ähnlich zum ersten Beweisschritt von Lemma \ref{lem:lipschitzExhaust} finden wir ein weiteres beschränktes \lipschitz\hyp{}Gebiet $\Omega_2$ mit $\overline\Omega_1 \subseteq \Omega_2 \subseteq \overline\Omega_2 \subseteq \Omega$.
  Dazu wählen wir ein $x_0 \in \Omega_1$ und finden aufgrund der vorausgesetzten Beschränktheit von $\Omega_1$ ein $r > 0$, sodass $\Omega_1 \subseteq B_r(x_0)$ gilt.
  Wir wählen sodann die $x_0$ enthaltende Zusammenhangskomponente $\widetilde\Omega_2$ von $B_r(x_0) \cap \Omega$ aus und konstruieren wie schon im Beweis von Lemma \ref{lem:lipschitzExhaust} das beschränkte \lipschitz\hyp{}Gebiet $\Omega_2 = \widehat\Omega_2$.

  Der Voraussetzung $f \in W_{\loc}^{-1,q}(\Omega)^n$ entnehmen wir, dass $f \in W^{-1,q}(\Omega_2)^n$ gilt. Zudem existiert aufgrund der Beschränktheit von $\Omega_2$ nach Lemma \ref{lem:divRepresentation} eine Darstellung
  $$
  f = \div F \quad\text{mit}\quad F = (F_{jl})_{j,l=1}^n \in L^q(\Omega_2)^{n^2}.
  $$

\end{proof}
