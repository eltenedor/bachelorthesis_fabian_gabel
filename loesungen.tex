\chapter{Lösungen von $\nabla p = f$}

In diesem Kapitel sammeln und beweisen wir eine Reihe von Hilfsaussagen für den Beweis der \helmholtz\hyp{}Zerlegung.
Die für diese Zerlegung wesentliche Darstellbarkeit von $L^2$\hyp{}Funktionen durch distributionelle Gradientenfelder wird das zentrale Resultat sein, auf das wir in den kommenden Abschnitten zuarbeiten.


\section{Lipschitz-Gebiete und Gebietsapproximation}
\label{sec:lipschitzDomains}

In diesem Abschnitt stellen wir einige technische Lemmata vor, die es ermöglichen werden beliebige Gebiete durch Gebiete deren Ränder eine höhere Regularität aufweisen auszuschöpfen.
Im Folgenden sei $\R^n$ immer mit der \euklid ischen Metrik versehen. Es bezeichne zudem
$$
\dist(X,Y) := \inf\{ \norm{x - y} \mid x \in X, y \in Y\}
$$
den \euklid ischen Abstand zweier Mengen $X,Y \subseteq \R^n$.

\begin{defn}
  Ein Gebiet $\Omega \subseteq \R^n$ mit $n \geq 2$ und $\partial \Omega \neq \emptyset$ heißt \lipschitz\hyp{}Gebiet, falls der Rand lokal der Graph einer \lipschitz\hyp{}stetigen Funktion ist.
\end{defn}

Diese Definition lässt sich noch weiter präzisieren, wie in \cite[S.25, 3.2]{sohr2001navier} nachzulesen ist. Für unsere Zwecke reicht sie jedoch bereits aus.

\begin{lem}
  \label{lem:distBoundary}
  Seien $\emptyset \subsetneq A,B \subsetneq \R^n$. Gilt $A \subseteq B$, so folgt $\dist(a,\partial A) \leq \dist(a, \partial B)$ für alle $a \in A$.
\end{lem}

\begin{proof}
  Da $\R^n$ ein zusammenhängender Raum ist, sind die einzigen Mengen mit leerem Rand die leere Menge und der ganze Raum.
  Dies wurde jedoch in der Voraussetzung des Lemmas bereits ausgeschlossen, daher nimmt $\dist$ nur endliche Werte an.

  Sei nun $a \in A$ und $b \in \partial B$
  Wir betrachten den Strahl $s \colon [0,1] \to \mathbb{R}^n$ mit $s(0) = a$ und $s(1) = b$.
  Zudem können wir annehmen, dass $a,b \not\in \partial A$ gilt, da ansonsten die Ungleichung sofort erfüllt ist.
  Wir wollen im Folgenden nachweisen, dass ein $t' \in (0,1)$ mit $s(t') \in \partial A$ existiert.

  Dazu definieren wir die Funktion
  $$
  f(x) := (1 - 2\chi_A(x)) \cdot \dist(x, \partial A), \quad x \in \R^n,
  $$
  wobei $\chi_A$ die charakteristische Funktion der Menge $A$ bezeichne.
  Als Nächstes weisen wir nach, dass $f$ auf $\R^n$ stetig ist.
  Es ist bekannt, dass die Funktion $\dist(\cdot, \partial A)$ stetig ist \cite[S.14]{koenigsberger2004ana2}.
  Auf $\R^n \setminus \partial A$ ist zudem $(1 - 2\chi_A(x))$ konstant gleich $-1$ beziehungsweise $1$, also ist $f$ dort stetig.

  Sei nun $(x_n)_{n \in \N}$ eine Folge in $\R^n$ mit $\lim_{n \to \infty} x_n = x \in \partial A$.
  Dann gilt
  $$
  |(1 - 2\chi_A(x_n)) \cdot \dist(x_n, \partial A)| 
  \leq \dist(x_n,\partial A)
  \to \dist(x,\partial A) 
  = 0,
  $$
  also auch 
  $$
  \lim_{n \to \infty} f(x_n) = 0 = f(x),
  $$
  da aufgrund der Abgeschlossenheit von $\partial A$ die Gleichheit $\dist(x,\partial A) = 0$ äquivalent zu $x \in \partial A$ ist ($\ast$).

  Auf $f \circ s$ lässt sich nun der Zwischenwertsatz anwenden, denn nach Voraussetzung gelten
  $$
  (f \circ s)(0) = f(a) = -1 \quad\text{und}\quad (f \circ s)(1) = f(b) = 1.
  $$
  Somit existiert also ein $t' \in (0,1)$ mit $(f\circ s)(t') = f(s(t')) = 0$, was nach ($\ast$) äquivalent zu $s(t') \in \partial A$ ist.
  Es ist also
  $$
  \norm{a - b} \geq \norm{a - s(t')} - \norm{s(t') - b} \geq \norm{a - s(t')} \geq \dist(x,\partial A).
  $$
  Da dies für alle $b \in \partial B$ gilt folgt sogleich $ \dist(a, \partial B) \geq \dist(a, \partial A)$.
\end{proof}

%\begin{bem}
  %Wie aus dem Beweis von Lemma \ref{lem:distBoundary} ersichtlich ist, reicht es bereits aus $A$ und $B$ als Teilmengen der konvexen Hülle von $B$ mit euklidischer Spurmetrik und entsprechender Abstandsfunktion $\dist$ zu betrachten.
  %Die Konvexität ist hierbei notwendig, wie man an einem Beispiel zeigen kann.
%\end{bem}

\begin{lem}
  \label{lem:lipschitzExhaust}
  Sei $\Omega \subseteq \R^n$ mit $n \geq 2$ ein Gebiet.
  Dann existiert eine Folge $(\Omega_j)_{j \in \N}$ beschränkter \lipschitz\hyp{}Gebiete $\Omega_j \subseteq \Omega$ und eine Folge $(\varepsilon_j)_{j \in \N}$ positiver reeller Zahlen mit folgenden Eigenschaften:
  \begin{enumerate}[a)]
    \item Für alle $j \in \N$ gilt $\overline{\Omega}_j \subseteq \Omega_{j + 1}$.
    \item Für alle $j \in \N$ gilt $\varepsilon_{j + 1} \leq \dist(\Omega_j, \partial \Omega_{j + 1}).$
    \item Es gilt $\lim_{j \to \infty} \varepsilon_j = 0$.
    \item Die Gebiete $\Omega_j$ schöpfen $\Omega$ aus.
  \end{enumerate}
\end{lem}

\begin{proof}
Im Folgenden bezeichne $B_r(x) \subseteq \R^n$ den bezüglich \euklid ischer Topologie offenen Ball mit Radius $r$ und Mittelpunkt $x$.

  Für ein festgewähltes $x_0 \in \Omega$ betrachten wir den Schnitt 
  $$
  \Omega' := \Omega \cap B_1(x_0).
  $$ 
  Als Schnitt offener Mengen ist $\Omega'$ wiederum offen. 
  Bezüglich der Teilraumtopologie muss $\Omega'$ jedoch nicht zwingend zusammenhängend sein.
  Wir bezeichnen nun mit $\widetilde\Omega_1$ die Zusammenhangskomponente von $\Omega'$, welche $x_0$ enthält.
  Da die Zusammenhangskomponenten eines topologischen Raumes immer eine Partition desselben bilden, ist $\Omega'$ eindeutig bestimmt.
  Insbesondere gilt für den Rand
  $$ 
  \partial \widetilde\Omega_1 \subseteq \overline{B_1(x_0)}, 
  $$
  er ist somit als abgeschlossene Teilmenge des Kompaktums $\overline{B_1(x_0)}$ selbst kompakt.
  Für alle $\varepsilon > 0$ lässt sich daher $\partial \widetilde\Omega_1$ durch endlich viele Bälle $B_\varepsilon(x_j)$, mit $x_j \in \partial \widetilde\Omega_1$ für alle $j = 1,\dots,m$, überdecken:
  $$ 
  \partial \widetilde\Omega_1 \subseteq \bigcup_{j = 1}^m B_\varepsilon(x_j).
  $$

  Wir definieren nun 
  $$
  \widehat\Omega_1 := \widetilde\Omega_1 \setminus \bigcap_{j = 1}^m \overline{B_\varepsilon(x_j)}
  $$
  und wählen $0 < \varepsilon < 1$ so klein, dass zusätzlich $x_0 \in \widehat\Omega_1$ gilt. 
  Dies lässt sich immer erreichen, da $\widetilde\Omega_1$ als bezüglich Teilraumtopologie offen-abgeschlossene Menge in $\Omega'$ auch in $\R^n$ offen ist und daher ein $\delta > 0$ mit $B_\delta(x_0) \subseteq \Omega'$ existiert.
  Hiermit besitzt bereits ein $\varepsilon < \dist(x_0, \partial\widetilde\Omega_1) - \delta$ die geforderte Eigenschaft.

  Man erkennt nun $\widehat\Omega_1$ als beschränktes \lipschitz\hyp{}Gebiet, da $\partial\widehat\Omega_1$ sämtlich aus Teilen der Ränder der Bälle $B_\varepsilon(x_j)$ besteht.
  Wir setzen nun $\Omega_1 := \widehat\Omega_1$ und $\epsilon_1 := \epsilon$ und führen diese Konstruktion weiter fort.

  Wir wählen wieder 
  $$
  \widetilde\Omega_2 \subseteq \Omega \cap B_2(x_0)
  $$
  als die $x_0$ enthaltende Zusammenhangskomponente des Schnitts von $\Omega$ und $B_2(x_0)$ und konstruieren analog zum ersten Schritt ein Gebiet $\widehat\Omega_2$ mit $0 < \varepsilon < \tfrac{1}{2}$ und $\varepsilon < \dist(\Omega_1, \partial\widetilde\Omega_2)$.
  Dies ist jedoch nur möglich, falls $0 < \dist(\Omega_1, \partial\widetilde\Omega_2)$ gilt, was wir im Folgenden beweisen werden.

  Zunächst gilt nach Konstruktion die Inklusionskette
  $$
  \widehat\Omega_1 \subseteq \widetilde\Omega_1 \subseteq \widetilde\Omega_2.
  $$
  Hieraus folgt mit Lemma \ref{lem:distBoundary}, dass
  \begin{displaymath}
    \dist(x,\partial\widehat\Omega_1) 
    \leq \dist(x,\partial\widetilde\Omega_1) 
    \leq \dist(x,\partial\widetilde\Omega_2) \tag{$\ast$}
  \end{displaymath}
  für alle $x \in \widehat\Omega_1$ gilt.
  Des Weiteren gilt 
  \begin{displaymath}
  0 < \lambda \leq \dist(\widehat\Omega_1,\partial\widetilde\Omega_1) \tag{$\ast\ast$},
  \end{displaymath}
  wobei $\lambda$ die \lebesgue\hyp{}Zahl der Überdeckung $B_{\varepsilon_1}(x_1),\dots,B_{\varepsilon_1}(x_m)$ von $\partial\widetilde\Omega_1$ aus dem ersten Schritt des Beweises bezeichne.
  Die Ungleichungen ($\ast$) und ($\ast\ast$) zusammen ergeben nun die Behauptung.

  Setzen wir noch $\Omega_2 := \widehat\Omega_2$ und $\varepsilon_2 := \varepsilon$, so erhalten wir einerseits $\overline\Omega_1 \subseteq \Omega_2$, denn $\Omega_1 \subseteq \Omega_2$ gilt nach Konstruktion, sowie $0 < \dist(x,\partial\widehat\Omega_2) =: d$ für alle $x \in \partial\widehat\Omega_1$. Dann gilt aber auch $B_{\frac{d}{2}}(x) \subseteq \widehat\Omega_2$, also insbesondere $x \in \widehat\Omega_2$ für alle $x \in \partial\widehat\Omega_1$. Damit folgt 
  $$
  \widehat\Omega_1 \cup \partial\widehat\Omega_1 = \overline{\widehat\Omega}_1 \subseteq \widehat\Omega_2.
  $$

  Andererseits gilt $\varepsilon_2 < \dist(\Omega_1, \partial\Omega_2)$, denn
  \begin{align*}
  \varepsilon_2 
  &< \frac{1}{2} \dist(\Omega_1, \partial\widetilde\Omega_2) \\
  & \leq \frac{1}{2} (\dist(\Omega_1, \partial\widehat\Omega_2) + \dist(\partial\widehat\Omega_2, \widetilde\Omega_2)) \\
  & \leq \frac{1}{2} (\dist(\Omega_1, \partial\widehat\Omega_2) + \varepsilon_2).
  \end{align*}

  Setzt man das beschriebene Vorgehen induktiv fort, so erhält man eine Folge $(\Omega_j)_{j \in \N}$ von \lipschitz\hyp{}Gebieten und eine Folge $(\varepsilon_j)_{j \in \N}$ für die nach Konstruktion $0 < \varepsilon_j < \tfrac{1}{j}$ für alle $j \in \N$ gilt.
  Die Eigenschaften a), b) und c) werden also erfüllt.

  Es gilt noch zu zeigen, dass die so konstruierte Folge $(\Omega_j)_{j \in \N}$ auch Eigenschaft d) erfüllt, also $ \Omega \subseteq \bigcup_{j \in \N} \Omega_j $ gilt.
  Sei dazu $x \in \Omega$ beliebig.
  Weil $\Omega$ zusammenhängend ist, existiert ein $j_0 \in \N$, sodass
  $$
  x \in \widetilde\Omega_{j_0} \subseteq \Omega \cap B_{j_0}(x_0)
  $$
  gilt.
  Sei $d:=\dist(x, \partial\widetilde\Omega_{j_0})$.
  Dann existiert ein $j_1 > j_0$ mit $\varepsilon_{j_1} < d$.
  Da die Inklusion $\widetilde\Omega_{j_0} \subseteq \widetilde\Omega_{j_1}$ gilt, folgt mit Lemma \ref{lem:distBoundary} die Ungleichung $\varepsilon_{j_1} < \dist(x,\partial\widetilde\Omega_{j_1})$, was wiederum $x \in \widehat\Omega_{j_1} = \Omega_{j_1}$ impliziert.
  Damit gilt auch Eigenschaft d).
\end{proof}

\begin{bemnumber}
  \label{bem:boundedSubset}
  Mit der im Beweis von Lemma \ref{lem:lipschitzExhaust} verwendeten Konstruktion gilt nun auch, dass für alle beschränkten Teilgebiete $\Omega' \subseteq \Omega$ mit $\overline{\Omega'} \subseteq \Omega$ ein $j \in \N$ existiert, sodass $\Omega' \subseteq \Omega_j$
\end{bemnumber}

\section{Kompakte Einbettungen}

Auch in diesem Abschnitt stellen wir technische Aussagen vor, die wir im Folgenden benötigen werden.
Diesmal handelt es sich um Einbettungseigenschaften von \sobolev\hyp{}Räumen.
Zu Beginn notieren wir eine Einbettungseigenschaft, welche sich aus einem Spezialfall des Einbettungssatzes von \rellich\hyp{}\kondrachov\ ergibt. 
Wir verzichten auf einen Beweis und verweisen auf \cite[S.168, Theorem 6.3]{adams2003sobolev}.

\begin{lem}
  \label{lem:compactEmbedding0}
  Sei $\Omega \subseteq \R^n$ mit $n \geq 1$ ein beschränktes Gebiet und $1 < q < \infty$.
  Dann ist die Einbettung
  $$
  L^q(\Omega) \subseteq W^{-1,q}(\Omega)
  $$
  kompakt.\qed
\end{lem}

Dieses Einbettungsresultat können wir nun nutzen, um die folgende Abschätzung zu beweisen.

\begin{lem}
  \label{lem:compactEmbedding}
  Sei $\Omega \subseteq \R^n$ mit $n \geq 2$ ein beschränktes \lipschitz\hyp{}Gebiet und $\Omega_0 \subseteq \Omega$ ein nichtleeres Teilgebiet.
  Zudem sei $1 < q < \infty$.
  Dann gilt die Ungleichung
  \begin{equation}
    \label{eq:compactInequality}
    \norm{u}_{L^q(\Omega)} 
    \leq C_1 \norm{\nabla u}_{W^{-1,q}(\Omega)^n}
    \leq C_1 C_2 \norm{u}_{L^q(\Omega)}
  \end{equation}
  für alle $u \in L^q(\Omega)$, welche die Integralgleichung
  $$
  \int_{\Omega_0} u \d x = 0
  $$
  erfüllen.
  Hierbei bezeichnen $C_1 = C_1(q, \Omega, \Omega_0) > 0$ sowie $C_2 = C_2(n) > 0$ Konstanten.
\end{lem}

\begin{proof}
  Wir halten zunächst fest, dass für alle Folgen $(v_j)_{j \in \N}$ mit $v_j \in C_0^\infty(\Omega)^n$ für alle $j \in \N$ und $$\lim_{j \to \infty} \norm{v_j - v}_{1,q} = 0$$ für ein $v \in \overline{C_0^\infty(\Omega)^n}^{\norm{\cdot}_{1,q}}$ auch $$\lim_{j \to \infty} \norm{\div v_j - \div v} = 0$$ gilt.  
  Hiermit ergibt sich für ein $u \in L^q(\Omega)$ folgende Darstellung der linearen Fortsetzung der Distribution $\nabla u \in (C_0^\infty(\Omega)^n)'$ auf den Raum $\overline{C_0^\infty(\Omega)^n}^{\norm{\cdot}_{1,q}}$:
  $$
    [\nabla u, \cdot] \colon v \to [\nabla u, v] 
    = - \langle u, \div v \rangle 
    = - \int_{\Omega} u\; \div v \d x.
  $$
  Wir können somit $\nabla u$ als Element von $W^{-1,q}(\Omega)$ auffassen.
  Mit dieser Eigenschaft lässt sich der zweite Teil der Ungleichung (\ref{eq:compactInequality}) beweisen. 
  Es gilt nämlich für alle $v \in W_0^{1,q'}(\Omega)^n$
  \begin{align*}
    |[\nabla u, v]|
    &= | \langle u, \div v \rangle| \\
    &\leq \norm{u}_q \norm{\div v}_{q'} \tag{\hoelder}\\
    &\leq \sum_{|\alpha| = 1} \norm{u}_q \norm{D^\alpha v}_{q'} \\
    &\leq C_2 \norm{u}_q \norm{v}_{W^{1,q'}(\Omega)^n},
  \end{align*}
  mit einem aus der Normäquivalenz auf $\R^n$ stammenden $C_2 = C_2(n)$.

  Wir beweisen nun den ersten Teil der Ungleichung (\ref{eq:compactInequality}) durch einen Widerspruchsbeweis.
  Dazu nehmen wir an, es existiere keine Konstante $C > 0$, sodass die Ungleichung
  $$
  \norm{u}_q \leq C \norm{\nabla u}_{-1,q}
  $$
  für alle $u \in L^q(\Omega)$ mit $\int_{\Omega0} u \d x = 0$ gelte.
  Dann existiert insbesondere für alle $j \in \N$ ein $u_j \in L^q(\Omega)$ mit
  $$
  \norm{u_j}_q > j \norm{\nabla u_j}_{-1,q}
  $$
  und $\int_{\Omega_0} u_j \d x = 0$.
  Wir betrachten nun die normierte Folge $(\tilde u_j)_{n \in \N}$ mit 
  $$
  \tilde u_j := \frac{u_j}{\norm{u_j}_{q}}, \quad n \in \N.
  $$
  So gilt weiterhin aufgrund der Homogenität des Integrals $\int_{\Omega_0} \tilde u_j \d x = 0$. 
  Darüberhinaus gilt $\norm{\tilde u_j}_q = 1$ und die Ungleichung 
  \begin{displaymath}
    \norm{\nabla \tilde u_j}_{-1,q} < \frac{1}{j}, \quad j \in \N. \tag{$\ast$}
  \end{displaymath}

  Nach dem Satz von \banach\hyp{}\alaoglu\ ist die Einheitskugel in reflexiven \banach\hyp{}Räumen schwach kompakt.
  Damit folgt, dass die beschränkte Folge $(\tilde u_j)_{j \in \N}$ eine schwach konvergente Teilfolge mit Grenzwert $u \in L^q(\Omega)$ enthält.
  Zur Vereinfachung bezeichnen wir diese konvergente Teilfolge wieder mit $(\tilde u_j)_{j \in \N}$.
  Nach Definition der schwachen Konvergenz gilt somit
  $$
  \langle u, v \rangle = \lim_{j \to \infty} \langle \tilde u_j, v \rangle
  $$
  für alle $v \in L^{q'}(\Omega)$.
  Insbesondere gilt aufgrund der vorausgesetzten Beschränktheit von $\Omega$ auch $\1 \in L^q(\Omega)$ und damit
  $$
  \langle u, \1 \rangle = \int_{\Omega_0} u \d x = 0.
  $$
  Unter Verwendung der Ungleichung ($\ast$) folgt
  \begin{displaymath}
    \lim_{j \to \infty} \norm{\nabla \tilde u_j}_{-1,q} = 0 \tag{$\ast\ast$}.
  \end{displaymath}
  Für alle $v \in W_0^{1,q'}(\Omega)^n$ und $j \in \N$ gilt zudem 
  \begin{align*}
    |\nabla \tilde u_j, v]|
    &= |\langle \tilde u_j, \div v \rangle| \\
    &\leq \norm{\tilde u_j}_q \norm{\div v}_{1,q} \tag{\hoelder}\\
    &\leq \norm{\tilde u_j}_q \norm{v}_{1,q} \\
    &\leq \norm{\nabla \tilde u_j}_{-1,q} \norm{v}_{1,q}, \tag{\poincare}
  \end{align*}
  woraus schließlich
  \begin{align*}
    |[\nabla u, v]|
    &= |\langle u, \div v \rangle \\
    &= \lim_{j \to \infty}|\langle \tilde u_j, \div v \rangle \\
    &= \lim_{j \to \infty} |[\nabla \tilde u_j, v ]| \\
    &= 0
  \end{align*}
  folgt.
  Im distributionellen Sinne gilt damit $\nabla u = 0$.
  In Abschnitt \ref{subsec:mollification} haben wir gezeigt, das dies gerade impliziert, dass $u$ konstant ist.
  Aus der Bedingung $\int_{\Omega_0} u \d x = 0$ folgt nun $u = 0$. 
  
  Nach \cite[S.45 Lemma 1.1.3]{sohr2001navier} existiert eine Konstante $C > 0$, sodass für alle $j \in \N$ die Ungleichung
  \begin{displaymath}
    1 = \norm{\tilde u_j}_q \leq C (\norm{\nabla \tilde u_j}_{-1,q} + \norm{\tilde u_j}_{-1,q}) \tag{$\ast\ast\ast$}
  \end{displaymath}
  gilt.
  Da die Folge $(\tilde u_j)_{j \in \N}$ beschränkt in $L^q(\Omega)$ ist und zudem die Einbettung nach Lemma $\ref{lem:compactEmbedding0}$ kompakt ist, existiert eine bezüglich der Norm auf $W^{-1,q}(\Omega)$ konvergente Teilfolge von $(\tilde u_j)_{j \in \N}$, welche gegen eine Funktion $\tilde u$ konvergiert. 
  Wir wollen die Teilfolge wieder mit $(\tilde u_j)_{j \in \N}$ bezeichnen.
  Insbesondere konvergiert die Folge $(\tilde u_j)_{j \in \N}$ schwach gegen $\tilde u$.
  Dann gilt aber $\lim_{j \to \infty} \tilde u_j = \tilde u = u = 0$ aufgrund der \hausdorff\hyp{}Eigenschaft der schwachen Topologie.

  Aus den Gleichungen ($\ast\ast$) und ($\ast\ast\ast$) folgt nun der Widerspruch
  \begin{displaymath}
  1 \leq \lim_{j \to \infty} ( \norm{\nabla \tilde u_j }_{-1,q} + \norm{\tilde u_j}_{-1,q} ) = 0.\qedhere
  \end{displaymath}
  

\end{proof}

\section{Darstellung von Funktionalen}

Zunächst beschäftigen wir uns mit dem Gradientenoperator.
Wir wollen zeigen, dass er unter gewissen Zusatzvoraussetzungen ein abgeschlossenes Bild besitzt.
Dies ist Inhalt des folgenden Lemmas.

\begin{lem}
  \label{lem:closedImageGradient}
  Sei $\Omega \subseteq \mathbb{R}^n$ mit $n \geq 2$ ein beschränktes Gebiet und $1 < q < \infty$.
  Dann gilt für die Abbildung
  $$
  \nabla \colon L^{q}(\Omega)^{n} \to (C_0^\infty(\Omega)^n)',
  $$
  dass das Bild
  $$
  \{\nabla v \in L^{q}(\Omega)^{n^2} \colon v \in W_0^{1,q}(\Omega)^n\} \subseteq L^{q}(\Omega)^{n^2}
  $$
  der Einschränkung von $\nabla$ auf $W_0^{1,q}$ eine abgeschlossene Teilmenge des Raumes $L^q(\Omega)^{n^2}$ ist.
\end{lem}

\begin{proof}
  Sei $(v_j)_{j \in \N}$ eine Folge in $W_0^{1,q}(\Omega)^n$, sodass die Folge der Gradienten $(\nabla v_j)_{j \in \N}$ in $L^q(\Omega)^{n^2}$ konvergiert.
  Als konvergente Folge ist diese insbesondere eine \cauchy\hyp{}Folge.
  Da $\Omega$ nach Voraussetzung ein beschränktes Gebiet ist, lässt sich die \poincare\hyp{}Ungleichung anwenden.
  Daraus folgt, dass auch die Folge $(v_j)_{j \in \N}$ eine \cauchy\hyp{}Folge in $L^q(\Omega)^n$ ist.
  Dann gilt jedoch mit der Definition der \sobolev\hyp{}Norm die Ungleichung
  $$
  \norm{v_j - v_k}_{W_{1,q}} \leq c (\norm{v_j - v_k}_q + \norm{\nabla v_j - \nabla v_k}_q)
  $$
  für ein aufgrund der verwendeten Normäquivalenz existierendes $c > 0$ und alle $j,k \in \N$.
  Die Folge $(v_j)_{j \in \N}$ ist also auch eine \cauchy\hyp{}Folge bezüglich \sobolev\hyp{}Norm.
  Der Raum $W_0^{1,q}$ ist nach Definition ein bezüglich \sobolev\hyp{}Norm abgeschlossener Unterraum von $W^{1,q}$ und damit gilt $\lim_{j \to \infty} v_j = v \in W_0^{1,q}$.
  Da zudem
  $$
  \norm{\nabla v_j}_{L_{1,q}} \leq \norm{v}_{W_{1,q}}  
  $$
  gilt, muss also auch $\lim_{j \to \infty} \nabla v_j = \nabla v$ gelten.
  Dies beweist, dass $D$ ein abgeschlossener Unterraum von $L^q(\Omega)^{n^2}$ ist.
\end{proof}

Basierend auf \cite[S.61, Lemma 1.6.1]{sohr2001navier} beweisen wir nun eine Verallgemeinerung des zitierten Lemmas. 
Hierzu erweitern wir die Wirkung des Divergenzoperators zunächst auf Matrizen $F = (F_{jl})_{j,l=1}^n$ durch
\begin{equation}
  \label{eq:divMatrix}
  \div(F) := (\sum_{j = 1}^n D_j F_{jl})_{l = 1,\dots,n},
\end{equation}
der Operator berechnet also spaltenweise die Divergenz.

\begin{lem}
  \label{lem:divRepresentation}
  Sei $\Omega \subseteq \R^n$ mit $n \geq 2$ ein beschränktes Gebiet und $f \in W^{-1,q}(\Omega)^n$ mit $1 < q < \infty$.
  Dann existiert eine Matrix $F \in L^q(\Omega)^{n^2}$, welche die Gleichung
  $$ f = \div F $$
  im distributionellen Sinne und die Ungleichungen
  $$
  \norm{ f }_{W^{-1,q}(\Omega)^n} 
  \leq \norm{ F }_{L^q(\Omega)^{n^2}} 
  \leq C \norm{ f }_{W^{-1,q}(\Omega)^n}
  $$
  mit $C = C(\Omega) > 0$ erfüllt.
\end{lem}

\begin{proof}
  Wir betrachten den Raum
  $$
  D := \{\nabla v \in L^{q'}(\Omega)^{n^2} \colon v \in W_0^{1,q'}(\Omega)^n\} \subseteq L^{q'}(\Omega)^{n^2}
  $$
  der Gradienten $\nabla v = (D_j v_l)_{j,l=1}^n$ von Funktionen $v = (v_1,\dots,v_n) \in W_0^{1,q'}(\Omega)^n$.
  Nach Lemma \ref{lem:closedImageGradient} ist $D$ ein abgeschlossener Unterraum von $L^{q'}(\Omega)^{n^2}$.

  Wir definieren das Funktional
  $$
  \tilde f \colon \nabla v \mapsto [\tilde f, \nabla v]\;, \quad \nabla v \in D
  $$
  durch $[\tilde f, \nabla v] := [f, v]$ für alle $v \in W_0^{1,q'}(\Omega)^n$.
  Dann liefert die \hoelder\hyp{}Ungleichung zusammen mit der \poincare\hyp{}Ungleichung eine Konstante $C = C(\Omega) > 0$, sodass 
  $$
  |[\tilde f, \nabla v]| 
  = |[f, v]| 
  \underset{\hoelder}{\leq} \norm{f}_{-1,q} \norm{v}_{1,q'}
  \underset{\poincare}{\leq} C \norm{f}_{-1,q} \norm{\nabla v }_{q'}
  $$
  für alle $\nabla v \in D$ gilt.
  Somit ist $\tilde f$ ein stetiges Funktional auf $D \subseteq L^{q'}(\Omega)^{n^2}$ mit 
  $$\norm{\tilde f}_{D'} \leq C\norm{f}_{-1,q}.$$
  Der Satz von \hahn\hyp{}\banach\ liefert eine normgleiche Fortsetzung von $D$ nach $L^{q'}(\Omega)^{n^2}$.
  Nach dem Darstellungssatz von \riesz\ über Funktionale auf $L^{q}$  existiert nun eine Matrix $F \in L^{q}(\Omega)^{n^2}$ mit
  $$
  \langle F, \nabla v \rangle
  = \sum_{j,l=1}^n \int_\Omega F_{jl}(D_j v_l) \d x
  = \int_\Omega F \cdot \nabla v \d x
  = [\tilde f, \nabla v] 
  = [f, v]
  $$
  für alle $v = (v_1,\dots,v_n) \in W_0^{1,q'}(\Omega)^n$.
  Beweise des Darstellungssatzes finden sich in \cite[S.47, Theorem 2.44]{adams2003sobolev} und \cite[S.60, Satz II.2.4]{werner2011fa}.
  Zudem gilt
  $$ \norm{F}_{L^q(\Omega)^{n^2}} = \norm{\tilde f}_{(L^{q'}(\Omega)^{n^2})'} \leq C \norm{f}_{-1,q},$$
  da die Identifikation von Funktionalen auf $L^{q'}(\Omega)^{n^2}$ mit Funktionen aus $L^q(\Omega)^{n^2}$ isometrisch ist.
  Des Weiteren gilt für alle $v \in W_0^{1,q'}(\Omega)^n$ mit der \hoelder\hyp{}Ungleichung
  \begin{align*}
    |[f,v]| 
    &= |\langle F, \nabla v\rangle|  \\
    &\leq \norm{F}_q \norm{\nabla v}_{q'}  \tag{\hoelder\hyp{}Ungleichung}\\
    &\leq \norm{F}_q (\norm{v}_{q'}^{q'} + \norm{\nabla v}_{q'}^{q'})^{\frac{1}{q'}} \tag{Übergang zu \sobolev\hyp{}Norm}\\
    &= \norm{F}_q \norm{v}_{W_0^{1,q'}(\Omega)^n}.
  \end{align*}
  Daraus folgt
  $$
  \norm{f}_{W^{-1,2}(\Omega)} \leq \norm{F}_2.
  $$
  Ist nun $v \in C_0^\infty(\Omega)^n$, so gilt
  \begin{align*}
    [f,v]
    &= \langle F, \nabla v\rangle \\
    &= \sum_{j,l=1}^n \langle F_{jl}, D_j v_l \rangle \\
    &= -\sum_{j,l=1}^n \langle D_j F_{jl}, v_l \rangle \\
    &= -[\div F, v], \\
  \end{align*}
  die Abbildungen $[f,\cdot]$ und $[-\div F, \cdot]$ stimmen also als Distributionen überein, die zu zeigende Aussage folgt durch den Übergang von $F$ zu $-F$.
\end{proof}

\section{Das Gradientenkriterium}

In diesem Abschnitt stellen wir ein für die \helmholtz\hyp{}Zerlegung fundamentales Kriterium vor, nach welchem es möglich ist, Funktionen $f \in W^{1,q}(\Omega)^n$ als Gradienten $f = \nabla p$ mit $p \in L^q(\Omega)$ darzustellen.

Wir stellen eine Modifikation eines aus der Funktionentheorie bekannten Lemmas voran.

\begin{lem}
  \label{lem:existencePotential}
  Sei $\Omega \subseteq \R^n$ ein beschränktes Gebiet und $g = (g_1, \dots, g_n) \in C^\infty(\overline\Omega)$.
  Falls für jede geschlossene $PC^1$-Kurve $\gamma \colon [0,1] \to \overline\Omega$ das Kurvenintegral verschwindet,
$$
  \int_0^1 g(\gamma(t)) \cdot \gamma'(t) \d t = 0,  
$$
  dann existiert ein $U\in C^\infty(\overline\Omega)$ mit $g = \nabla U$.
\end{lem}

\begin{proof}
  Mit dem Wissen aus der Funktionentheorie folgt zunächst die Existenz einer Funktion $U \in C^\infty(\Omega)$, welche auf $\Omega$ die Gleichung $g = \nabla U$ erfüllt.
  Nach Voraussetzung ist $g \in C^\infty(\overline\Omega)$.
  Daraus folgt mit Lemma \ref{lem:CInftyClosedOmega}, dass $g|_\Omega = \nabla U$ und sämtliche partiellen Ableitungen der Komponenten von $\nabla U$ gleichmäßgig stetig auf $\Omega$ sind und sich daher stetig auf $\overline\Omega$ fortsetzen lassen.
  Also folgt wieder nach Lemma \ref{lem:CInftyClosedOmega}, dass auch $U \in C^\infty(\overline\Omega)$ gilt.
  Insbesondere ist aufgrund der Kompaktheit von $\overline\Omega$ die Funktion $U$ \lipschitz\hyp{}stetig, wie durch eine Anwendung des Schrankensatzes folgt.
  Als \lipschitz\hyp{}stetige Funktion besitzt $U$ somit auch eine stetige Fortsetzung auf $\overline\Omega$.
  Daraus folgt $U \in C^\infty(\overline\Omega)$ mit Lemma \ref{lem:CInftyClosedOmega}.
\end{proof}

\begin{lem}
  \label{lem:gradientCriterion}
  Sei $\Omega \subseteq \R^n$ mit $n \geq 2$ ein Gebiet und $\Omega_0 \subseteq \Omega$ ein beschränktes Teilgebiet mit $\emptyset \neq \Omega_0 \subseteq \overline\Omega_0 \subseteq \Omega$ und $1 < q < \infty$.
  Angenommen, für $f \in W_{\loc}^{-1,q}(\Omega)^n$ gelte
  \begin{equation}
    \label{eq:vKernel}
    [f,v] = 0, \quad \text{für alle} \quad v \in C_{0,\sigma}^\infty(\Omega).
  \end{equation}
  Dann existiert ein eindeutig bestimmtes $p \in L^q_{\loc}(\Omega)$, welches die Gleichung $\nabla p = f$ im distributionellen Sinne erfüllt und für das zusätzlich
  \begin{equation}
    \int_{\Omega_0} p \d x = 0
    \label{eq:0integral}
  \end{equation}
  gilt.
\end{lem}

\begin{proof}
  Wir zeigen zunächst, dass für ein beliebiges beschränktes \lipschitz\hyp{}Gebiet $\Omega_1 \subseteq \Omega$ mit $\overline\Omega_0 \subseteq \Omega_1 \subseteq \overline\Omega_1 \subseteq \Omega$ ein eindeutig bestimmtes $p \in L^q(\Omega_1)$ existiert, welches die Behauptung des Lemmas erfüllt.

  Ähnlich zum ersten Beweisschritt von Lemma \ref{lem:lipschitzExhaust} finden wir ein weiteres beschränktes \lipschitz\hyp{}Gebiet $\Omega_2$ mit $\overline\Omega_1 \subseteq \Omega_2 \subseteq \overline\Omega_2 \subseteq \Omega$.
  Dazu wählen wir ein $x_0 \in \Omega_1$ und finden aufgrund der vorausgesetzten Beschränktheit von $\Omega_1$ ein $r > 0$, sodass $\Omega_1 \subseteq B_r(x_0)$ gilt.
  Wir wählen sodann die $x_0$ enthaltende Zusammenhangskomponente $\widetilde\Omega_2$ von $B_r(x_0) \cap \Omega$ aus und konstruieren wie schon im Beweis von Lemma \ref{lem:lipschitzExhaust} das beschränkte \lipschitz\hyp{}Gebiet $\Omega_2 = \widehat\Omega_2$.

  Der Voraussetzung $f \in W_{\loc}^{-1,q}(\Omega)^n$ entnehmen wir, dass $f \in W^{-1,q}(\Omega_2)^n$ gilt. Zudem existiert aufgrund der Beschränktheit von $\Omega_2$ nach Lemma \ref{lem:divRepresentation} eine Darstellung von $f$ als Distribution
  $$
  f = \div F \quad\text{mit}\quad F = (F_{jl})_{j,l=1}^n \in L^q(\Omega_2)^{n^2}.
  $$

  Im Folgenden bezeichne $F^\varepsilon := \mathcal{F}_\varepsilon \ast F = (\mathcal{F}_\varepsilon \ast F_{jl})_{j,l=1}^n$ mit $0 < \varepsilon < \dist(\Omega_1, \partial\Omega_2)$ die in Abschnitt \ref{subsec:mollification} definierte Faltung von $F$ mit einem Glättungskern, für den, wie bereits gezeigt, $F^\varepsilon \in C^\infty(\overline\Omega_1)^{n^2}$ gilt.
  Wir wollen beweisen, dass eine Darstellung der Form
  \begin{equation}
    \label{eq:divNabla}
    \div F^\varepsilon = \nabla U_\varepsilon
  \end{equation}
  mit einem $U_\varepsilon \in C^\infty(\overline\Omega_1)$ existiert, wobei die Divergenz, wie in Gleichung (\ref{eq:divMatrix}) definiert, spaltenweise wirkt.
  Nach Lemma \ref{lem:existencePotential} ist dies genau dann der Fall, wenn
  $$
  \oint_\gamma \div F^\varepsilon \cdot \d s = \int_0^1 \div F^\varepsilon(\gamma(\tau))\cdot \gamma'(\tau) \d\tau = 0
  $$
  für alle $PC^1$-Kurven $\gamma \colon [0,1] \to \overline\Omega_1$ gilt.

  Dazu definieren wir für alle $x\in\Omega_2$ und $PC^1$\hyp{}Kurven $\gamma \colon [0,1] \to \overline\Omega_1$ den Wert des Integrals
  $$
  V_{\gamma,\varepsilon}(x) := \int_0^1 \mathcal{F}_\varepsilon(x - \gamma(\tau))\gamma'(\tau)\d\tau
  $$
  und erhalten $V_{\gamma,\varepsilon} \in C_0^\infty(\Omega_2)^n$ durch wiederholte Anwendung der \leibniz\hyp{}Regel für Parameterintegrale.
  Darüber hinaus gilt für eine geschlossene Kurve $\gamma$ in $\overline\Omega_1$ folgende Rechnung
  \begingroup
  \addtolength{\jot}{0.5em}
  \begin{align*}
    \div V_{\gamma,\varepsilon}(x) 
    &= \int_0^1 \sum_{j=1}^n (D_j \mathcal{F}_\varepsilon)(x - \gamma(\tau))\gamma_j'(\tau) \d\tau \tag{\leibniz}\\
    &= -\int_0^1 \frac{\d\ }{\d\tau}\mathcal{F}_\varepsilon(x - \gamma(\tau)) \d\tau \tag{Kettenregel}\\
    &= - \mathcal{F}_\varepsilon(x - \gamma(1)) + \mathcal{F}_\varepsilon(x - \gamma(0)) \tag{Hauptsatz}\\
    &= 0 \tag{geschlossene Kurve}
  \end{align*}
  \endgroup

  Hiermit folgt $V_{\gamma,\varepsilon} \in C_{0,\sigma}^\infty(\Omega_2)^n$.
  Unter Verwendung der Voraussetzung aus Gleichung (\ref{eq:vKernel}) und dem Satz von \fubini\ folgt
  \begingroup
  \addtolength{\jot}{1em}
  \begin{align*}
    0
    %
    &= [f, V_{\gamma, \varepsilon}] %\qquad &&\text{Voraussetzung (\ref{eq:vKernel})}\\
    %
    = [\div F, V_{\gamma, \varepsilon}] %\\
    %
    = \int_{\Omega_2} \div F \cdot V_{\gamma, \varepsilon} \d x \tag{duale Paarung aus (\ref{eq:pairingVector})} \\
    %
    &= \int_{\Omega_2} \sum_{j,l=1}^n D_j F_{jl}(x) \cdot \left( \int_0^1 \mathcal{F}_\varepsilon(x - \gamma(\tau))\gamma_l'(\tau) \d\tau \right) \d x \\
    %
    %&= \sum_{j,l=1}^n \int_{\Omega_2} D_j F_{jl}(x) \left( \int_0^1 \mathcal{F}_\varepsilon(x - w(\tau))w_l'(\tau) \d\tau \right) \d x \\
    %
    %&= \sum_{j,l=1}^n \int_{\Omega_2} \int_0^1 D_j F_{jl}(x) \mathcal{F}_\varepsilon(x - w(\tau))w_l'(\tau) \d\tau \d x \\
    %
    %&= \sum_{j,l=1}^n \int_0^1 \int_{\Omega_2}  D_j F_{jl}(x) \mathcal{F}_\varepsilon(x - w(\tau))w_l'(\tau) \d\tau \d x 
    %
    &= \sum_{j,l=1}^n \int_0^1 \left(\int_{\Omega_2}  D_j F_{jl}(x) \mathcal{F}_\varepsilon(x - \gamma(\tau))\d x \right) \cdot \gamma_l'(\tau) \d\tau \tag{Satz von \fubini}\\
    %
    &= \sum_{j,l=1}^n \int_0^1 \left(\int_{\Omega_2}  D_j F_{jl}(x) \mathcal{F}_\varepsilon(\gamma(\tau) - x)\d x \right) \cdot \gamma_l'(\tau) \d\tau \tag{$\mathcal{F}_\varepsilon(x) = \mathcal{F}_\varepsilon(-x)$}\\
    %
    &= \sum_{j,l=1}^n \int_0^1 \left(D_j F_{jl} \ast \mathcal{F}_\varepsilon\right)(\gamma(\tau)) \cdot \gamma_l'(\tau) \d\tau \\
    %
    &= \sum_{j,l=1}^n \int_0^1 D_j\left(F_{jl} \ast \mathcal{F}_\varepsilon\right)(\gamma(\tau)) \cdot \gamma_l'(\tau) \d\tau \tag{(\ref{eq:convolutionDiff}) mit $F_{jl} \in L^2(\Omega) \subseteq C_0^\infty(\Omega)'$}\\
    %
    &= \sum_{j,l=1}^n \int_0^1 (D_j F_{jl}^\varepsilon)(\gamma(\tau)) \cdot \gamma_l'(\tau) \d\tau \\
    %
    &= \int_0^1 (\div F^\varepsilon)(\gamma(\tau)) \cdot \gamma'(\tau) \d\tau \\
    %
    &= \oint_\gamma \div F^\varepsilon \cdot \d s.
  \end{align*}
  \endgroup
  %\footnote{Ich habe cdot durchgehend verwendet um etwas Übersichtlichkeit zu schaffen, auch wenn cdot in meiner Arbeit nur für das Standardskalarprodukt verwendet werden sollte}

  Es existiert also ein $U_\varepsilon \in C^\infty(\overline\Omega_1)$ mit $\div F^\varepsilon = \nabla U_\varepsilon$.
  Die Funktion $U_\varepsilon$ ist bis auf eine Konstante $c$ eindeutig bestimmt, denn es gilt $\nabla U_\varepsilon = \nabla( U_\varepsilon - c)$.
  Wir können durch die Wahl $c = \int_{\Omega_0} U_\varepsilon \d x$ erreichen, dass $\int_{\Omega_0} U_\varepsilon - \gamma \d x = 0$  gilt.
  Im Folgenden bezeichnen wir die so gewählte Funktion wieder mit $U_\varepsilon$.
  Da $\overline\Omega_1$ kompakt ist, gilt insbesondere $U_\varepsilon \in L^q(\Omega_1)$.
  Mit Lemma \ref{lem:compactEmbedding} folgt
  \begingroup
  \addtolength{\jot}{1em}
  \begin{align*}
    \norm{U_\varepsilon}_{L^q(\Omega_1)}
    &\leq C \norm{ \nabla U_\varepsilon }_{W^{-1,q}(\Omega_1)} \\
    &= C \sup_{0 \neq v \in C_0^\infty(\Omega_1)^n} \left(\frac{|[\nabla U_\varepsilon, v]|}{\norm{\nabla v}_{q'}} \right) \\
    &\overset{(\ast)}{=} C \sup_{0 \neq v \in C_0^\infty(\Omega_1)^n} \left( \frac{|\langle F^\varepsilon, \nabla v \rangle|}{\norm{v}_{q'}}\right) \\
    &\leq C \norm{F^\varepsilon}_{L^q(\Omega_1)} \tag{$\star$}
  \end{align*}
  \endgroup
  für ein $C = C(q,\Omega_0,\Omega_1)$, welches nicht von $\varepsilon$ abhängt. 
  Die Gleichheit ($\ast$) ergibt sich dabei wie folgt:
  \begin{align*}
    \langle F^\varepsilon, \nabla v \rangle
    &= \int_{\Omega_1} F^\varepsilon \cdot \nabla v \d x \\
    &= \int_{\Omega_1} \sum_{j,l=1}^n F_{jl}^\varepsilon (D_j v_l) \d x \\
    &= \int_{\Omega_1} \sum_{j,l=1}^n -(D_jF_{jl}^\varepsilon) v_l \d x \tag{Kompakter Träger}\\
    &= \int_{\Omega_1} -\div F^\varepsilon \cdot v \d x \\
    &= \int_{\Omega_1} -\nabla U_\varepsilon \cdot v \d x \tag{Gleichung (\ref{eq:divNabla}) gilt}\\
    &= -[\nabla U_\varepsilon, v].
  \end{align*}

  Eine wesentliche Eigenschaft der Glättungskerne ist, dass $\lim_{\varepsilon \to 0} \norm{F - F^\varepsilon}_{L^q(\Omega_1)} = 0$ nach Lemma \ref{lem:mollification} gilt.
  Das Netz $(F^\varepsilon)_{\varepsilon \in \R^+}$ ist, da es konvergiert, auch ein \cauchy-Netz.
  Mit Gleichung ($\star$) gilt dann für alle $0 < \eta < \varepsilon$ die Ungleichung
  $$
    \norm{U_\varepsilon - U_\eta}_{L^q(\Omega_1)}
    \leq C \norm{F^\varepsilon - F^\eta}_{L^q(\Omega_1)}.
  $$
  Also ist auch $(U_\varepsilon)_{\varepsilon \in \R^+}$ ein \cauchy-Netz, welches aufgrund der Vollständigkeit des Raumes $L^q(\Omega_1)$ einen eindeutig bestimmten Grenzwert $U \in L^q(\Omega_1)$ besitzt.
  Da zudem $\Omega_0$ beschränkt ist und damit endliches Maß besitzt, folgt aus der \hoelder\hyp{}Ungleichung
  $$
    | \int_{\Omega_0} U - U_\varepsilon \d x |
    \leq  \norm{U - U_\varepsilon}_{L^1(\Omega_0)} 
    \underset{\hoelder}{\leq} \norm{U - U_\varepsilon}_{L^q(\Omega_0)}\cdot \norm{1}_{L^{q'}(\Omega_0}
    \to 0
  $$
  für $\varepsilon \to 0$, also auch
  $$
    \int_{\Omega_0} U \d x = \lim_{\varepsilon \to 0} \int_{\Omega_0} U_\varepsilon \d x =  0,
  $$
  aufgrund der Wahl von $U^\varepsilon$.

  Nun zeigen wir noch, dass auf $\Omega_1$ die Gleichheit $\div F = \nabla U$ im distributionellen Sinne gilt. 
  Wir halten dazu zunächst fest, dass die Konvergenz auf $L^q(\Omega_1)^{n^2}$ die komponentenweise Konvergenz auf $L^q(\Omega_1)$ impliziert.
  Es gilt somit $\norm{F_{jl} - F_{jl}^\varepsilon}_{L^q(\Omega_1)} \to 0$ für $\varepsilon \to 0$ für alle $j,l \in \{ 1,\dots,n\}$.
  Sei nun $v \in C_0^\infty(\Omega_1)^n$.
  Dann gilt
  \begin{align*}
    [\nabla U, v]
    &= \int_{\Omega_1} \nabla U \cdot v \d x \\
    &= \int_{\Omega_1} \sum_{j = 1}^n (D_j U) v_j \d x \\
    &= -\int_{\Omega_1} \sum_{j = 1}^n U (D_j v_j) \d x \\ 
    &= -\int_{\Omega_1} \sum_{j = 1}^n \lim_{\varepsilon \to 0} U_\varepsilon (D_j v_j) \d x \\
    &\overset{(\clubsuit)}{=} \lim_{\varepsilon \to 0} \int_{\Omega_1} \nabla U_\varepsilon \cdot v \d x \\
    &= \lim_{\varepsilon \to 0} \int_{\Omega_1} \div F^\varepsilon \cdot v \d x \\
    &= -\lim_{\varepsilon \to 0} \int_{\Omega_1} \sum_{j,l=1}^n F_{jl}^\varepsilon (D_j v_l) \d x \\
    &\overset{(\clubsuit)}{=} \int_{\Omega_1} \sum_{j,l=1}^n (D_j F_{jl}) v_l \d x \\
    &= \langle \div F, v \rangle,
  \end{align*}
  wobei bei ($\clubsuit$) verwendet wurde, dass $\Omega_1$ beschränkt ist.

  Zuletzt zeigen wir, wie sich die gesuchte Funktion $p$ konstruieren lässt.
  Wie der Beweis zeigt, lässt sich für jedes beschränkte \lipschitz\hyp{}Gebiet $\Omega_1$ ein $U \in L^q(\Omega_1)$ finden, welches zudem aufgrund der Forderung 
  $$
  \int_{\Omega_0} U \d x = 0
  $$
  eindeutig bestimmt ist.
  Nach Lemma \ref{lem:lipschitzExhaust} lässt sich jedes Gebiet $\Omega$ durch eine Folge $(\Omega_j)_{j \in \N}$ beschränkter \lipschitz\hyp{}Gebiete ausschöpfen.
  Zudem gilt nach Bemerkung \ref{bem:boundedSubset}, dass jedes beschränkte Teilgebiet $\Omega' \subseteq \Omega$ in einem beschränkten \lipschitz\hyp{}Gebiet $\Omega_j, j \in \N$ enthalten ist.
  Wir können direkt annehmen, dass dies bereits für $j = 1$ erfüllt ist.
  Führen wir nun den vorangehenden Beweis für $\Omega_j$, so erhalten wir eine eindeutig bestimmte Funktion $p \in L^q(\Omega_j)$, die die Gleichung $f = \nabla p$ auf $\Omega_j$ im distributionellen Sinne erfüllt und für die des Weiteren das Integral $\int_{\Omega_0} p \d x$ verschwindet.
  Da nach Lemma \ref{lem:lipschitzExhaust} die Gleichheit $\bigcup_{j \in \N} \Omega_j = \Omega$ gilt, lässt sich $p$ auch eindeutig bis auf Nullmengen auf $\Omega$ definieren.
  Ist zudem $B$ ein Ball mit $\overline B \subseteq \Omega$, so existiert ein $j \in \N$ mit $\overline B \subseteq \Omega_j$.
  Da $p \in L^q(\Omega_j)$ gilt, folgt sofort $p \in L^q(B)$.
  Die Definition zum Begriff der lokalen Integrierbarkeit in Abschnitt \ref{subsec:distributionsSobolev} liefert hiermit $p \in L^q_{\loc}(\Omega)$.

  Dies beweist die Behauptung.
\end{proof}
