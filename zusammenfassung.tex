\chapter{Zusammenfassung und Ausblick}

In dieser Arbeit wurde die Existenz und Eindeutigkeit der \helmholtz\hyp{}Zerlegung des Funktionenraumes $L^2(\Omega)$ für ein Gebiet $\Omega \subseteq \R^n$ bewiesen.
Die Hauptarbeit bestand darin, unter geeigneten Zusatzannahmen die Lösbarkeit der im distributionellen Sinne zu verstehenden Gradientengleichung $\nabla p = f$ zu sichern.
Dazu war es zunächst nötig zu beweisen, wie sich \lipschitz\hyp{}Gebiete geeignet von innen approximieren lassen.
Zudem implizierten die Hilfsresultate auch einen Darstellungssatz für Funktionale auf \sobolev\hyp{}Räumen.
Der Beweis der \helmholtz\hyp{}Zerlegung gestaltete sich aufgrund dieser Vorarbeit übersichtlich.
Zusätzlich wurde eine Charakterisierung eines an der Zerlegung beteiligten Funktionenräume im Falle $\Omega = \R^n$ bewiesen.
%In \cite[S.81ff., II.2.5]{sohr2001navier} finden sich weitere Charakterisierungen für beschränkte \lipschitz\hyp{}Gebiete sowie eine Dichtheitseigenschaft.

Anwendung findet die \helmholtz\hyp{}Zerlegung in der Konstruktion schwacher Lösungen der \stokes\hyp{}Gleichung \cite[S.129f.]{sohr2001navier}. 
Es lässt sich zeigen, dass der \stokes\hyp{}Operator bis auf einen Vorfaktor mit der Verkettung $(P \circ \Delta)$ zweier Operatoren übereinstimmt, wobei $P$ die zur \helmholtz\hyp{}Zerlegung gehörige orthogonale Projektion und $\Delta$ den \laplace\hyp{}Operator darstellt.

Kann man die Existenz der \helmholtz\hyp{}Zerlegung auf weitere Funktionenräume verallgemeinern?
Für allgemeine $L^q$\hyp{}Räume ist dies nicht möglich, wie Gegenbeispiele in \cite{maslennikova1986elliptic} oder \cite{bogovski1986decomposition} demonstrieren.
In \cite{simader1992new}, \cite[S.146, Lemma III.1.2]{galdi2011navier} wird gezeigt, dass die Existenz der \helmholtz\hyp{}Zerlegung äquivalent zur Lösbarkeit eines geeigneten \neumann\hyp{}Problems ist. 
Über diese Charakterisierung der Zerlegung wird in \cite[S.152, Theorem III.1.2]{galdi2011navier} für $C^2$-Gebiete oder Halbräume des $\R^n$ mit $n \geq 2$ die Existenz einer \helmholtz\hyp{}Zerlegung bewiesen.
Andererseits zeigt \cite{farwig05anLq} für beliebige Gebiete, dass sich die Resultate zur \helmholtz\hyp{}Zerlegung in \hilbert\hyp{}Räumen auf Schnitte beziehungsweise direkte Summen bestehend aus einem $L^2$- und einem $L^q$\hyp{}Raum übertragen lassen.

