\chapter{Zusammenfassung und Ausblick}

In dieser Arbeit wurde die Existenz und Eindeutigketi \helmholtz\hyp{}Zerlegung des Funktionenraumes $L^2(\Omega)$ für ein Gebiet $\Omega \subseteq \R^n$ bewiesen.
Nach einer Wiederholung und Vertiefung integrationstheoretischer und funktionalanalytischer Grundlagen bestand die Hauptarbeit darin, unter geeigneten Zusatzannahmen, die Lösbarkeit der im distributionellen Sinne zu verstehenden Gradientengleichung $\nabla p = f$ zu sichern.
Dazu war es zunächst nötig zu Beweisen wie sich \lipschitz\hyp{}Gebiete geeignet von Innen approximieren lassen.
Zudem beinhalteten die Hilfsresultate auch einen Darstellungssatz für Funktionale auf \sobolev\hyp{}Räumen.
Der Beweis der \helmholtz\hyp{}Zerlegung gestaltete sich aufgrund dieser Vorarbeit nunmehr übersichtlich.
Zusätzlich wurde eine Charakterisierung eines an der Zerlegung beteiligten Funktionenräume im Falle $\Omega = \R^n$ bewiesen.
In \cite[S.81ff., II.2.5]{sohr2001navier} finden sich weitere Charakterisierungen für beschränkte \lipschitz\hyp{}Gebiete sowie eine Dichtheitseigenschaft.

Anwendung findet die \helmholtz\hyp{}Zerlegung beispielsweise in der Konstruktion schwacher Lösungen der \stokes\hyp{}Gleichung \cite[S.129f.]{sohr2001navier}. 
Es lässt sich zum Beispiel zeigen, dass der \stokes\hyp{}Operator bis auf einen Vorfaktor mit der Verkettung der Operatoren $(P \circ \Delta)$ übereinstimmt, wobei $P$ die zur \helmholtz\hyp{}Zerlegung gehörige orthogonale Projektion und $\Delta$ den \laplace\hyp{}Operator darstellt.

Es stellt sich die Frage, inwieweit, sich die Existenz der \helmholtz\hyp{}Zerlegung und damit verbundener Aussagen auf weitere Funktionenräume verallgemeinern lassen.
Für allgemeine $L^q$\hyp{}Räume ist dies nicht möglich, wie Gegenbeispiele in \cite{maslennikova1986elliptic} oder \cite{bogovski1986decomposition} zeigen.
Unter zusätzlichen Annahmen zeigen jedoch \cite{farwig05thehelmholtz,farwig05anLq}, dass sich theoretische Resultate aus der Lösungstheorie in \hilbert\hyp{}Räume auf Schnitte bzw. direkte Summen bestehend aus einem $L^2$- und einem $L^q$\hyp{}Raum.

    
