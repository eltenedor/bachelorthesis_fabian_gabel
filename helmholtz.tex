\chapter{Helmholtz-Zerlegung in $L^2$}

Nach der in den vorangehenden Kapiteln geleisteten Vorarbeit sind wir nun in der Lage den zentralen Satz dieser Arbeit zu behandeln.
Wir interessieren uns dafür, unter welchen Bedingungen sich die Vektorräume $L^q(\Omega)^n$ in für die Strömungsmechanik interessante Teilräume zerlegen lassen.
Die Darstellung in dieser Arbeit begrenzt sich auf die Zerlegung für $q = 2$.
Hier ermöglicht der für die Theorie der \hilbert\hyp{}Räume grundlegende Begriff der Orthogonalität eine orthogonale Zerlegung des Lösungsraumes $L^2(\Omega)^n$.

Wir bezeichnen mit 
$$
G(\Omega) := \{f \in L^2(\Omega)^n \mid f = \nabla p \text{ für ein } p \in L^2_{\loc}(\Omega)\}
$$
den Raum aller Funktionen $f \in L^2(\Omega)^n$, die im distributionellen Sinne ein skalares Potential besitzen.

\begin{thm}
  \label{lem:helmholtz}
  Sei $\Omega \subseteq \R^n$ mit $n \geq 2$ ein beliebiges Gebiet.
  Dann gilt
  $$
  L^2(\Omega)^n = G(\Omega) \oplus L^2_\sigma(\Omega).
  $$
\end{thm}

\begin{proof}
  Wir zeigen, dass 
  \begin{equation}
    \label{eq:orthComplement}
    G(\Omega) = L_\sigma^2(\Omega)^\perp
  \end{equation}
  gilt.
  Die zu zeigende Aussage folgt dann aus dem Fundament der Theorie der \hilbert\hyp{}Räume: dem Satz von der Orthogonalprojektion.

  Sei dazu $f \in L_\sigma^2(\Omega)^\perp$.
  Wir fassen $f$ als Distribution auf, also als Element von $(C_0^\infty(\Omega)^n)'$.
  Für jedes beschränkte Teilgebiet $\Omega_0 \subseteq \Omega$ mit $\overline\Omega_0 \subseteq \Omega$ folgt für alle $v \in C_0^\infty(\Omega_0)^n$ mit der \cauchy\hyp\schwartz-\ (C.S.) und \poincare\hyp{}Ungleichung \ref{lem:poincare}
  $$
  |[f, v]| 
  = |\langle f, v \rangle| 
  \underset{\textsc{C.S.}}{\leq}  \norm{f}_{L^2(\Omega_0)^n} \norm{v}_{L^2(\Omega_0)^n}
  \underset{\poincare}{\leq} C \norm{f}_{L^2(\Omega_0)^n} \norm{\nabla v}_{L^2(\Omega_0)^{n^2}},
  $$
  mit einer Konstante $C = C(\Omega_0) > 0$.
  Es gilt also 
  \begin{align*}
    |[f, v]| 
    &\leq \norm{f}_{L^2(\Omega_0)^n} \cdot ( C ) \cdot (\norm{v}_{L^2(\Omega_0)^n} + \norm{\nabla v}_{L^2(\Omega_0)^{n^2}}) \\
    &\leq \norm{f}_{L^2(\Omega_0)^n} \cdot C'(n) \cdot ( C ) \cdot \norm{v}_{1,2}
  \end{align*}
  mit einem aus der Normäquivalenz der Normen auf $\R^n$ folgenden $C'(n)$.
  Unter Berücksichtigung von Gleichung (\ref{eq:operatornorm}) folgt daraus $f \in W_{\loc}^{-1,2}(\Omega)^n$.
  Nach Voraussetzung gilt zudem 
  $$
  [f,v] = \langle f, v \rangle = 0
  $$
  für alle $v \in C_{0, \sigma}^\infty(\Omega)$.
  Die Anwendung des Gradientenkriteriums aus Lemma \ref{lem:gradientCriterion} liefert nun die Existenz einer Funktion $p \in L_{\loc}^2(\Omega)$, welche die Gleichung
  $$
  f = \nabla p
  $$
  im distributionellen Sinne löst.
  Daraus folgt $f \in G(\Omega)$.

  Sei umgekehrt $f \in G(\Omega)$ mit $f = \nabla p$ für ein $p \in L^2_{\loc}(\Omega)$.
  Dann gilt
  \begin{align*}
  \langle f, v \rangle
  &= [ \nabla p, v ] \\
  &= \sum_{j = 1}^n [ D_j p , v_j ] \\
  &= - \sum_{j = 1}^n \langle p, D_j v \rangle \\
  &= - \langle p, \div v \rangle  \\
  &= - \langle p, 0 \rangle  \\
  &= 0
  \end{align*}
  für alle $v \in C_{0, \sigma}^\infty(\Omega)$ und wegen der Beziehung $\nabla p \in L^2(\Omega)$ gilt dies aufgrund der Stetigkeit des Skalarproduktes auch für alle $v \in \overline{C_{0, \sigma}^\infty(\Omega)}^{\norm{\cdot}_2} = L_\sigma^2(\Omega)$.
  Damit ist $f \in L_{0,\sigma}^2(\Omega)^\perp$.
\end{proof}

Aus Gleichung (\ref{eq:orthComplement}) folgt insbesondere, dass der Raum $G(\Omega)$ bezüglich der Norm $\norm{\cdot}_2$ abgeschlossen ist.
Als wichtiges Resultat der Theorie der \hilbert\hyp{}Räume, existiert zu dem abgeschlossenen Teilraum $L_\sigma^2(\Omega)$ ein eindeutig bestimmter Projektionsoperator, die sogenannte \helmholtz\hyp{}Projektion
$$
P \colon L^2(\Omega)^n \to L_\sigma^2(\Omega),\quad Pf \mapsto f_0,
$$
für alle $f = f_0 + \nabla p \in L^2(\Omega)$ mit $f_0 \in L_\sigma^2(\Omega)$ und $\nabla p \in G(\Omega)$ \cite[S.226f., Theorem V.3.4]{werner2011fa}.

Im Falle $\Omega = \R^n$ lassen sich die Räume $G(\Omega)$ und $L_\sigma^2(\Omega)$ noch auf eine andere Art charakterisieren.
Dies ist Inhalt des folgenden Satzes.
Zuvor erwähnen wir noch ein dazu benötigtes Hilfsresultat.
Einen Beweis dazu findet man in \cite[S.44, Lemma 1.1.2]{sohr2001navier}.

\begin{lem}
  Sei $\Omega \subseteq \R^n$ ein beschränktes Lipschitz gebiet mit $n \geq 2$, $\Omega_0 \subseteq \Omega$ ein Teilgebiet und $1 < q < \infty$.
  Dann gilt die Ungleichung
  $$
  \norm{u}_{L^q(\Omega)} \leq C \left( \norm{\nabla u}_{L^q(\Omega)^n} + \left| \int_{\Omega_0} u \d x \right|\, \right)
  $$
  für alle $u \in W^{1,q}(\Omega)$, wobei $C = C(q,\Omega, \Omega_0) > 0$ eine Konstante sei.
  \label{lem:poincare2}
\end{lem}

\begin{thm}
  Sei $n \in \N$ mit $n \geq 2$.
  Dann gilt
  \begin{equation}
    \label{eq:LCharacterization}
    L_\sigma^2(\R^n) = \{ f \in L^2(\R^n)^n \mid \div f = 0 \}
  \end{equation}
  und $G(\R^n)$ ist der Abschluss des Raumes
  $$
  \nabla C_0^\infty(\R^n) := \{ \nabla p \mid p \in C_0^\infty(\R^n) \}
  $$
  bezüglich der Norm $\norm{\cdot}_{L^2(\R^n)^n}$.
  Es gilt also
  \begin{equation}
    \label{eq:GCharacterization}
    G(\R^n) = \overline{\nabla C_0^\infty(\R^n)}^{\norm{\cdot}_2}.
  \end{equation}
\end{thm}

\begin{proof}
  Im folgenden Beweis wollen wir eine Funktion $\varphi \in C_0^\infty(\R^n)$ mit den Eigenschaften
  \begin{equation}
    \label{eq:propPhi}
  0 \leq \varphi \leq 1
  \quad,\quad
  \varphi(x) = 1 \text{, falls } |x| \leq 1
  \quad,\quad
  \varphi(x) = 0 \text{, falls } |x| \geq 2,
\end{equation}
  betrachten.
  Eine solche Funktion erhält man zum Beispiel aus der Faltung der charakteristischen Funktion $\chi_{B_{3/2}}(0)$ mit dem Standardglättungskern $\mathcal{F}_{1/2}$.
  Zudem definieren wir die Funktionen
  $$
  \varphi_j \in C_0^\infty(\R^n)
  \quad,\quad
  \varphi_j(x) := \varphi\left(\frac{x}{j}\right)
  \quad,\quad
  x \in \R^n, j \in \N.
  $$
  Nach Konstruktion folgt 
  \begin{equation}
    \label{eq:limPhi}
    \lim_{j \to \infty} = \varphi_j(x) = 1
  \end{equation}
  für alle $x \in \R^n$.
  Setzen wir
  $$
  B_j := \{ x \in \R^n \mid |x| < j \} 
  \quad,\quad
  G_j := B_{2j}\setminus \overline B_j,
  $$
  so erhalten wir $\supp \nabla \varphi_j \subseteq \overline G_j$, da nach Definition $\varphi_j$ auf $B_j$ konstant ist sowie $\supp \varphi_j \subseteq \overline B_{2j}, j \in \N$ gilt.

  Wir beginnen damit, die Gültigkeit von Gleichung (\ref{eq:GCharacterization}) zu zeigen.
  Da sich der Raum $C_0^\infty(\R^n)$ als Unterraum von $L^2(\R^n)$ auffassen lässt, folgt sogleich $G(\R^n) \supseteq \nabla C_0^\infty(\R^n)$.
  Zudem folgt aus Satz \ref{lem:helmholtz}, dass $G(\R^n)$ als orthogonales Komplement ein bezüglich $\norm{\cdot}_2$ abgeschlossener Unterraum von $L^2(\R^n)^n$ ist.
  Damit gilt auch $G(\R^n) = \overline{G(\R^n)}^{\norm{\cdot}_2} \supseteq \overline{\nabla C_0^\infty(\R^n)}^{\norm{\cdot}_2}$.

  Für die andere Inklusion betrachten wir den Gradienten $\nabla p \in G(\R^n)$.
  Wir wählen Konstanten 
  $$
  K_j := \frac{\int_{G_j} p \d x}{\int_{G_j} \1 \d x}, \quad j \in \N.
  $$
  Aus der \poincare\hyp{}Ungleichung aus Lemma \ref{lem:poincare2} folgt sodann
  \begin{align}
  \norm{p - K_1}_{L^2(G_1)} 
  &\leq C \left(\norm{\nabla (p - K_1)}_{L^2(G_1)^n} + | \int_{G_1} p - K_1 \d x |\right) \nonumber\\
  &= C \left(\norm{\nabla p - \nabla K_1)}_{L^2(G_1)^n} + | \int_{G_1} p \d x  - \int_{G_1} \left(\frac{\int_{G_1} p \d y}{\int_{G_1} \1 \d y} \right) \d x |\right) \nonumber\\
    \label{eq:poincareP}
  &=  C \;\norm{\nabla p}_{L^2(G_1)^n}.
\end{align}
  Mit der Substitutionsregel erhalten wir
\begin{align*}
  \norm{p - K_j}_{L^2(G_j)} 
  &= \left( \int_{G_j} |p(x) - K_j|^2 \d x \right)^\frac{1}{2} \\
  &= \left( \int_{G_1} |p(jy) - K_j|^2 \d y \right)^\frac{1}{2} j^\frac{n}{2} \tag{Substitutionsregel}\\
  &\leq C j^{\frac{n}{2}}\left( \int_{G_1} |\nabla_y p(jy) |^2 \d y \right)^{\frac{1}{2}} \tag{mit Gleichung (\ref{eq:poincareP})}\\
  &= C j^{\frac{n}{2}}j^{-\frac{n}{2}}j \left( \int_{G_j} |\nabla p(x) |^2 \d x \right)^{\frac{1}{2}} \tag{Ketten- u. Substitutionsregel} \\
  &= C j \norm{\nabla p}_{L^2(G_j)^n}
\end{align*}
  für alle $j \in \N$.
  Zusammenfassend gilt also
  \begin{equation}
    \label{eq:gradientInequality}
    \norm{p - K_j}_{L^2(G_j)} \leq j C\norm{\nabla p}_{L^2(G_j)^n}.
  \end{equation}

  Wir definieren nun $p_j := \varphi_j (p - K_j)$.
  Die komponentenweise Anwendung der Produktregel ergibt 
  \begin{equation}
    \label{eq:productrule}
    \nabla p_j =  (\nabla \varphi_j)(p - K_j) + \varphi_j \nabla p,
  \end{equation}
  wobei $\nabla \varphi_j(x) = \frac{1}{j} \nabla \varphi (\frac{x}{j})$ gilt.
  Hieraus erhalten wir unter Verwendung der Konstanten 
  \begin{equation}
    \label{eq:constantCprime}
    C' := \sup_x |\nabla \varphi(x)|
  \end{equation}
  und Gleichung (\ref{eq:gradientInequality}) die folgende Abschätzung:
  \begin{align}
    \norm{\nabla p - \nabla p_j}_{L^2(\R^n)^n}
    &\leq \norm{\nabla p - \varphi_j \nabla p}_{L^2(\R^n)^n} + \norm{\varphi_j \nabla p - \nabla p_j}_{L^2(\R^n)^n} \nonumber\\
    &\hspace{-0.15cm}\underset{(\ref{eq:productrule})}{=} \norm{\nabla p - \varphi_j \nabla p}_{L^2(\R^n)^n}    + \norm{\nabla\varphi_j (p - K_j)}_{L^2(\R^n)^n} \nonumber\\
    &\hspace{-0.15cm}\underset{(\ref{eq:constantCprime})}{\leq} \norm{\nabla p - \varphi_j \nabla p}_{L^2(\R^n)^n} + \frac{C'}{j} \norm{p - K_j}_{L^2(G_j)^n}\nonumber\\
    &\hspace{-0.15cm}\underset{(\ref{eq:gradientInequality})}\leq \norm{\nabla p - \varphi_j \nabla p}_{L^2(\R^n)^n} + C'C \norm{\nabla p}_{L^2(G_j)^n} \label{eq:gradDiff}.
  \end{align}
  Hierbei wurde beim Übergang von $\norm{\cdot}_{L^2(\R^n)^n}$ zu $\norm{\cdot}_{L^2(G_j)^n}$ ausgenutzt, dass nach Konstruktion $\supp \nabla \varphi_j \subseteq G_j$ gilt.

  Aus Gleichung (\ref{eq:propPhi}) folgt $|1 - \varphi_j(x)| = |1 - \varphi(\tfrac{x}{j})| \leq 2$. 
  Die Glieder der Funktionenfolge $\left((1 - \varphi_j) D_k p\right)_{j \in \N}$ besitzen somit für $k = 1, \dots, n$ die integrierbare Majorante $2\, D_k p \in L^2(\R^n)$.
  Mit dem Satz von \lebesgue\ über majorisierte Konvergenz ergibt sich daraus unter Verwendung von Gleichung (\ref{eq:limPhi})
  $$
  \lim_{j \to \infty} \norm{(1 - \varphi_j) \,D_k p}_2 
  = \left( \lim_{j \to \infty} \int_{\R^n} |(1 - \varphi_j)|^2 |\,D_k p|^2 \d x \right)^{\frac{1}{2}}
  = 0,
  $$
  was wiederum
  \begin{equation}
    \label{eq:lebesgueDom}
  \lim_{j \to \infty} \norm{\nabla p - \varphi_j \nabla p}_{L^2(\R^n)^n}
  = \left(\, \sum_{k = 1}^n \lim_{j \to \infty} \norm{(1 - \varphi_j)\, D_k p}_2^2  \right)^\frac{1}{2} = 0
  \end{equation}
  impliziert.
  Ebenso folgt aus dem Satz über majorisierte Konvergenz 
  \begin{equation}
    \label{eq:limitGradient}
  \lim_{j \to \infty} \norm{\nabla p}_{L^2(G_j)^n}
  = 0.
  \end{equation}
  Aus Gleichung (\ref{eq:gradDiff}) folgt nun mit (\ref{eq:lebesgueDom}) und (\ref{eq:limitGradient})
  \begin{equation}
    \label{eq:limNablaP}
    \lim_{j \to \infty} \norm{\nabla p - \nabla p_j}_{L^2(\R^n)^n} = 0.
  \end{equation}

  Nach Konstruktion gilt $\nabla p_j \in L^2(\R^n)^n$.
  Durch Lemma \ref{lem:mollification} lässt sich somit $\nabla p_j$ als Grenzwert von glatten Funktionen mit kompaktem Träger darstellen. Für alle $j \in \N$ existiert daher ein $\varepsilon_j > 0$ mit
  \begin{equation}
    \label{eq:convolutionNablaP}
  \norm{\nabla p_j - \nabla (\mathcal{F}_{\varepsilon_j} \ast p_j)}_{L^2(\R^n)^n}
  \overset{(\ref{eq:convolutionDiff})}{=} \norm{\nabla p_j - \mathcal{F}_{\varepsilon_j} \ast \nabla p_j}_{L^2(\R^n)^n}
  \leq \frac{1}{j}.
  \end{equation}
  Definieren wir nun $\tilde p_j := \mathcal{F}_{\varepsilon_j} \ast p_j$, so gilt $\tilde p_j \in C_0^\infty(\R^n)$, da 
  $$
  \supp p_j = \supp \varphi_j(p - K_j) 
  \subseteq \supp \varphi_j
  \subseteq \overline B_{2j}
  $$
  impliziert, dass $p_j$ einen kompakten Träger besitzt und selbiges auch für $\tilde p_j$ als Faltung zweier Funktionen mit kompaktem Träger gilt.
  Mit Gleichungen (\ref{eq:limNablaP}) und (\ref{eq:convolutionNablaP}) erhält man
  $$
  \lim_{j \to \infty} \norm{\nabla p - \nabla \tilde p_j}_{L^2(\R^n)^n}
  \leq 
  \lim_{j \to \infty} \norm{\nabla p - \nabla p_j}_{L^2(\R^n)^n} + \lim_{j \to \infty} \norm{\nabla p_j - \nabla \tilde p_j}_{L^2(\R^n)^n}
  = 0, 
  $$
  woraus die Behauptung aus Gleichung (\ref{eq:GCharacterization}) folgt.

  Als Nächstes beweisen wir die Charakterisierungsgleichung (\ref{eq:LCharacterization}).
  Dazu sei im Folgenden
  $$
  L := \{ f \in L^2(\R^n)^n \mid \div f = 0 \}.
  $$

  Wir beweisen zunächst die Inklusion $L_\sigma^2(\R^n) \subseteq L$.
  Sei dazu $f \in L_\sigma^2(\R^n)$.
  Nach Definition existiert eine Folge $(f_j)_{j \in \N}$ mit $f_j \in C_{0,\sigma}^\infty(\R^n)$ für alle $j \in \N$ und 
  $$
  \lim_{j \to \infty} \norm{f_j - f}_{L^2(\R^n)^n} = 0.
  $$
  Daraus folgt unter Verwendung der \cauchy\hyp{}\schwartz\hyp{}Ungleichung
  \begin{align*}
    |[\div f, \varphi] |
    &= |[\div f_j - f, \varphi] | \\
    &= |\langle f_j - f, \nabla \varphi \rangle| \\
    & \leq \norm{f_j - f}_{L^2(\R^n)^n} \norm{\nabla \varphi}_{L^2(\R^n)^n} 
    \to 0, \quad j \to \infty,
  \end{align*}
  für alle $\varphi \in C_0^\infty(\R^n)^n$.
  Damit folgt $[\div f, \varphi] = 0$ für alle $\varphi \in C_0^\infty(\R^n)$.
  Es gilt also $\div f = 0$ im distributionellen Sinne und wir erhalten $f \in L$.

  Zum Beweis der umgekehrten Inklusion sei $f \in L$.
  Dann gilt
  $$
  \langle f, \nabla p \rangle = -[\div f, p] = 0
  $$
  für alle $p \in C_0^\infty(\R^n)$.
  Also ist die Abbildung $\langle f, \cdot\rangle$ stetig auf $\nabla (C_0^\infty(\R^n))$ und besitzt somit eine stetige Fortsetzung auf den Abschluss des Definitionsbereichs. 
  Dieser stimmt jedoch nach dem ersten Teil des Beweises mit $G(\R^n)$ überein.
  Daher erfüllt die Fortsetzung die Identität
  $$
  \langle f, \nabla p \rangle = 0 
  $$
  für alle $\nabla p \in G(\R^n)$.
  Dies bedeutet aber gerade, dass $f \in G(\R^n)^\perp$ gilt und aus Satz \ref{lem:helmholtz} zur \helmholtz\hyp{}Zerlegung folgt damit
  $$
  f \in G(\R^n)^\perp 
  = L_\sigma^2(\R^n)^{\perp\perp} 
  = \overline{L_\sigma^2(\R^n)}^{\norm{\cdot}_2}
  = L_\sigma^2(\R^n).
  $$
  Daraus folgt $L \subseteq L_\sigma^2(\R^n)$ und Gleichung (\ref{eq:LCharacterization}) ist bewiesen.
\end{proof}
