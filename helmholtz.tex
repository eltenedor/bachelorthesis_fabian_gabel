\chapter{Helmholtz-Zerlegung in $L^2$}

Nach der in den vorangehenden Kapiteln geleisteten Vorarbeit sind wir nun in der Lage den zentralen Satz dieser Arbeit zu behandeln.
Wir interessieren uns dafür, unter welchen Bedingungen sich die Vektorräume $L^q(\Omega)^n$ in für die Strömungsmechanik interessante Teilräume zerlegen lassen.
Die Darstellung in dieser Arbeit begrenzt sich auf die Zerlegung für $q = 2$.
Hier ermöglicht der in Theorie der \hilbert\hyp{}Räume grundlegende Begriff der Orthogonalität eine orthogonale Zerlegung des Lösungsraumes $L^2(\Omega)^n$.

Wir bezeichnen mit 
$$
G_2(\Omega) := \{f \in L^2(\Omega)^n \mid f = \nabla p \text{ für ein } p \in L^2_{\loc}(\Omega)\}
$$
den Raum aller Funktionen $f \in L^2(\Omega)^n$ die im distributionellen Sinne ein skalares Potential besitzen.

\begin{lem}
  \label{lem:Helmholtz}
  Sei $\Omega \subseteq \R^n$ mit $n \geq 2$ ein beliebiges Gebiet.
  Dann gilt
  $$
  L^2(\Omega)^n = G_2(\Omega) \oplus L^2_\sigma(\Omega).
  $$
\end{lem}

\begin{proof}
  Wir zeigen, dass 
  \begin{equation}
    \label{eq:orthComplement}
    G_2(\Omega) = L_\sigma^2(\Omega)^\perp
  \end{equation}
  gilt
  Die zu zeigende Aussage folgt dann aus dem Fundament der Theorie der \hilbert\hyp{}Räume: dem Satz von der Orthogonalprojektion.

  Sei dazu $f \in L_\sigma^2(\Omega)^\perp$.
  Wir fassen $f$ als Distribution, also als Element von $(C_0^\infty(\Omega)^n)'$ auf.
  Für jedes beschränkte Teilgebiet $\Omega_0 \subseteq \Omega$ mit $\overline\Omega_0 \subseteq \Omega$ folgt für alle $v \in C_0^\infty(\Omega_0)^n$ mit der \poincare\hyp{}Ungleichung
  $$
  |[f, v]| 
  = |\langle f, v \rangle| 
  \underset{\hoelder}{\leq}  \norm{f}_2 \norm{v}_{L^2(\Omega_0)^n}
  \underset{\poincare}{\leq} C \norm{f}_2 \norm{\nabla v}_{L^2(\Omega_0)^{n^2}},
  $$
  mit einer Konstante $C = C(\Omega_0) > 0$.
  Daraus folgt $f \in W_{\loc}^{-1,2}(\Omega)^n$.
  Nach Voraussetzung gilt zudem 
  $$
  [f,v] = \langle f, v \rangle = 0
  $$
  für alle $v \in C_{0, \sigma}^\infty(\Omega)$.
  Die Anwendung des Gradientenkriteriums \ref{lem:gradientCriterion} liefert nun die Existenz einer Funktion $p \in L_{\loc}^2(\Omega)$, welche die Gleichung
  $$
  f = \nabla p
  $$
  im distributionellen Sinne erfüllt.
  Daraus folgt $f \in G_2(\Omega)$.

  Sei umgekehrt $f \in G_2(\Omega)$ mit $f = \nabla p$ für ein $p \in L^2_{\loc}(\Omega)$.
  Dann gilt
  \begin{align*}
  \langle \nabla p, v \rangle
  &= \sum_{j = 1}^n \int_\Omega (D^j p)\;v_j \d x \\
  &= - \sum_{j = 1}^n \int_\Omega p\;(D^j v_j) \d x \\
  &= - \langle p, \div v \rangle  \\
  &= - \langle p, 0 \rangle  \\
  &= 0
  \end{align*}
  für alle $v \in C_{0, \sigma}^\infty(\Omega)$ und da $\nabla p \in L^2(\Omega)$ gilt dies aufgrund der Stetigkeit des Skalarproduktes auch für alle $v \in \overline{C_{0, \sigma}^\infty(\Omega)} = L_\sigma^2(\Omega)$.
  Damit ist $f \in L_{0,\sigma}^2(\Omega)^\perp$.
\end{proof}

Aus Gleichung (\ref{eq:orthComplement}) folgt insbesondere, dass der Raum $G_2(\Omega)$ abgeschlossen bezüglich $\norm{\cdot}_2$ abgeschlossen ist.

Im Falle $\Omega = \R^n$ lassen sich die Räume $G_2(\Omega)$ und $L_\sigma^2(\Omega)$ noch auf eine andere Art charakterisieren.
Dies ist Inhalt des folgenden Lemmas.

\begin{lem}
  Sei $n \in \N$ mit $n \geq 2$.
  Dann gilt
  $$
  L_\sigma^2(\R^n) = \{ f \in L^2(\R^n)^n \mid \div f = 0 \}
  $$
  und $G(\R^n)$ ist der Abschluss des Raumes
  $$
  \nabla C_0^\infty(\R^n) := \{ \nabla p \mid p \in C_0^\infty(\R^n) \}
  $$
  bezüglich der Norm $\norm{\cdot}_{L^2(\R^n)^n}$.
  Es gilt also
  \begin{equation}
    \label{eq:GCharacterization}
    G(\R^n) = \overline{\nabla C_0^\infty(\R^n)}^{\norm{\cdot}_2}.
  \end{equation}
\end{lem}

\begin{proof}
  Im folgenden Beweis wollen wir eine Funktion $\varphi \in C_0^\infty(\R^n)$ mit den Eigenschaften
  $$
  0 \leq \varphi \leq 1
  \quad,\quad
  \varphi(x) = 1 \text{, falls } |x| \leq 1
  \quad,\quad
  \varphi(x) = 0 \text{, falls } |x| \geq 2,
  $$
  betrachten.
  Zudem definieren wir die Funktionen
  $$
  \varphi_j \in C_0^\infty(\R^n)
  \quad,\quad
  \varphi_j(x) := \varphi\left(\frac{x}{j}\right)
  \quad,\quad
  x \in \R^n, j \in \N.
  $$
  Nach Konstruktion folgt $\lim_{j \to \infty} = \varphi_j(x) = 1$ für alle $x \in \R^n$.
  Setzen wir
  $$
  B_j := \{ x \in \R^n \mid |x| < j \} 
  \quad,\quad
  G_j := B_{2j}\setminus \overline B_j,
  $$
  so erhalten wir $\nabla \varphi_j \subseteq \overline G_j$, da nach Definition $\varphi_j$ auf $B_j$ konstant ist sowie $\supp \varphi_j \subseteq \overline B_{2j}, j \in \N$.

  Wir beginnen damit, die Güligkeit der Gleichung (\ref{eq:GCharacterization}) zu zeigen.
  Da sich der Raum $C_0^\infty(\R^n)$ als Unterraum von $L^2(\R^n)$ auffassen lässt folgt sogleich $G(\R^n) \supseteq \nabla C_0^\infty(\R^n)$.
  Zudem folgt aus Lemma \ref{lem:Helmholtz}, dass $G(\R^n)$ als orthogonales Komplement ein bezüglich $\norm{\cdot}_2$ abgeschlossener Unterraum von $L^2(\R^n)^n$ ist.
  Damit gilt auch $G(\R^n) = \overline{G(\R^n)}^{\norm{\cdot}_2} \supseteq \overline{\nabla C_0^\infty(\R^n)}^{\norm{\cdot}_2}$.

  Für die andere Inklusion betrachen wir den Gradienten $\nabla p \in G(\R^n)$.
  Wir wählen Konstanten $K_j = \int_{G_j} p \d x, j \in \N$.
  Aus der \poincare\hyp{}Ungleichung \cite[S.44, Lemma 1.1.2]{sohr2001navier} folgt sodann
  \begin{equation}
  \norm{p - K_1}_{L^2(G_1)} \leq C \left(\norm{\nabla (p - K_1}_{L^2(G_1)^n} + | \int_{G_1} p - K_1 \d x |\right) =  C \norm{\nabla p}_{L^2(G_1)^n}.
\end{equation}
  Mit der Substitutionsregel erhalten wir
\begin{align*}
  \norm{p - K_j}_{L^2(G_j} 
  &= \left( \int_{G_j} |p(x) - K_j|^2 \d x \right)^\frac{1}{2} \\
  &= \left( \int_{G_1} |p(jy) - K_j|^2 \d y \right)^\frac{1}{2} j^\frac{n}{2} \\
  \leq 
\end{align*}<++>

  
\end{proof}
