% Vorlage fuer Abschlussarbeiten am Fachbereich Mathematik der TU Darmstadt.
% Geeignet fuer Bachelorarbeiten, Masterarbeiten und Diplomarbeiten.

\documentclass[11pt,a4paper,twoside]{report}
% Benutzt man die Option draft, so kann man die Umbrueche ueberpruefen.

% Hier werden alle benoetigten Pakete und Einstellungen geladen. Auch
% hier sind Sie frei diese direkt in der Praeambel zu laden.


\usepackage[ngerman]{babel}

% Kodierung fuer unixoide Systeme und Windowssysteme.

\usepackage[utf8]{inputenc}

% Schriftart Times New Roman.

\usepackage{lmodern}

% Erweitert den Zeichenvorrat, so dass z.B. auch Umlaute im PDF-Dokument
% gefunden werden.

\usepackage[T1]{fontenc}

% Zur Einbindung von Bildern.

\usepackage{graphicx}

% Erweiterte enumerate-Umgebung.

\usepackage{enumerate}

% Verschiedene Pakete, die nuetzlich sind um Mathematik in LaTeX zu setzen.

\usepackage{amsmath, amssymb, amsthm, dsfont}

% Doppelte 1 (\mathbb{1})

\usepackage{bbold}

% Indentation

\usepackage{scrextend}

% Spaces for newcommand

\usepackage{xspace}

% tikz
\usepackage{tikz}
%\usetikzlibrary{arrows.meta}% arrow tip library
%\usetikzlibrary{bending}% better arrow head for bended lines

% Hyphenation of words that already contain a hyphen
% http://tex.stackexchange.com/questions/2706/adequate-hyphenation-of-words-already-containing-a-hyphen
% Example:
% \textsc{Alexandroff}\hyp{}\textsc{Urysohn}\hyp{}Kompaktheit

\usepackage{hyphenat}

% Hyperlinks

%\usepackage{hyperref}
\usepackage[pdfauthor={Fabian Gabel},%
pdftitle={Bachelorarbeit - Die Helmholtz-Zerlegung in L2},%
%pagebackref=true,%
%pdftex
]{hyperref}


%\usepackage[
%%nonumberlist, %keine Seitenzahlen anzeigen
%toc,          %Einträge im Inhaltsverzeichnis
%section]      %im Inhaltsverzeichnis auf section-Ebene erscheinen
%{glossaries}

%%%%%%%%%%%%%%%%
% Seitenlayout %
%%%%%%%%%%%%%%%%

% DIV# gibt den Divisor für die Layoutberechnung an.
% Vergrößern des Divisors vergrößert den Textbereich.
% BCOR#cm gibt die Breite des Bundstegs an.
\usepackage[DIV14,BCOR2cm]{typearea}

% Abstand obere Blattkante zur Kopfzeile ist 2.54cm - 15mm
\setlength{\topmargin}{-15mm}

% Keine Einrueckung nach einem Absatz.

%\parindent 0pt

% Abstand zwischen zwei Abs\"atzen.

%\parskip 12pt

% Zeilenabstand.

\linespread{1.25}

% Inhaltsverzeichnis erstellen.

\usepackage{makeidx}
\makeindex
\usepackage{fancyhdr}


% Hier werden Makros und Umgebungen eingebunden. Diese werden separat
% in der Datei befehle.tex definiert. Sie sind frei diese Befehle auch
% direkt in der Praeambel zu definieren.
% Makros

\newcommand{\N}{\mathbb{N}} % natuerliche Zahlen
\newcommand{\Z}{\mathbb{Z}} % ganze Zahlen
\newcommand{\Q}{\mathbb{Q}} % rationale Zahlen
\newcommand{\R}{\mathbb{R}} % reelle Zahlen
\newcommand{\K}{\mathbb{K}} % Körper
\newcommand{\co}{\mathrm{co}} % compact open
\newcommand{\W}{\mathcal{W}} % Weyl-Gruppe

\DeclareMathOperator{\Ret}{Re} % Realteil
%\newcommand{\PIT}{\textbf{PIT }} % Prime Ideal Theorem
\newcommand{\AC}{\textbf{AC}\xspace} % Prime Ideal Theorem
\newcommand{\UFT}{\textbf{UFT}\xspace} % Prime Ideal Theorem
\newcommand{\PIT}{\textbf{PIT}\xspace}
\newcommand{\ZF}{\textbf{ZF}\xspace}
\newcommand{\ZFC}{\textbf{ZFC}\xspace}
\newcommand{\CC}{\textbf{CC}\xspace}
\newcommand{\CCR}{\textbf{CC}($\R$)\xspace}
\newcommand{\PCC}{\textbf{PCC}\xspace}
\newcommand{\PCCR}{\textbf{PCC}($\R$)\xspace}

\newcommand{\grad}[1]{\gv{\nabla} #1} % for gradient
\let\divsymb=\div % rename builtin command \div to \divsymb
\renewcommand{\div}[1]{\mathrm{div\,} #1} % for divergence
\renewcommand{\d}[1]{\ensuremath\, {\operatorname{d}\!{#1}}}
\newcommand{\norm}[1]{\lVert #1 \rVert}

\newcommand{\banach}{\textsc{Banach}}
\newcommand{\cauchy}{\textsc{Cauchy}}
\newcommand{\euklid}{\textsc{Euklid}}
\newcommand{\fubini}{\textsc{Fubini}}
\newcommand{\hahn}{\textsc{Hahn}}
\newcommand{\helmholtz}{\textsc{Helmholtz}}
\newcommand{\hoelder}{\textsc{Hölder}}
\newcommand{\killing}{\textsc{Killing}}
\newcommand{\lebesgue}{\textsc{Lebesgue}}
\newcommand{\leibniz}{\textsc{Leibniz}}
\newcommand{\lie}{\textsc{Lie}}
\newcommand{\lipschitz}{\textsc{Lipschitz}}
\newcommand{\poincare}{\textsc{Poincaré}}
\newcommand{\sobolev}{\textsc{Sobolev}}
\newcommand{\weyl}{\textsc{Weyl}}



% Umgebungen für Definitionen, Sätze, usw.

%\newtheorem{defn}{Definition}[section]
%\newtheorem{ex}{Beispiel}[chapter]

\theoremstyle{plain}
\newtheorem{thm}{Satz}[chapter]
\newtheorem{lem}[thm]{Lemma}
\newtheorem{cor}[thm]{Korollar}
\newtheorem{prop}[thm]{Proposition}

\theoremstyle{definition}
\newtheorem{defn}[thm]{Definition}

\theoremstyle{remark}
\newtheorem*{bem}{Bemerkung}

\def\Satzrefname{Satz}

\DeclareMathOperator{\spn}{span}
\DeclareMathOperator{\sign}{sign}
\DeclareMathOperator{\GL}{GL}
\DeclareMathOperator{\Sym}{Sym}
\DeclareMathOperator{\hgt}{ht}
\DeclareMathOperator{\dist}{dist}
\DeclareMathOperator{\loc}{loc}

%\renewcommand{\refname}{Literaturverzeichnis}
\addto\captionsngerman{%
\renewcommand*{\bibname}{Literaturverzeichnis}
}

%\makeglossaries

\pagestyle{fancy}
\fancypagestyle{MyStyle}{%
\fancyhf{} %Clean headers
\fancyhead[LE,RO]{\thepage}
\fancyhead[RE]{\normalfont \nouppercase{\leftmark}}
\fancyhead[LO]{\normalfont \nouppercase{\rightmark}}
\fancyfoot[C]{\tiny Version vom \today}
\renewcommand{\chaptermark}[1]{\markboth{\thechapter\ {\normalfont{##1}}}{}}
\renewcommand{\sectionmark}[1]{\markright{\thesection\ {\normalfont{##1}}}{}}
}

\fancypagestyle{plain}{%
\fancyhf{} % clear all header and footer fields
\fancyfoot[LE,RO]{\thepage} % except the center
\renewcommand{\headrulewidth}{0pt}
\renewcommand{\footrulewidth}{0pt}}


\begin{document}
% Auf der Titelseite und im Inhaltsverzeichnis sollen keine
% Seitenzahlen dargestellt werden.
\pagestyle{empty}

% Hier wird die Titelseite eingebunden.
\begin{titlepage}
  \begin{center}
    \vspace{1cm}
    \includegraphics[width=0.5\linewidth]{TU_Darmstadt_Logo.pdf}
    \vspace{1cm}
    
    \large{Fachbereich Mathematik}
    \vspace{2.5cm}
    
    \large{Bachelorarbeit}
    \vspace{2cm}

    \huge{Die Helmholtz-Zerlegung in $L^2$}
    
    \vspace*{3cm}    
    
		\large
                %\href{mailto:gabel@mathematik.tu-darmstadt.de}{Fabian Gabel}
                Fabian Gabel
    \vspace*{1.0cm}

    10.10.2016 \\
    \vspace*{2cm}

    Betreuer: PD Dr. Robert Haller-Dintelmann

    \vspace*{.5cm}

    %Zweiter Gutachter: Name des zweiten Gutachters\\[2ex]
    \vspace*{\fill}
    %\tiny{Version vom \today}
  \end{center}
\end{titlepage}
\vspace*{\fill}


% Inhaltsverzeichnis erstellen.
\tableofcontents

% Ab sofort werden Seitenzahlen in der Kopfzeile angezeigt.
%\pagestyle{headings}
%\pagestyle{MyStyle}

\chapter*{Einleitung}\index{Einleitung}
\addcontentsline{toc}{chapter}{Einleitung}\index{Einleitung}

\begin{addmargin}[2em]{2em}% 1em left, 2em right
  \textit{... as Sir Cyril Hinshelwood has observed ... fluid dynamicists
were divided into hydraulic engineers who observed things that
could not be explained and mathematicians who explained things
that could not be observed.} 
  \flushright(James Lighthill)
\end{addmargin}
\vspace{1.5cm}

Die akkurate Modellierung des Verhaltens \newton scher Fluide ist zentral für unzählige Anwendungen der Aerodynamik, Verbrennungsforschung oder chemischen Industrie.
Die Grundlage dafür bildet ein System partieller Differentialgleichungen, welches \navier\ und \stokes\ unabhängig voneinander in der ersten Hälfte des 19. Jahrhunderts einführten. 
Man bezeichnet dieses Gleichungssystem heute als \navier\hyp\stokes\hyp{}Gleichungen.
Sie umfassen im inkompressiblen dreidimensionalen Fall eine Gleichung zur Beschreibung der Massenerhaltung, die sogenannte Kontinuitätsgleichung, und für jede Raumrichtung eine Impulsgleichung:

\begin{align*}
  \div u &= 0, \\
  \frac{\partial u}{\partial t} - \nu \Delta u + u \cdot \nabla u + \nabla p &= f.
\end{align*}

Die Existenz und Eindeutigkeit klassischer glatter Lösungen dieses Gleichungssystems gehört nach dem \textsc{Clay}\hyp{}Institute zu einem der wichtigsten ungelösten mathematischen Probleme unseres Jahrtausends und ist daher in der Liste der Millenium Pobleme zu finden.

In der Lösungstheorie der \navier\hyp\stokes\hyp{}Gleichungen aber auch partieller Differentialgleichungen im Allgemeinen hat es sich als nützlich erwiesen, auf der Suche nach klassischen Lösungen einen Umweg einzuschlagen und zunächst die Existenz sogenannter schwacher Lösungen nachzuweisen.
In einem zweiten Schritt wird dann überprüft, ob die gefundenen schwachen Lösungen über zusätzliche Regularitätseigenschaften verfügen, welche die Existenz einer Lösung im klassischen Sinne garantieren.

Im Rahmen der Suche nach schwachen Lösungen im \hilbert\hyp{}Raum $L^2(\Omega)^n$, wobei $\Omega$ ein Teilgebiet des $\R^n$, $n \geq 2$, bezeichne, findet auch das in dieser Arbeit vorgestellte Hilfsmittel Verwendung: die \helmholtz\hyp{}Zerlegung.
Dabei handelt es sich um eine orthogonale Zerlegung des Lösungsraumes, welche es unter anderem ermöglicht den neben dem gesuchten Geschwindigkeitsfeld $u$ unbekannten Druck $p$ vorerst aus dem Gleichungssystem zu eliminieren und so die Anzahl der Unbekannten zu reduzieren.
Die Anwendung der \helmholtz\hyp{}Zerlegung auf ein Element des Lösungsraumes zerlegt dieses in einen divergenzfreien und einen rotationsfreien Anteil.

Ziel dieser Arbeit ist es, aufbauend auf den im Grundstudium vermittelten Kenntnissen der Funktionentheorie, Integrationstheorie und Funktionalanalysis, die Existenz und Eindeutigkeit der \helmholtz\hyp{}Zerlegung auf dem Raum $L^2(\Omega)^n$ zu beweisen.

Das erste Kapitel dient dazu die nötigen funktionalanalytischen Grundlagen bereitzustellen und die in der Arbeit verwendete Notation einzuführen.
Das Hauptaugenmerk dieses Kapitels liegt auf der Definition der schwachen Differenzierbarkeit und des Distributionsbegriffs.

Im zweiten Kapitel werden die zum Beweis der \helmholtz\hyp{}Zerlegung nötigen Hilfsaussagen bereitgestellt.
Das zentrale Resultat dieses Teils der Arbeit ist ein Kriterium, welches unter gewissen Bedingungen die schwache Lösbarkeit der Gradientengleichung $\nabla p = f$ sicherstellt.

Kapitel drei behandelt schließlich die \helmholtz\hyp{}Zerlegung $$L^2(\Omega)^n = L^2_\sigma(\Omega) \oplus G_2(\Omega)$$ und eine Charakterisierung dieser Zerlegung für den Fall, dass $\Omega = \R^n$ gilt.

Das letzte Kapitel fasst nochmals die zentralen Resultate zusammen und gibt einen Ausblick auf Problemstellungen, in denen die \helmholtz\hyp{}Zerlegung zum Einsatz kommt.

\chapter{Grundlagen}
\label{cp:grundlagen}

\section{Physikalische Grundlagen}

\begin{itemize}
  \item Physikalische Motivation dieses Gleichungssystems
  \item Linearisierung der Navier-Stokes-Gleichungen (Motivation der Stokes-Gleichung)
  \item Schleichende Strömungen z.B. \cite{spurk10stroemungslehre}[S.112,S.489]. 
\end{itemize}

\section{Funktionalanalytische Grundlagen -- Distributionen und Sobolevräume}

Dieses Unterkapitel beschäftigt sich mit den für die kommenden Kapitel zentralen Funktionenräumen und dient zudem der Einführung der verwendeten Notation und Bildung der verwendeten Begriffe.
Die Notation ist an \cite{sohr2001navier} angelehnt.

%\subsection{Theorie des Hilbertraums}
%Hauptsächlich zur Einführung der Verwendeten Notation
\subsection{Glatte Funktionen und Glättungskerne}
\label{subsec:smoothMolly}

Ziel dieses Abschnittes ist es die nötigen Begriffe und Definitionen im Zusammenhang mit glatten Funktionen bereitzustellen.
Im Folgenden bezeichne $\Omega \subseteq \R^n$ stets ein nichtleeres Gebiet.

Sei $k \in \N$ so bezeichne $C^k(\Omega)$ die Menge aller Funktionen
$$
u \colon \Omega \to \R, \quad x \mapsto u(x),
$$
sodass alle partiellen Ableitungen $D^\alpha u$ existieren und stetig sind für alle Multiindices $\alpha \in N_0^n$ mit $0 \leq |\alpha| \leq k$. Mit
$$
  C^\infty(\Omega) := \bigcap_{k = 0}^\infty C^k(\Omega)
$$
bezeichnen wir den Raum der glatten Funktionen auf $\Omega$.

Im Kontext von $L^p$-Räumen spielt ein Funktionenraum eine wichtige Rolle bei der Approximation von $L^p$-Funktionen: der Raum der glatten Funktionen mit kompaktem Träger
$$
C_0^\infty := \{ u \in C^\infty(\Omega) \mid \supp u \text{ kompakt und } \supp u \subseteq \Omega\}.
$$
Wir werden zudem den Raum $C^\infty(\overline\Omega)$ aller Restriktionen $u|_{\overline\Omega}$ von Funktionen aus $C^\infty(\Omega)$ mit
\begin{displaymath}
  \sup_{|\alpha| < \infty, x \in \R^n} |D^\alpha u(x)| < \infty \tag{$\ast$}
\end{displaymath}
benötigen.
Unter der Voraussetzung aus ($\ast$) lässt sich der Raum $C^\infty(\overline\Omega)$ mit einer Norm ausstatten:
$$
\norm{u}_{C^\infty} = \norm{u}_{C^\infty(\overline\Omega} := \sup_{|\alpha|\leq k, x \in \overline\Omega} |D^\alpha u(x)|.
$$

Alle eingeführten Räume lassen sich auf natürliche Weise auf Räume von Vektorfeldern verallgemeinern.
Man erhält so
\begin{align*}
  C^\infty(\Omega)^m &:= \{(u_1,\dots,u_m) \mid u_j \in C^\infty(\Omega), j = 1,\dots,m\} \\
  C_0^\infty(\Omega)^m &:= \{(u_1,\dots,u_m) \mid u_j \in C_0^\infty(\Omega), j = 1,\dots,m\} \\
  C^\infty(\overline\Omega)^m &:= \{(u_1,\dots,u_m) \mid u_j \in C^\infty(\overline\Omega), j = 1,\dots,m\} 
\end{align*}
wobei der letzte Vektorraum durch die Norm
$$
\norm{u}_{C^\infty} = \norm{u}_{C^\infty(\overline\Omega)}^m := \sup_{j = 1,\dots, m} \norm{u_j}_{C^k(\overline\Omega)}
$$
zu einem normierten Vektorraum wird.

Von besonderer Rolle für die Lösungstheorie der stationären inkompressiblen \navier\hyp{}\stokes\hyp{}Gleichungen ist der Raum
$$
C_{0,\sigma}^\infty(\Omega) := \{u \in C_0^\infty(\Omega)^n \mid \div u = 0\}
$$
der divergenzfreien Vektorfelder.

\begin{itemize}
  \item \cite{sohr2001navier}[Die Glättungsmethode S.64ff.]
\end{itemize}

\subsection{Topologisierung des Raums der Testfunktionen und ein Stetigkeitsbegriff}

\begin{itemize}
  \item Inhalte ganz zu Beginn von \cite{sohr2001navier}[S.34] wiedergeben, zusätzliche (topologische Eigenschaften) beweisen, aus Werner s.u.
  \item \cite{werner2011fa}[S.430]
  \item Lemma VIII.5.1 (a)(d), VIII.2.3
  \item Satz VIII.5.4(iii)
  \item lokale Integrierbarkeit
  \item Einbettung von $L^1_{\mathrm{loc}}$ in $C_0^\infty(\Omega)'$
\end{itemize}

\subsection{Differentiation von Distributionen -- Schwache Differenzierbarkeit und Sobolevräume}

\begin{itemize}
  \item \cite{sohr2001navier}[S.34ff.]
  \item \cite{werner2011fa}[S.433f.]
  \item Differentiation von Distributionen
  \item Divergenzfreie Test-Funktionen
  \item Sobolevräume und ihre Normen \cite{sohr2001navier}[S.38ff.]
\end{itemize}




\cleardoublepage
%\nocite{*}
\bibliographystyle{geralpha}
\bibliography{bachelorthesis_fabian_gabel}
%\addcontentsline{toc}{chapter}{Literaturverzeichnis}
\addcontentsline{toc}{chapter}{Literaturverzeichnis}


\cleardoublepage
\renewcommand{\glossarysection}[2][\theglstoctitle]{%
    \def\theglstoctitle{#2}%
      \setcounter{secnumdepth}{-1}%
        \chapter[#1]{#2}%
}
\printglossary[title=Symbolverzeichnis,toctitle=Symbolverzeichnis]
      

% Keine Kopf-/Fußzeile auf dieser Seite
\thispagestyle{empty}

\vspace*{4cm}
\section*{Erklärung}\index{Erklärung}

Hiermit versichere ich, dass ich die vorliegende Arbeit selbstständig verfasst habe und alle benutzten Quellen einschließlich der Quellen aus dem Internet und alle sonstigen Hilfsmittel angegeben habe.\vspace{20pt}

\noindent
Darmstadt, den 10.10.2016\vspace{60pt}

% Unterschrift (handgeschrieben)

\noindent
Fabian Gabel



% Keine Kopf-/Fußzeile auf dieser Seite
\thispagestyle{empty}

\vspace*{4cm}
\section*{Erklärung}\index{Erklärung}

Hiermit versichere ich, dass ich die vorliegende Arbeit selbstständig verfasst habe und alle benutzten Quellen einschließlich der Quellen aus dem Internet und alle sonstigen Hilfsmittel angegeben habe.\vspace{20pt}

\noindent
Darmstadt, den 10.10.2016\vspace{60pt}

% Unterschrift (handgeschrieben)

\noindent
Fabian Gabel



\end{document}
